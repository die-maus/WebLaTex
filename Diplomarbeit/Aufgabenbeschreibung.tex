\chapter{Aufgabenbeschreibung}

Eine wesentliche Teilaufgabe bei der Analyse eines Textdokuments ist die Segmentierung
des Dokumentbildes in Textbl"ocke ({\em Zoning\/}). Wenn die Textbl"ocke bestimmt sind, k"onnen
die Zeilen, W"orter und die Schriftzeichen in diesem Textblock synthetisiert und somit
auch klassifiziert werden.

Die Aufgabe der Diplomarbeit besteht darin, die in der Literatur bekannten Verfahren
zur Segmentierung in Textbl"ocke zu sichten (\KapRef{LayoutanalyseMethoden}), geeignete
Verfahren f"ur die zu bearbeiteten Dokumenttypen (Anhang \ref{DokuKlassen}) 
auszuw"ahlen, zu modifizieren und geeignet zu kombinieren.
Die Erkennungsleistung des entwickelten Segmentierungsverfahrens und deren Parameter
sollen an einer Lernstichprobe optimiert und an einer Teststichprobe deren Leistungsf"ahigkeit
und Grenzen aufgezeigt werden (\KapRef{ErgebnisZoning}).

Die Basis der zu entwickelnden Verfahren bilden Zu\-sammen\-hangs\-objekte
(\KapRef{Zusammenhangsanalyse}), die das Bin"arbild durch Konturen dieser Gebiete kodieren.
Diese Vielzahl dieser primitiven Objekte -- von 1000 bis weit "uber 10000 -- mu"s
zu Textbl"ocken mit homogenen Objekten zusammengefa"st werden. Homogenit"at bezieht
sich dabei auf die Eigenschaften, die sich aus den Zusammenhangsobjekten ableiten
lassen (\KapRef{Homogenitaet}). Insbesondere soll mehrspaltiges Layout unterst"utzt und die
nat"urliche Abfolge der Textbl"ocke ermittelt werden (\KapRef{Lesefolge}). 

Die entwickelten Algorithmen werden in die bestehende Entwicklungsumgebung
(NeXT Workstation unter einer grafischen Programmierumgebung) eingebunden 
(Anhang \ref{Experimentierumgebung}),
die die Entwicklung von Verfahren durch eine Vielzahl von bestehenden Software--Moduln unterst"utzt.

\newpage
{\large \bf Anforderungen}\label{Anforderungen}

Folgende Anforderungen sollen an die zu erarbeitenden Methoden gestellt werden:

\begin{itemize}

  \item Hauptanwendung: Gesch"aftsbriefe aus dem Umfeld {\em Investor Relations\/}
        (Anfragen bez"uglich Gesch"aftsberichte), wissenschaftliche 
	Ver"offentlichungen, Zeitschriften, Tageszeitungen und Steuerbescheide.
	Typische Beispiele dieser Dokumentklassen siehe Anhang \ref{DokuKlassen}.

  \item Mehrspaltiges Layout mit beliebigem Konturverlauf.

  \item Dokumente beinhalten ausschlie"slich Textgebiete und sind im Winkel horizontal
  ausgerichtet.
  
  \item Dokumente sind dem romanischen Sprachraum entnommen, d.h.\ ihre Schreibrichtung ist
        von links nach rechts, und von oben nach unten.
  
  \item Das zu entwickelnde Verfahren soll nur layoutunabh"angige oder selbstadaptierende
  Parameter verwenden.
	
  \item Transparenz der Parameter, d.h\ alle (soweit sinnvoll) Parameter sollen dem Benutzer
        zug"anglich und vor Beginn der Analyse ver"anderbar sein.
	
  \item Integration in die bestehende Entwicklungs-- bzw.\ Experimentierumgebung.
  
\end{itemize}
