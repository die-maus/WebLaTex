\chapter{Zusammenfassung}

In dieser Diplomarbeit wurde ein Verfahren vorgestellt, welches ein Dokumentenbild
in Textbl"ocke segmentiert. Die Textblockdefinition basiert
auf den geometrischen Eigenschaften der in r"aumlicher N"ahe stehenden Zusammenhangsobjekte.
Durch diese Textblocksegmentierung wurde erreicht, da"s nachfolgende
Segmentierungsschritte (Zeilen, W"orter und Buchstaben) in ihrer Erkennungsleistung
verbessert wurden, da ihnen nun als Eingangsdaten Textbl"ocke homogenen Inhalts zur Verf"ugnug
stehen.

Die verwendeten Methoden machen keine Einschr"ankungen bez"uglich der Layoutstruktur
und der verwendeten
Dokumentklasse. Die jeweiligen Parameter sind layoutunabh"angig bzw.\ passen sich 
automatisch dem Dokument an. Mehrspaltige Layouts mit beliebiger Kontur der Textbl"ocke
werden ber"ucksichtigt. Es wird vorausgesetzt, da"s die Dokumente nur 
Textgebiete enthalten, eine Winkelkorrektur bereits vorgenommen wurde und da"s sie aus dem
romanischen Sprachraum entstammen, d.h.\ die Schreibrichtung verl"auft von links nach rechts und von
oben nach unten.

Das Verfahren kn"upft an zwei bekannte Vorgehensweisen an: 
Der Top--Down Ansatz (XY--Cut) versucht das Dokument rekursiv, entlang wei"sem 
Zwischenraum zu trennen, und der Top--Down Ansatz (Smearing) 
verbindet kleine Objekte zu gr"o"seren Bereichen. Um die Smearing Parameter unabh"angig von dem
verwendeten Layout zu halten, werden sie, f"ur jedes beim XY--Cut entstandenen Gebiet separat,
"uber den Zeilenabstand direkt aus dem Dokument berechnet. Das Smearing wird mit der
morphologischen Operation Dilatation mit anschlie"sender Erosion (Closing) durchgef"uhrt. Um
die Datenmenge bei diesem nichtlinearen Filter zu reduzieren, wird zuvor das Bin"arbild in
seiner Aufl"osung um den Faktor acht in der Vertikalen und Horizontalen reduziert.

Im Gegensatz zu bisherigen Verfahren wird in diesem Ansatz die Konsistenz der Textbl"ocke "uberpr"uft: 
Textbl"ocke sind definiert als eine Menge von homogenen Zusammenhangsgebieten. Da die
morphologischen Operationen alle Objekte verbinden, die in r"aumlicher N"ahe stehen, mu"s die
Homogenit"at innerhalb der hypothetischen Textbl"ocke "uberpr"uft werden. Hierbei wird
die mittlere Buchstabengr"o"se und --st"arke jeder Textzeile bestimmt und mit der ihr folgenden
verglichen und ggf.\ der Textblock zwischen den zwei Zeilen getrennt.

In einem letzten Segmentierungsschritt wird versucht Textbl"ocke zu verbinden, um eine m"oglichst
kleine Menge von Textbl"ocken zu erhalten. Dabei wird die Graphenstruktur des Minimum Spanning
Tree nach der Methode von Kruskal aufgebaut und anschlie"send Kante f"ur Kante abgearbeitet.
Dieser Schritt wird solange wiederholt, bis ein station"arer Zustand erreicht ist
(Relaxationsmethode).

Im Anschlu"s werden die gefundenen Textbl"ocke nach ihrer logischen Lesefolge sortiert. Dabei
wird eine Baumstruktur aufgebaut, die die hierarchische Struktur der Textbl"ocke
beinhaltet. Bei der Sortierung werden Liniensegmente, wei"se horizontale Zwischenr"aume und Textrahmen ber"ucksichtigt.
Die Lesefolge ergibt sich aus der Pre--Order Traversion der Baumstruktur.

Das Verfahren wurde anhand einer Lernstichprobe von 47 Dokumenten sowie an einer
Teststichprobe von 116 Dokumenten, bei der die Leistungsf"ahigkeit und Grenzen aufgezeigt wurden, 
erarbeitet.
Die implementierten Methoden zeigen konstant gute Ergebnisse (Erkennungsrate 96\%) 
bei den verschiedensten Layouts (Gesch"aftsbriefe, wissenschaftliche Artikel, Zeitschriften,
Tageszeitungen und Steuerbescheide). 
Probleme bereiteten St"orungen durch Nicht--Textelemente. 
Die L"osung dieses Problems w"are ein vorgeschaltetes 
Text--Grafik--Unterscheidungs Modul, 
wobei bis zu 5\% h"ohere Erkennungsraten erreicht werden k"onnten.
Die Laufzeit der kompletten Textblocksegmentierung (XY--Cut, Smearing, Konsistenzpr"ufung und
Sortierung) ist anh"angig von der Anzahl der Zusammenhangsgebiete und 
betr"agt zwischen ca.\ 2 Sekunden f"ur einen Gesch"aftsbrief und ca.\ 11 Sekunden 
f"ur DIN--A4--gro"se Ausschnitte aus Tageszeitungen (gemessen auf einer HP--PA Workstation).

Die Algorithmen wurden in ANSI--C implementiert und in eine bestehende Softwareumgebung
integriert. 


