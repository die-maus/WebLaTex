\thispagestyle{empty}
%
\BildPur{UniUlmLogoScanX}{4}
%
\begin{center}
  \Large\bf
  Bestimmung von Textbl"ocken in strukturierten
  Dokumenten auf der Basis von Zusammenhangsgebieten
\end{center}

\vfill

\begin{center}
  \large
  Diplomarbeit\\
  von\\
  Reiner Kr"autle
\end{center}

\vfill

\begin{center}
  Abteilung Allgemeine Elektrotechnik und Mikroelektronik\\
  Universit"at Ulm\\
  August 1995
\end{center}

\newpage\thispagestyle{empty}

\begin{center}
  \large
  Abteilung Allgemeine Elektrotechnik und Mikroelektronik\\
  Universit"at Ulm
\end{center}
\vfill

\centerline{09.\ Februar bis 08.\ August 1995}
\vfill

\begin{center}
  \begin{large}
    \begin{tabular}{rl}
      Bearbeiter:                & Reiner Kr"autle              \\[5mm]
      Pr"ufer:                   & Prof.~Dr.--Ing.~A.~Rothermel \\[5mm]
      Betreuer in der Industrie: & Dr.~Thomas Bayer,            \\
                                 & Daimler--Benz AG, F3M/T
    \end{tabular}
  \end{large}
\end{center}

\clearpage{\thispagestyle{empty}\cleardoublepage}
\thispagestyle{empty}
\begin{small}
  \centerline{\bf Kurzfassung}

  Es wird ein Verfahren zur Segmentierung des Dokumentbildes in Textbl"ocke ausgehend
  von Zusammenhangsobjekten beschrieben.
  Voraussetzung ist, da"s die Dokumente nur maschinengeschriebene Textgebiete
  enthalten, eine Winkelkorrektur bereits vorgenommen wurde und da"s sie aus dem romanischen
  Sprachraum entstammen. Bez"uglich der Layoutstruktur wird keine Einschr"ankung gemacht und die
  Verfahren werden automatisch mit den vorliegenden Daten parametriert.

  Textbl"ocke werden definiert als eine Menge von homogenen Zusammenhangsgebieten. Sie sind
  charakterisiert durch die mittlere Buchstabengr"o"se und --st"arke, das umschreibende Rechteck
  und der Position innerhalb der logischen Lesefolge.

  Das Verfahren basiert auf einer Kombination der Top--Down--Analyse (XY--Cut) und der
  Bottom--Up--Analyse (Subsampling und Smearing).
  Die ben"otigten Smearing--Parameter werden "uber den Zeilenabstand lokal
  (innerhalb der bei den XY--Cuts entstandenen Gebieten) aus dem Dokument berechnet.
  Die Textblockhypothesen werden in den folgenden beiden Schritten bez"uglich zweier
  Homogenit"atskriterien "uberpr"uft: Textblockhypothesen werden aufgespalten, wenn die
  mittlere Buchstabenh"ohe oder --st"arke in folgenden Zeilen stark variiert;
  Textblockhypothesen werden zusammengefa"st, wenn r"aumlich benachbarte homogene
  Textblockhypothesen in vorigen Schritten getrennt wurden.
  Im Anschlu"s werden die gefundenen Textbl"ocke unter Ber"ucksichtigung ihrer geometrischen Struktur nach ihrer logischen Lesefolge sortiert.

  Das Stichprobenmaterial umfa"st 163 Dokumente aus den Klassen der Gesch"aftsbriefe,
  wissenschaftlichen Artikel, Zeitschriften, Tageszeitungen und Steuerbescheide,
  aufgeteilt in eine Trainingsstichprobe von 47 Dokumenten,
  an derer die Verfahren erarbeitet und parametriert wurden, und in eine Teststichprobe
  von 116 Dokumenten, an derer die Leistungsf"ahigkeit und Grenzen aufgezeigt wurden.
  Die Verfahren zeigen gute bis sehr gute Ergebnisse bei den betrachteten Dokumentklassen.
  Die Erkennungsrate (Verh"altnis richtig erkannter Textbl"ocke zu der Gesamtanzahl der erkannten
  Textbl"ocke)
  liegt f"ur alle Dokumentklassen im Mittel zwischen 90 und 95\%.
\end{small}
\vfill
{\bf Erkl"arung}

Hiermit erkl"are ich, da"s ich diese Diplomarbeit selbstst"andig verfa"st und keine anderen
als die angegebenen Quellen und Hilfsmittel verwendet habe.

\vspace*{2cm}

Ulm, den 08.\ August 1995\hfill
\begin{minipage}[t]{5cm}
  \centerline{Reiner Kr"autle}
  \centerline{\small (Matr.~Nr.~223875)}
\end{minipage}
\hspace*{\fill}
\clearpage{\thispagestyle{empty}\cleardoublepage}
