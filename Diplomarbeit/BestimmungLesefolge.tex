\chapter{Bestimmung der logischen Lesefolge}\label{Lesefolge}

Die Verfahren aus Kapitel \ref{BestimmungZones} liefern Hypothesen f"ur Textbl"ocke.
Damit der Text aus dem Dokument originalgetreu rekonstruiert werden kann, mu"s die der Segmentierung
folgende OCR--Analyse die Textbl"ocke in der richtigen Reihenfolge auswerten.

Diese Abfolge der Textbl"ocke mu"s somit der {\em logischen Lesefolge\/} innerhalb des
Dokuments entsprechen.
Somit besitzt jeder hypothetisierte Textblock seine -- wiederum hypothetische~-- Position
innerhalb der Menge der Textbl"ocke.

Die Sortierung erfolgt nach folgendem Schema \cite{Saitoh93}:
\begin{enumerate}
  \item Geometrische Beziehungen zwischen Textbl"ocken werden als hierarchische Beziehungen
  interpretiert und in eine Baumstruktur "ubertragen.
  \item Die Bl"ocke k"onnen nun einfach innerhalb des Baumes sortiert werden.
  \item Die Pre--Order Traversion\footnote{Zuerst wird die Wurzel des Baumes besucht, 
  anschlie"send die Teilb"aume} des Baumes ergibt die hypothetische Lesefolge.
\end{enumerate}

\section{Transformation der geometrischen Struktur}

Untersuchungen haben ergeben, da"s eine Vielzahl von Dokumenten hierarchisch aufgebaut ist:
Ein inhaltlich zusammenh"angender Artikel besteht meist aus einem \mbox{("Uberschrift--)} Textblock, der
darunterliegende, weniger breite Textbl"ocke "uberspannt, 
die wiederum "uber anderen stehen k"onnen usw.

Bei der Sortierung ist zu beachten, da"s sich die normale Lesefolge (von links nach rechts und
von oben nach unten) innerhalb hierarchisch angeordneter Textbl"ocke "andert 
(s.\ \BildRef{Einfluss}). Jeder Textblock "ubt somit einen {\em Einflu"s\/} auf die darunterliegenden
Textbl"ocke aus. Das Gebiet, in dem ein Textblock andere beeinflu"st, wird im folgenden
{\em Einflu"sbereich\/} genannt.

\Bild{Einfluss}{10}{Ber"ucksichtigung von hierarchischen Stukturen bei der Lesefolge}

Unter Beachtung der geometrischen Anordnung werden die Textbl"ocke in eine hierarchische
Baumstruktur "ubertragen. Ausgehend von einem alles "uberschreibenden (beeinflussenden)
virtuellen Textblock, werden Vater--Sohn-- und Bruder--Beziehungen definiert:

\begin{itemize}
  \item Befindet sich ein Textblock unter einem anderen, d.h. innerhalb dessen Einflu"sbereich,
  wird zwischen ihnen eine Vater--Sohn--Beziehung festgelegt. Der Einflu"sbereich wird von dem
  Vater--Textblock zu dem Sohn--Textblock vererbt (s.\ TB1 und TB2 in \BildRef{Einflussbereich}).
  
  Liegen direkt unter einem Textblock mehrere andere, so wird der Einflu"sbereich unter ihnen
  aufgeteilt (s.\ TB3 und TB4 in \BildRef{Einflussbereich}).
 
  \Bild{Einflussbereich}{4.5}{Vererbung und Aufteilung der Einflu"sbereiche}

  \item Liegen Textbl"ocke nebeneinander, d.h. es gilt
  $$Y_{1max} > Y_{2min} \quad \wedge \quad Y_{1min} < Y_{2max}
  \quad\wedge\quad 
  \left( X_{1max} < X_{2min} \quad \vee \quad X_{2max} < X_{1min} \right),$$
  
  wobei $X_{min}$ und $X_{max}$ der linken
  bzw.\ rechten, $Y_{min}$ und $Y_{max}$ der oberen bzw.\ unteren Kante eines Textblocks
  entspricht, definiert man in diesem Fall eine Bruder--Beziehung.
\end{itemize}

Diese Beziehungen lassen sich in eine Baumstruktur "ubertragen, wobei jedem Knoten ein Textblock
zugeordnet ist (s.\ \BildRef{BeziehungenInBaum}). 
Damit sich bei einer Pre--Order Traversion des Baumes die Lesefolge ergibt,
m"ussen die Textbl"ocke, zwischen denen eine Bruder--Beziehung besteht, 
d.h.\ die denselben Vaters besitzen, im Baum von links nach rechts, entsprechend ihrer
geometrischen Position, angeordnet werden.

\Bild{BeziehungenInBaum}{8}{"Ubertragen der geometrischen Beziehungen (links) in eine hierarchische Baumstruktur (rechts)}

Die bisherigen Definitionen und Regeln reichen bei komplexeren Dokumenten nicht aus und
m"ussen deshalb um folgenden Sonderfall erweitert werden:
Textbl"ocke, die sich nicht oder nur teilweise
in einem Einflu"sgebiet eines anderen befinden (TB 7 aus \BildRef{LesefolgeStrukturiert}), 
wurden bisher noch nicht ber"ucksichtigt.
Sie stellen eigenst"andige (unabh"angige) Bl"ocke dar und werden deshalb auf dieselbe
hierarchische Ebene wie die des virtuellen Textblocks gestellt, d.h.\ zwischen ihnen und dem
virtuellen Textblock wird eine Bruder Beziehung definiert. Da diese Textbl"ocke nicht
gezwungenerma"sen wie sonstige Br"uder--Textbl"ocke, nebeneinander liegen, m"ussen die Knoten nicht 
von rechts nach links sortiert werden. F"ur diese S"ohne der Wurzel m"ussen
weitere Sortierregeln aufgestellt werden.

\Bild{LesefolgeStrukturiert}{10}{Spezialfall: Textbl"ocke die nicht oder nur teilweise in einem Einflu"sgebiet liegen (TB 7) stellen unabh"angige Bl"ocke dar}

Die S"ohne der Wurzel werden nach folgendem
Schema sortiert. Zwei Knoten werden getauscht, wenn mindestens eine der Bedingungen
erf"ullt ist:
\begin{enumerate}
  \item Textblock des linken Knotens liegt im Einflu"sbereich des rechten Knotens (s.\ \BildRef{ErweiterteSorierung} links)
  \item $Y_{links} \; > \; Y_{rechts}$ und $X_{links} \; > \; X_{rechts}$ (s.\ \BildRef{ErweiterteSorierung} Mitte)
  \item Textblock des rechten Knotens liegt nicht unter dem Einflu"sbereich des linken Knotens
  und $X_{links} \; > \; X_{rechts}$ (s.\ \BildRef{ErweiterteSorierung} rechts),
\end{enumerate}
wobei der Punkt $(X,Y)$ die linke obere Ecke des zum Knoten geh"orenden Textblocks darstellt.

\Bild{ErweiterteSorierung}{13}{Erweiterte Sortierregeln f"ur die S"ohne der Wurzel}

\subsection*{Layoutbegrenzer}

Liniensegmente (s.\ \KapRef{Filterung}) und wei"ser horizontaler Zwischenraum 
(bestimmt durch einen horizontalen XY--Cut) dienen in einem Dokument als
Layoutbegrenzer und werden deshalb bei der Sortierung ber"ucksichtigt
(\BildRef{LayoutBegrenzer}). Liegt zwischen zwei Textbl"ocken ein solcher
Layoutbegrenzer, wird zwischen dem Textblockpaar keine Beziehung definiert.

\Bild{LayoutBegrenzer}{10}{Die urspr"ungliche Sortierreihenfolge (links) wird aufgebrochen, wenn Layoutbegrenzer zwischen Textbl"ocken liegen (rechts)}

\subsection*{Textrahmen}

Ein oft benutztes Gestaltungsmittel, um inhaltlich zusammenh"angende Textbereiche
zu kennzeichnen, ist die Verwendung von Textrahmen. Da Textbl"ocke innerhalb
dieser Rahmen dieselbe Komplexit"at aufweisen k"onnen, wie diejenigen innerhalb 
des normalen Dokuments, werden sie separat, d.h.\ wie ein eigenst"andiges Dokument, betrachtet.
Hierzu ist die
Baumstruktur geeignet zu erweitern: Jeder Textrahmen bekommt seine eigene Wurzel
zugewiesen, so da"s zuerst alle Textbl"ocke innerhalb der Textrahmen betrachtet werden,
und anschlie"send die verbleibenden.

\section{Bestimmung der Lesefolge aus den hierarchischen Beziehungen}

Die logische Lesefolge l"a"st sich durch eine Pre--Order Traversion der Baumes herauslesen.
Die Wurzeln und die virtuellen Textbl"ocke werden dabei nicht weiter beachtet
(s.\ \BildRef{Lesefolge}).

\Bild{Lesefolge}{12}{Ergebnis der logischen Textblocksortierung}

\section{Ergebnisse der Sortierung -- Ausblick}
Das dargestellte Verfahren beschreibt, wie aus der geometrischen Anordung von Textbl"ocken
deren hypothetische Lesefolge gewonnen werden kann.
Es zeigt sehr gute Ergebnisse bei klar stukturierten Dokumenten
(s.\ \BildRef{LesefolgeBeispiele} und \ref{eps:LesefolgeBeispiele2}
Seite \pageref{eps:LesefolgeBeispiele2} und \pageref{eps:LesefolgeBeispiele}).
Da als Eingangsdaten jedoch ausschlie"slich die umschreibenden Rechtecke der Textbl"ocke benutzt werden, sind Fehler durch Mehrdeutigkeiten unvermeidbar.
Die Verfahren zur Sortierung stellten nicht den Schwerpunkt dieser Arbeit dar. Aus diesem Grund,
wurde nur die grunds"atzliche Eignung der Verfahren "uberpr"uft,
jedoch die Erkennungsrate nicht qualitativ gemessen. 
Dazu w"are analog zu der Berechnung der Zoning--Erkennungsraten 
(s.\ \KapRef{ErgebnisZoning}) wiederum eine aufwendige Erstellung von Referenzdaten samt
Auswertung n"otig gewesen, die den zeilichen Rahmen dieser Arbeit gesprengt h"atte.

Nur aufgrund der geometrischen Verh"altnisse auf die Lesefolge zu schlie"sen, stellt sich,
selbst f"ur einen menschlichen Betrachter, als schwieriges Problem dar. Der Leser kann sich
dies verdeutlichen, wenn der die linke Bildh"alfte von Abbildung \ref{eps:LesefolgeProblem} 
betrachtet und dabei die rechte verdeckt. 
Der Mensch kann aufgrund des inhaltlichen Zusammenhangs auf den
logisch n"achsten Block schlie"sen -- er "uberfliegt die ersten Zeilen der in Frage kommenden
und w"ahlt intuitiv den richtigen aus. Je ausgefallener und komplexer das Layout, desto
schwieriger die Aufgabe. 
So wird deutlich, da"s die Dokumentmenge auf klar strukturierte Dokumente
(Gesch"aftsbriefe, wissenschaftlichen Artikel, Zeitschriften) beschr"ankt werden mu"ste.
Bei Steuerbescheiden tritt die Frage auf, welche Abfolge innerhalb einer Tabelle "uberhaupt 
die richtige ist.

\Bild{LesefolgeProblem}{12}{Die umschreibenden Blockrechtecke als Ausgangsdaten (links) f"ur das Sortierverfahren -- selbst f"ur den menschlichen Betrachter ist die Bestimmung der Lesefolge ohne Zuhilfenahme des Originals nicht einfach}

Folgende Probleme sind aufgetreten:
\begin{itemize}
  \item Aufgrund der umschreibenden Rechtecke kann nicht immer eindeutig auf eine Lesefolge
  geschlossen werden (s.\ \BildRef{LSZweideutigkeiten}). Die Sortierung erfolgt nach der
  ersten M"oglichkeit, obwohl bei inhaltlicher Betrachtung die zweite
  vielleicht die richtige w"are (\BildRef{LSFehler} links).
  
  \Bild{LSZweideutigkeiten}{8}{Lesefolge l"a"st sich aufgrund der Eingangsdaten (links) nicht eindeutig bestimmen}
  
  Um diese Fehler zu vermeiden, m"u"sten Regeln eingef"uhrt werden, die zus"atzlich
  zu der geometrischen Anordnung der Textbl"ocke, deren Inhalt ber"ucksichtigen 
  (z.B.\ Buchstabengr"o"se). Eine fehlerfreies Ergebnis wird man nur erhalten, wenn der Text
  gelesen und verstanden wurde.

  \item Textbl"ocke unter oder "uber Tabellen und Grafiken 
  werden direkt in den Textflu"s einbezogen (\BildRef{LSFehler} rechts). Um dies zu verhindern,
  m"u"sten Textbl"ocke unter bzw.\ "uber Nicht--Textgebieten als Bildunter-- 
  bzw.\ Bild"uberschriften
  klassifiziert und bei der Sortierung separat betrachtet werden. Hierzu m"u"sten zuvor
  Text-- und Nicht--Textgebiete erkannt und getrennt werden.
\end{itemize}

\Bild{LSFehler}{9.5}{Fehler bei der Bestimmung der Lesefolge: Mehrdeutigkeiten (links) und Einbeziehung von Bildunterschriften in den Textflu"s (rechts)}

Eine entscheidende Voraussetzung f"ur die Richtigkeit der Sortierung ist ein gutes Ergebnis aus der
Textblocksegmentierung. Traten hier schon Fehler auf, werden sich diese
bei der Sortierung weiter fortsetzen. Somit gelten die bei der Segentierung genannten
Verbesserungsans"atze auch bei der Sortierung.

Grunds"atzlich w"are jede neue auf Sonderf"alle angepa"ste, Sortierungsregel nur eine unter
vielen anderen denkbaren. 

\Bild{LesefolgeBeispiele2}{11}{Beispiele f"ur die Sortierung (1)}
\Bild{LesefolgeBeispiele}{13}{Beispiele f"ur die Sortierung (2)}

