\section{Stand der Technik in der Layoutanalyse}\label{LayoutanalyseMethoden}

Die in der Literatur beschriebenen Verfahren zur Segmentierung von Layoutbl"ocken
lassen sich grob in zwei Vorgehensweisen unterteilen (s.\ \BildRef{VergleichTopDownBottomUp}):

\Bild{VergleichTopDownBottomUp}{8}{Top--Down Verfahren (links) und Bottom--Up Verfahren.}

\subsection{Top--Down Vorgehen}\label{XYCutTheroie}

Die Analyse nach der Top--Down Strategie beginnt typischerweise mit einer sehr groben
Einteilung des Dokuments.
Diese Hypothesen werden in den nachfolgenden Schritten weiter verfeinert und die Analyse endet
bei der niedersten Stufe, den Pixeln oder Zusammenhangsgebieten.

George Nagy schlug in \cite{Nagy84} eine Methode vor, mit der ein
Dokument in eine hierarchische Datenstruktur (von den Autoren {\em XY--Tree\/} genannt)
"uberf"uhrt werden kann:
Durch alternierende Schnitte parallel zu der X-- bzw.\ Y--Achse ({\em XY--Cut\/}) wird die Seite in zwei oder mehrere Unterbl"ocke aufgeteilt (s.\ \BildRef{XYCutIdee}).
Dieses Verfahren wird so lange rekursiv angewandt, bis eine Seite nicht weiter segmentiert werden
kann. Die Position der Schnitte wird durch ein horizontales bzw.\ vertikales Projektionsprofil
der schwarzen Pixel bestimmt.

\Bild{XYCutIdee}{11}{Idee der rekursiven Schnitte (XY--Cuts)}

%\pagebreak
{\bf Vorteile:}
\begin{itemize}
  \item Anschauliche Vorgehensweise
  \item Transformation des Dokuments in eine leicht handhabbare Datenstruktur
\end{itemize}

{\bf Nachteile:}
\begin{itemize}
  \item Nur f"ur {\em Manhattan--Layouts} geeignet. Dies sind Layouts, die ausschlie"slich
        durch eine Menge von horizontalen und vertikalen Geraden, die durch wei"se Gebiete laufen,
        getrennt werden k"onnen, vergleichbar mit der Stra"senanordnung von Manhattan
        (Nicht--Manhattan--Layout s.\ \BildRef{DemoXYCutError} und
        \ref{eps:ZoningErg2} auf Seite \pageref{eps:ZoningErg2}).

  \item Erzeugung des Projektionsprofils aus dem Bin"arbild ist wegen der gro"sen Datenmengen sehr zeitintensiv.

  \item Die Bestimmung der Schnittpositionen, d.h.\ der Minima, innerhalb des Projektionsprofils
        ist, speziell bei einer sehr feinen Segmentierung, problematisch.

\end{itemize}

\Bild{DemoXYCutError}{7}{Unl"osbares Problem f"ur die XY--Cut Analyse}

\subsection{Bottom--Up Vorgehen}\label{BottomUp}

Diese Methode schreitet den umgekehrten Weg: Ausgehend von der kleinsten Detailierungsstufe
(Pixel oder Zusammenhangsgebiete) werden immer gr"o"sere Bl"ocke gebildet --
Buchstaben zu W"ortern, W"orter zu Zeilen, Zeilen zu Textbl"ocken -- solange,
bis die gesamte Seite bearbeitet wurde.

\begin{enumerate}

  \item {\bf Smearing:}\\
        Ziel des Smearings ist, kleine Schwarzgebiete (Buchstaben, W"orter und Zeilen) zu
        gr"o"seren Bl"ocken zu verschmelzen (s.\ \cite{Wahl} und \BildRef{DemoSmearing}). Wie der
        Name erkennen l"a"st, werden die Schwarzfl"achen des Bin"arbildes in horizontaler und/oder vertikaler
        Richtung {\it verschmiert}. Ein sehr wichtiger Faktor f"ur zufriedenstellende Ergebnisse
        ist hierbei die Wahl der Parameter (Smearing--Window), um nicht f"alschlicherweise mehrere
        nebeneinanderliegende Spalten zu verbinden. Da bei einer statischen Festlegung die
        unterschiedlichen Layoutstrukturen nicht ber"ucksichtigt werden k"onnen, ist eine direkte
        Parameterberechnung aus dem Dokument unbedingt notwendig.\label{Smearing}

        {\bf Vorteile:}
        \begin{itemize}
          \item einfache Implementierung mittels der morphologischen Operationen Dilatation und Erosion (Closing)
          \item gute Ergebnisse bei einer Vielzahl von Layouttypen
        \end{itemize}

        {\bf Nachteile:}
        \begin{itemize}
          \item Lokale Parametrierung notwendig.
          \item Das Smearing verbindet jede Art von Objekten, so da"s Konsistenzpr"ufungen notwendig
                werden.
          \item Da die Smearingalgorithmen auf der Pixelebene mit einem gro"sen Datenvolumen arbeiten,
                sind die Laufzeiten entsprechend hoch.
        \end{itemize}

        \Bild{DemoSmearing}{8}{Smearing an einem Beispiel.}

  \item {\bf Unterabtastung (Subsampling):}\\
        Hierbei wird das Bin"arbild in nicht"uberlappende Rechtecke
        konstanter Gr"o"se eingeteilt. Jeder dieser Bl"ocke wird nun auf einen Pixel abgebildet
        (reduziert). Beispiele siehe \BildRef{DemoSubsampling} und \BildRef{Subsampling} auf Seite
        \pageref{eps:Subsampling}.\label{Subsampling}

        Diese Operation wird Rangordnungsfilter genannt. Zun"achst wird f"ur jedes Rechteck seine
        Rangordnung bestimmt, d.h.\ die Anzahl der schwarzen Pixel gez"ahlt. Ist diese
        Anzahl gleich oder gr"o"ser einem bestimmten Schwellwert, so wird das korrespondierende Pixel
        gesetzt, ansonsten gel"oscht.

        Das Resultat der Datenreduzierung ist eine
        h"ohere Verarbeitungsgeschwindigkeit (siehe \KapRef{Speed}) nachfolgender
        Algorithmen wie z.B.\ morphologischer Operationen
        (eine mit 300dpi digitalisierte DIN--A4--Seite hat
        ein Datenvolumen von etwa 1MB -- bei einer Reduzierungsrate von 64:1, d.h.\ ein 8x8 Pixel gro"ser
        Block wird auf ein Pixel abgebildet, verbleiben noch 17KB).
        Zus"atzlich entstehen gr"o"sere
        zusammenh"angende Schwarzgebiete "ahnlich denen beim Smearing.

        \Bild{DemoSubsampling}{8}{Unterabgetastetes Bin"arbild (wieder in Originalgr"o"se)}

  \item {\bf Zusammenhangsobjekte verbinden:}\\
        Der erste Schritt bei dieser Methode ist die Transformation des Bin"arbildes in
        Zusammenhangsobjekte. Um eine Segmentierung in hierarchische,
        ineinander aufbauende Bl"ocke zu erhalten, m"u"sen die BCC--Objekte in einem zweiten Schritt
        zu gr"o"seren Gruppen verbunden werden. Die dazu ben"otigten Regeln lassen sich aus Abstandsma"sen,
        Gr"o"senverh"altnissen und deren geometrischen Strukturen herleiten (s.\ \KapRef{Zusammenfassen}).

        \pagebreak[3]
        {\bf Probleme:}
        \begin{itemize}
          \item Ausgehend von nur einzelnen BCC--Objekten ist es schwierig, verl"a"sliche Aussagen
                "uber den Grad ihrer "Ahnlichkeit zu machen, da die Statistik sehr unsicher ist, wenn
                nur wenige Elemente zur Verf"ugung stehen.
          \item Da Dokumente mehrere tausend Zusammenhangsobjekte enthalten k"onnen, ist
                der Vorgang des Verbindens zeitaufwendig.
        \end{itemize}

\end{enumerate}

\subsection{Hybride Vorgehensweisen}

Ferner existieren Methoden, die sich nicht eindeutig in obige zwei Sparten einordnen lassen.
\begin{enumerate}
  \item
        Ein Ansatz hybrider Vorgehensweisen besteht darin, nicht die schwarzen Gebiete des Vordergrundes,
        sondern den wei"sen Hintergrund zu untersuchen. In aufwendigen Layout\-strukturen, bei denen Textspalten sehr nahe beieinander stehen, dienen d"unne schwarze Linien und wei"se Fl"achen als Layoutbegrenzungen.

        \begin{itemize}
          \item Henry Baird schl"agt in \cite{Baird92} vor, die gefundenen wei"sen Rechtecke zu sortieren und
                ihrer Reihenfolge nach bis zu einer Schwelle wieder zu vereinigen (vom Autor {\em
                    `global--to--local'} Strategie genannt), so da"s schrittweise die Untergliederung verfeinert
                wird. Probleme bereitet der rechtzeitige Abbruch der
                Vereinigungsphase sowie das Design der Sortierreihenfolge
                (Regel: `gro"se' sowie `lang und schmale' Rechtecke werden zuerst verbunden).

          \item Masayuki Okamoto \cite{Okamoto93} verbindet Zusammenhangsobjekte, die zwischen Linien
                und wei"sem Zwischenraum liegen, zu Bl"ocken. Hier treten "ahnliche Probleme, wie bei den
                zuvor besprochen Bottom--Up Methoden, auf.
        \end{itemize}

  \item
        J.~Wieser und A.~Pinz schlagen in \cite{Wieser93} eine mehrmalige Abfolge von Smearing und XY--Cuts vor:
        Durch Smearing mit einem relativ gro"sen Smearing--Window sollen viele Objekte auf dem Rasterbild
        zu gro"sen Schwarzbereichen verbunden werden. Diese Schwarzbereiche werden zun"achst durch
        rekursieve XY--Cuts in kleinere St"ucke zerteilt, auf die im Originalbild wiederum Smearing mit
        anschlie"senden XY--Cuts angewandt wird. Die dabei ben"otigten Parameters"atze werden aus diesen
        kleinen Bl"ocken gewonnen.
        Eine Zusammenfassung zerbrochener Bl"ocke bildet den Schlu"s der Segmentierung.

          {\bf Vorteil:}
        \begin{itemize}
          \item Lokale Berechnung der Smearing Parameter direkt aus dem Dokument (nur beim zweiten
                Smearing). Dadurch sind die Verfahren relativ robust gegen"uber verschiedenen Schriftgr"o"sen
                innerhalb eines Dokuments.
        \end{itemize}

        \pagebreak[3]
        {\bf Nachteile:}
        \begin{itemize}
          \item Zweimaliger zeitaufwendiger Smearing-- und XY-Cut--Proze"s.
          \item Es erfolgt keine Konsistenzpr"ufung.
          \item Mittels XY--Cuts k"onnen nur Bl"ocke getrennt werden, bei denen ein horizontaler
                oder vertikaler Durchschu"s (also eine Leerzeile oder --spalte) existiert. Bei "uberlappenden
                Objekten scheitert das Verfahren.
        \end{itemize}
\end{enumerate}