\chapter{Grundlegende Techniken}
\label{GrundlegendeTechniken}

Die in diesem Abschnitt beschriebenen Techniken dienen als Grundlage f"ur die Layoutanalysemethoden 
in Kapitel \ref{LayoutanalyseMethoden}. Kapitel \ref{Morphologie} und \ref{Zusammenhangsanalyse}
stellen Methoden der Bildverarbeitung vor:

\begin{description}
  \item[Ikonische Bildverarbeitung:] Die ikonische Bildverarbeitung besch"aftigt sich mit 
  Operationen, die ausschlie"slich mit einzelnen Bildpunkten arbeiten (\KapRef{Morphologie}).

  \item[Symbolische Bildverarbeitung:] Nach der Transformation des Bin"arbildes in den
    symbolischen Bereich stehen den symbolischen Bildverarbeitungsmethoden Objekte
    mit Informationen "uber ihre Kontur und Farbe zur Verf"ugung
    (\KapRef{Zusammenhangsanalyse} und \ref{Homogenitaet}).
    
\end{description}
Die Histogrammanalyse wird in Kapitel \ref{Histogrammanalyse} als Methode zur Bestimmung einer
Verteilungsfunktion aus einer Stichprobe vorgestellt.  
Diese grundlegenden Verfahren sind nicht Gegenstand dieser Arbeit, Implementationen hiervon existieren bereits in der verwendeten Softwareumgebung.

%%%%%%%%%%%%%%%%%%%%%%%%%%%%%%%%%%%%%%%%%%%%%%%%%%%%%%%%%%%%%%%%
\section{Morphologische Operationen}
\label{Morphologie}

In diesem Kapitel werden Methoden der ikonischen Bildverarbeitung beschrieben. Diese
Verfahren ben"otigen als Eingangsdaten ein Bin"arbild und liefern als Ergebnis
wiederum ein Bin"arbild. 

Lineare Filtermethoden (wie z.B.\ Tiefpa"sfilter) arbeiten in einem Wertebereich
au"serhalb des bei der Klassifikation von Schriftzeichen verwendeten Bereichs von
nur zwei Grauwerten. Diese Methoden liefern bei der Anwendung auf bin"are Muster
im allgemeinen keine Ergebnisse, deren Werte ebenfalls wieder bin"ar sind. Dies mu"s
in einer nachfolgenden Schwellwertoperation sichergestellt werden. Eine andere
M"oglichkeit besteht darin, f"ur bin"are Muster spezialisierte nichtlineare Operationen
zu verwenden \cite{Niemann83}. Beispielsweise haben lineare Operationen bei der Gl"attung
den Nachteil, da"s sie nicht nur hochfrequente St"orungen beseitigen, sondern auch 
Bildkonturen verschleifen. Um diesen Nachteil zu vermeiden wurden nichtlineare Operationen
entwickelt, die auf der Ordnung der Funktionswerte einer kleinen Nachbarschaft beruhen.
Dieses ist auch das Prinzip der allgemeinen Rangordnungsoperationen. 
Man betrachtet einen Funktionswert
$f_{jk} \in \re$ aus einem zweidimensionalen Raum in der Folge $[f_{jk}]$ und 
bezeichnet mit $N_M$ eine Nachbarschaft von $f_{jk}$, die $M$ Werte enth"alt,
\begin{eqnarray*}
N_M &=& \left\{ f_{j+\mu, k+\nu} \;|\; \mu = 0, \pm 1,\ldots, \pm m;\; \nu = 0, \pm 1, \ldots, \pm n \right\},\\
M   &=& (2m + 1)(2n + 1).
\end{eqnarray*}
Die Elemente von $N_M$ werden der Gr"o"se nach geordnet, wobei der kleinste Wert aus $N_M$
mit $r_1$ bezeichnet wird, das n"achstgr"o"sere mit $r_2$ und so weiter. Dieses ergibt die
Rangordnung der um $f_{jk}$ liegenden Funktionswerte
$$R_{jk} = \left\{ r_1, r_2, \ldots, r_M \;|\; r_\nu \in N_M,\;r_\nu \le r_{\nu+1},\;\nu = 1,\ldots,M \right\}.$$
Eine Rangordnungsoperation ist definiert durch 
$$h_{jk} = \varphi\:(R_{jk}).$$
Drei spezielle Operationen sind
\[ \begin{array}{llcl}
\mbox{Erosion}          & h_{jk} &=& r_1,\\
\mbox{Median}           & h_{jk} &=& r_{(M + 1) / 2},\\
\mbox{Dilatation\qquad} & h_{jk} &=& r_M.
\end{array} \]
Angewandt auf Bin"arbilder mit $f_{jk} \in \{ 0, 1 \}$ lassen sich diese Operationen wie folgt anschaulich beschreiben:
\begin{description}
  \item[Erosion:] Ein Bildpunkt wird dann gesetzt, wenn {\em alle} Bildpunkte innerhalb 
  der Maske $N_M$
  gesetzt sind. Somit schrumpfen Schwarzbereiche zusammen (\BildRef{Erosion}).
  \Bild{Erosion}{7}{Ausgangsbild links, Ergebnis der Erosion rechts}
  \item[Median:]  Ein Bildpunkt wird dann gesetzt, wenn mindestens die {\em H"alfte} der Bildpunkte 
  innerhalb der Maske gesetzt ist. Dadurch l"a"st sich Rauschen (Pepper Noise) von einer
  Fl"ache mit dem Fl"acheninhalt $g$ l"oschen, wenn $g \le \frac{M}{2}$ gilt.
  \item[Dilatation:] Ein Bildpunkt wird dann gesetzt, wenn {\em mindestens ein} Bildpunkt 
  innerhab der Maske gesetzt ist. Somit breiten sich Schwarzbereiche aus.
\end{description}

Um St"orungen, die eine gewisse Gr"o"se nicht "uberschreiten, zu beseitigen, k"onnen Dilatation und
Erosion kombiniert eingesetzt werden:

\begin{description}
  \item[Opening:] Die Aufeinanderfolge von Erosion und Dilatation wird als {\em Opening\/}--
  Operation bezeichnet. Rauschen, bedingt durch eine schlechte Vorlagenqualit"at,
  kann mit diesem Filter abgeschw"acht werden (\BildRef{Opening}).
  
  \Bild{Opening}{7}{Reduzierung von Rauschen mit dem Opening--Filter.}
  
  \item[Closing:] Die umgekehrte Abfolge, also Dilatation mit anschlie"sender Erosion ist
  als \mbox{\em Closing\/} definiert. Sie dient dazu Zwischenr"aume zu schlie"sen, wie zum Beispiel
  das Verbinden zerbrochener Buchstaben (\BildRef{Closing}).
  
  \Bild{Closing}{3.5}{Zerbrochene Buchstaben werden durch die Closing--Operation verbunden} 
\end{description}

%\clearpage
%%%%%%%%%%%%%%%%%%%%%%%%%%%%%%%%%%%%%%%%%%%%%%%%%%%%%%%%%%%%%%%%
\section{Zusammenhangsanalyse}
\label{Zusammenhangsanalyse}

Ausgangsbasis f"ur die Segmentierung ist ein Bin"arbild. Es besteht aus einer Matrix aus
kleinen quadratischen Bildelementen, denen jeweils ein bestimmter Farbwert, entweder schwarz
oder wei"s, zugeordnet ist. Jedes Bildelement, im weiteren Pixel genannt, l"a"st sich "uber
eine Koordinatenangabe in einem orthogonalen Koordinatensystem ansprechen 
(s.~\BildRef{Binaerbild}). Merkmale "uber den Inhalt des Bildes (wie z.B.\ die Buchstabenh"ohe)
lassen sich hierbei nur schwer berechnen.

\Bild{Binaerbild}{3}{Repr"asentation eines Bin"arbildes als Pixelmatrix}

Die Zusammenhangsanalyse, BCC--Analyse ({\em B\/}inary {\em C\/}onnected {\em C\/}omponent)
genannt, "uberf"uhrt ein Rasterbild nicht wiederum in ein anderes Rasterbild, sondern in
Bildobjekte. Dabei werden aus der Bildmatrix topologisch
zusammenh"angende Mengen schwarzer Pixel, die sogenannten Zusammenhangskomponenten, extrahiert.
Dieser Verfahrensschritt ist datenerhaltend und deshalb reversibel. Es wird
eine rechnerinterne Darstellung mit Hilfe einer hierarchischen Datenstruktur verwendet. In
ihr werden die R"ander (Kontur) eines Objekts sowie die {\em Enthaltensein\/}--Relation von
inneren Konturen zu ihren "au"seren Konturen vermerkt. Diese Darstellung erm"oglicht eine
effiziente Berechnung von Merkmalen (s.~\KapRef{Homogenitaet}), die f"ur
die Layoutanalyse ben"otigt werden. 

Die Kanten der Kontur bestehen
ausschlie"slich aus waagerechten und senkrechten Ge\-ra\-den\-st"u"cken, 
wobei einer senkrechten Kante
immer eine waagerechte folgt. Des weiteren kreuzen sich keine Kanten und die Kontur
ist stets geschlossen.
Ineinanderliegende Konturen werden als "ubereinanderliegende, sich "uberdeckende Zusammenhangsobjekte
alternierender Farben interpretiert. Diese hierarchische Zuordnung wird {\em Vater--Sohn} Relation
genannt; weiterhin wird zwischen Objekten, die unmittelbar in einem gemeinsamen
Zusammenhangsgebiet liegen, eine {\em Bruder} Beziehung definiert (s.~\BildRef{BccHierarchie}).

\Bild{BccHierarchie}{11.5}{Veranschaulichung der Hierarchiebeziehungen}

Neben der Kontur, der Farbe und den Hierarchiebeziehungen wird 
au"serdem das umschreibende Rechteck jedes Zusammenhangsgebietes berechnet 
({\em BCC--Bounding Box\/}).
Diese Rechtecke sind in der Layoutanalyse von besonderer Bedeutung 
(s.~\BildRef{BccBoxes}). Weitere Informationen und Implementierungsdetails sind \cite{MaOb90} 
zu entnehmen.

\Bild{BccBoxes}{13}{Bin"arbild, BCC--Konturen, umschreibende Rechtecke (v.l.n.r.)}

%%%%%%%%%%%%%%%%%%%%%%%%%%%%%%%%%%%%%%%%%%%%%%%%%%%%%%%%%%%%%%%%
\section{Histogrammanalyse}\label{Histogrammanalyse}

Es sei eine Stichprobe $(x_1,\ldots,x_n)$ mit dem Merkmal $X$ gegeben. Die Verteilung von $X$
sei nicht bekannt. Um bei einem qualitativen Merkmal eine Vorstellung von dieser Verteilung
zu erhalten, konstruiert man das sogenannte {\em Histogramm}.

Man f"uhrt eine Zerlegung der reellen Achse in endlich viele aneinander grenzende Intervalle 
$\Delta_1,\ldots,\Delta_k$, sogenannte {\em Klassen}, durch. Es wird nachgez"ahlt, wieviele
Werte $x \in X$ in $\Delta_j \; (1 \le j \le k)$ liegen. Die Anzahl $m_j$ der Objekte wird
{\em Klassenh"aufigkeit\/} genannt. 
"Uber $\Delta_j$ wird ein Rechteck der H"ohe $\frac{m_j}{n}$ gezeichnet
(relative Klassenh"aufigkeiten). Das so entstehende Stufenbild wird als Histogramm 
zum Merkmal $X$ bezeichnet.

Eine h"aufige Anwendung ist das Finden von Klassenh"aufigkeitsmaxima ({\em Peaks\/}) innerhalb der Verteilung
(z.B.\ bei der Zeilensuche \KapRef{Zeilenabstand}).
Hier empfiehlt es sich einen sogenannten {\em Gravitationsfilters\/} anzuwenden.
Bei diesem iterativen Filter wandern die
Histogrammeintr"age in Richtung der gr"o"sten `Massen', d.h.\ in Richtung der gr"o"sten Anh"aufung von
Histogrammeintr"agen. Der Einflu"s der `Schwerkraft' (Filterradius) ist variabel einstellbar.
Durch die wiederholte Anwendung werden die Maxima herausgearbeitet, und man erh"alt schlie"slich ein 
Histogramm, das idealerweise nur noch aus einzelnen Peaks besteht (s.\ \BildRef{DemoHistogramm}).

\Bild{DemoHistogramm}{7}{Histogramm ungefiltert (links) und nach der Gravitationsfilterung}

