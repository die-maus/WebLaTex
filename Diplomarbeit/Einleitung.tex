\chapter{Einleitung und Motivation}
\section{Dokumentanalyse}
\label{DokAna}

Das Papier als Informationstr"ager ist in unserer heutigen Zeit nicht
mehr wegzudenken. Die meisten B"ucher, Zeitschriften und Artikel, vor
allem "alteren Jahrgangs, sind nur in ihrer gedruckten Form
erh"altlich. Dennoch m"ochte man die Papierberge, die sich auf
Schreibtischen und in Aktenschr"anken t"urmen, mit dem
B"urocomputer elektronisch verarbeiten. 
Schon vor Jahren wurde von den Propheten der Computergesellschaft das
Zeitalter der {\em papierlosen B"uros\/} angek"undigt. 
Der Wunsch ganz auf das Papier
zu verzichten liegt deshalb nahe, er wird sich jedoch auf lange Zeit nicht verwirklichen lassen.
Um diesem Ziel n"aher zu kommen, mu"s die
auf das Papier gebrachte Information zun"achst in eine dem Computer
verst"andliche Form konvertiert und interpretiert werden. Anschlie"send m"ussen
entsprechende Handlungen ausgel"ost werden. Dies sind Aufgaben der
{\em Dokumentanalyse\/} (s.\ \BildRef{Dokumentanalyse} und
\cite{PixelToContents}):

\Bild{Dokumentanalyse}{15}{Ablauf der Dokumentanalyse}

\begin{enumerate}

\item {\em Digitalisierung:}\\
Das grundlegende Funktionsprinzip eines Scanners besteht in dem zeilenweisen 
Abtasten (engl.\ {\em to scan\/}) der
Vorlagen durch eine Leiste lichtempfindlicher Sensoren. 
Beim Digitalisieren wird die Vorlage mit Hilfe einer starken
Lichtquelle angestrahlt, wodurch das Bild auf eine Reihe von CCD--Elementen
({\sl C\/}harge {\sl C\/}oupled {\sl D\/}evices) geworfen wird. Diese
Sensoren nehmen das unterschiedlich reflektierte Licht auf und wandeln
die Helligkeitsinformation in digitale Werte. F"ur die Texterkennung
gen"ugen i.d.R.\ Ein--Bit--Scanner, so da"s nach der Digitalisierung ein
Bin"arbild  zur Verf"ugung steht, d.h.~jeder Bildpunkt wird mit einem Bit kodiert.

Zu dem wichtigsten Qualit"atsmerkmal eines Scanners geh"ort die Aufl"osungsf"ahigkeit. 
Es hat sich 300dpi ({\em d\/}ots {\em p\/}er {\em i\/}nch) 
in der Texterkennung als Standard etabliert. 
Jeweils 300 CCD--Elemente liegen pro Inch nebeneinander.
Die CCD--Zeile bewegt sich au"serdem mit 300 Schritten pro Inch unter der Vorlage hinweg. Jedes
Element beliefert bei jedem Schritt einen Lichtwert als Pixel an den Scanner. Bei einer Aufl"osung
von 300x300dpi in einem Feld von 1x1 Inch werden 90.000 Punkte erzeugt. Da die Anzahl
der Punkte in der horizontalen und vertikalen Achse identisch sind, spricht man abk"urzend
nur jeweils von der Anzahl der Punkte in einer Ebene. Bei 300dpi handelt es sich um eine
Aufl"osung, die normalerweise f"ur Texte im Zeitungs-- und Zeitschriftenbereich ausreichend ist.
Sind allerdings sehr kleine Schriften zu erkennen, mu"s die Aufl"osung erh"oht werden, um ein gutes
Ergebnis zu erzielen.

\item {\em Bildaufbereitung:}\\
Bedingt durch die Vorlagenqualit"at sowie anderer Beeintr"achtigungen,
wie z.B.\ Papierfehler, Knicke oder Staub auf der Glasfl"ache des Scanners,
kann die Erkennung nachhaltig gest"ort werden. Zur Beseitigung dieser
St"orungen ist die Anwendung verschiedenartiger Filter\-ungs\-methoden zu empfehlen. Verfahren
zur Beseitigung von {\em Salt and Pepper Noise\/} (wei"se Einschl"usse in schwarzen Gebieten
und schwarze St"orungen auf wei"sem Hintergrund) sind z.B\ in \cite{Bartneck90} beschrieben.
Zus"atzlich kann durch unsachgem"a"ses Einlegen oder falsch justierte 
automatische Papiereinz"uge die Vorlage schr"ag erfa"st werden. 
Ist dies der Fall, mu"s das schr"ag vorliegende Dokumentbild horizontal ausgerichtet (rotiert) werden.

\item {\em Segmentierung:}\\
Das Dokument wird in seine geometrische Struktur zerlegt, d.h.\
in Spalten bzw.\ Bl"ocke, Textzeilen, W"orter und letztendlich in
Schriftzeichen \cite{Bohnacker93}. Zuvor m"ussen grafische Bestandteile von
den Textbereichen getrennt werden. Die Segmentierung in Spalten
bzw.\ Bl"ocke ist Bestandteil dieser Arbeit.

\item {\em Zeichenbedeutung:}\\
Die bei der Segmentierung gefundenen geometrischen Objekte werden zu
einer oder mehreren Bedeutungshypothesen zugeordnet \cite{Schuermann77}, d.h.\ es wird jedes
einzelne Schriftzeichen erkannt (Klassifikation).

\item {\em Kontextverfahren:}\\
Konkurrierende Zeichenbedeutungen werden durch Betrachtung der Nachbarzeichen bewertet (Kontext).
%geometrischer Kontext, linguistischer Kontext, etc.

\item {\em Interpretation:}\\
Im letzten Schritt wird versucht das Dokument in verschiedene
Textkategorien einzuteilen (Rechnung, Bestellung, Gesch"aftsbrief
etc.). Zus"atzlich werden strukturierte und unstrukturierte
Informationen extrahiert, und ggf.\ ein Antwortschreiben automatisch
generiert.
\end{enumerate}

\section{Das Projekt InfoPort}
\label{InfoPort}

Die Arbeiten der Abteilung {\sl Textverstehende Systeme\/} des
Daimler--Benz Forschungszentrums in Ulm befassen sich im wesentlichen mit 
der Erkennung von Schriftzeichen --- maschinengeschriebenen, handblockschriftlichen, kursiven oder 
Kanji\footnote{Japanische Schriftzeichen} ---  und
der Interpretation von Texten. Diese Texte k"onnen als bildhafte Daten
(Papierdokument) oder als elektronischer Text repr"asentiert
sein. Ziel des Projektes {\sl InfoPort\/} \cite{InfoPortLab} ist es,
die Information aus diesen Texten im Kontext einer speziellen Anwendung
zu extrahieren, um die Weiterverarbeitung dieser Information zu unterst"utzen,
insbesondere, wenn es sich um Routinearbeit handelt. 
Zu diesen Hilfen z"ahlen unter anderem die automatische
Indexierung und Speicherung in Datenbanken, die automatische
Informationsweiterleitung an die zust"andige Person und die
Unterst"utzung bei der Bearbeitung der Dokumente.

Als beispielhafte Anwendung wurde der Text der Dokumentklasse
Gesch"aftsbriefe in einem speziellen Anwendungskontext (Anfragen
nach Zusendung eines Daimler--Benz Ge\-sch"afts\-berichts) gew"ahlt. Das
Ergebnis dieser Anwendung ist ein automatisch generiertes
Antwortschreiben mit passender Adressierung an den Absender.

Die Analyse erfolgt in vier Arbeitsschritten:
\begin{itemize}
  
  \item {\sl Dokumentbildanalyse:}\\
    Transformation des Bildes in elektronischen Text --
    optische Zeichenerkennung.\footnote{{\em O\/}ptical {\em C\/}haracter {\em R\/}ecognition (OCR)}

  \item {\sl Textkategorisierung:}\\
    Bestimmung des Textthemas.
  
  \item {\sl Informationsextraktion aus strukturierten Texten:}\\
    Interpretation der strukturierten Textteile (Absender, Empf"anger,
    etc.).

  \item {\sl Informationsextraktion aus unstrukturierten Texten:}\\
    Interpretation des Briefrumpfes

\end{itemize}

Ziel weiterer Forschungsaktivit"aten ist es, die Robustheit zu
steigern und das Wissen, das die einzelnen Verfahren zur Interpretation ben"otigen,
automatisch zu erwerben. Die Methoden werden f"ur weitere 
konkrete Beispiele umgesetzt, u.a. auf Steuerbescheide (Unterst"utzung von Steuerberatern bei
Einspr"uchen) und auf per Telefax eingehende Stra"senverkehrsmeldungen (Aktualisierung einer digitalen Stra"senkarte).

\section{Einordnung der Arbeit in den Analyseablauf}

Der bisherige Segmentierungsproze"s ging von der Annahme aus, da"s die Dokumentseite
ausschlie"slich Text einer Schriftgr"o"se beinhaltet. 
Es wurde die statistisch dominierende Zeichengr"o"se
bestimmt (s.\ \BildRef{AnalyseBisherJetzt}), die 
alle nachfolgenden Analysemethoden beeinflu"ste (Segmentierung in
Zeilen, W"orter und Zeichen) \cite{Bohnacker93}. Diese Annahme f"uhrt bei Dokumenten
mit deutlich variierenden Schriftgr"o"sen (wie in \BildRef{UlrichsProbleme}) zu Fehlern.
Die bisherigen Methoden arbeiten zudem nur mit einspaltigem Text. 
Bei mehreren Spalten mu"ste bisher jede einzelne Spalte manuell segmentiert werden, 
wobei auf die logische Lesefolge nicht geachtet wurde.

\Bild{AnalyseBisherJetzt}{8}{Vergleich der Segmentierungsabl"aufe}

Um dies zu vermeiden, mu"s das Dokument in Bereiche eingeteilt werden, welche gleichartige
(homogenen) Textobjekte enthalten. Aus diesen Bereichen k"onnen nun jeweils die f"ur die
weitere Segmentierung wichtigen Gr"o"sen abgeleitet werden: Zeichengr"o"se, Strichdicke,
Zeilenabstand, etc.

Somit stellt diese Arbeit -- die Segmentierung in homogene Textbl"ocke (Paragraphen) --
die logische und konsequente Fortsetzung und Verbesserung der bestehenden Methoden zur
Dokumentanalyse dar.

\Bild{UlrichsProbleme}{14}{Fehlerhafte Zeilensegmentierung durch global berechnete Parameter (Verbessertes Ergebnis durch vorgeschaltete Textblocksegmentierung siehe \protect\BildRef{UlrichsProblemGeloest} Seite \protect\pageref{eps:UlrichsProblemGeloest})}
