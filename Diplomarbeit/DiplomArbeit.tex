% !TEX root = DiplomArbeit.tex

%%%%%%%%%%%%%%%%%%%%%%%%%%%%%%%%%%%%%%%%%%%%%%%%%%%%%%%%%%%%%%%%%%%%%%%%%%%%%%%%%
%%                              D I P L O M A R B E I T                        %%
%%%%%%%%%%%%%%%%%%%%%%%%%%%%%%%%%%%%%%%%%%%%%%%%%%%%%%%%%%%%%%%%%%%%%%%%%%%%%%%%%
%%      von    Reiner Kräutle              Abteilung: F3M/T                    %%                    
%%%%%%%%%%%%%%%%%%%%%%%%%%%%%%%%%%%%%%%%%%%%%%%%%%%%%%%%%%%%%%%%%%%%%%%%%%%%%%%%%

\documentclass[12pt,a4paper,twoside]{report}

\usepackage[ngerman]{babel}
\usepackage[hidelinks]{hyperref}
\hypersetup{
    pdftitle={Bestimmung von Textblocken in strukturierten Dokumenten auf der Basis von Zusammenhangsgebieten},
    pdfsubject={Diplomarbeit},
    pdfauthor={Reiner Kräutle}
}

\usepackage[top=3cm,bottom=3cm,left=2cm,right=2cm]{geometry}
\usepackage{parskip}
\usepackage{setspace}
\setstretch{1.15}

\usepackage{fancyhdr}
\pagestyle{fancyplain}
\fancyhf{}
\fancyhead[LE]{\small \em \rightmark}
\fancyhead[RO]{\small \em \leftmark}
\fancyfoot[LE,RO]{\small \em \thepage}
\renewcommand{\plainheadrulewidth}{0.4pt}
\renewcommand{\plainfootrulewidth}{0.4pt}
\renewcommand{\footrulewidth}{0.4pt}
\addtolength{\headheight}{3pt}
\renewcommand{\chaptermark}[1]{\markboth{\thechapter.\ \chaptername\ #1}{}}
\renewcommand{\sectionmark}[1]{\markright{\thesection\ #1}}

\usepackage{graphicx}
\newcommand{\Bild}[3]{
  \begin{figure}[htbp]
    \vskip 2mm
    \centerline{\includegraphics[width=#2cm]{Pictures/#1.eps}}
    \caption{\label{eps:#1} #3}
  \end{figure}
}
\newcommand{\BildHier}[3]{
  \vskip 2mm
  \centerline{\includegraphics[width=#2cm]{Pictures/#1.eps}}
  \stepcounter{figure}
  \begin{center}{\figurename\thefigure:~#3}\end{center}
}
\newcommand{\BildPur}[2]{
  \centerline{\includegraphics[width=#2cm]{Pictures/#1.eps}}
}

\renewcommand*{\thefootnote}{\fnsymbol{footnote}}
\def\figurename{Abb.}
\newcommand{\BildRef}[1]{\figurename~\ref{eps:#1}}
\newcommand{\KapRef}[1]{Kap.~\ref{#1}}
\def\re{\hbox{$I\kern -0.275em R$}}

\begin{document}
%
\thispagestyle{empty}
%
\BildPur{UniUlmLogoScanX}{4}
%
\begin{center}
  \Large\bf
  Bestimmung von Textbl"ocken in strukturierten
  Dokumenten auf der Basis von Zusammenhangsgebieten
\end{center}

\vfill

\begin{center}
  \large
  Diplomarbeit\\
  von\\
  Reiner Kr"autle
\end{center}

\vfill

\begin{center}
  Abteilung Allgemeine Elektrotechnik und Mikroelektronik\\
  Universit"at Ulm\\
  August 1995
\end{center}

\newpage\thispagestyle{empty}

\begin{center}
  \large
  Abteilung Allgemeine Elektrotechnik und Mikroelektronik\\
  Universit"at Ulm
\end{center}
\vfill

\centerline{09.\ Februar bis 08.\ August 1995}
\vfill

\begin{center}
  \begin{large}
    \begin{tabular}{rl}
      Bearbeiter:                & Reiner Kr"autle              \\[5mm]
      Pr"ufer:                   & Prof.~Dr.--Ing.~A.~Rothermel \\[5mm]
      Betreuer in der Industrie: & Dr.~Thomas Bayer,            \\
                                 & Daimler--Benz AG, F3M/T
    \end{tabular}
  \end{large}
\end{center}

\clearpage{\thispagestyle{empty}\cleardoublepage}
\thispagestyle{empty}
\begin{small}
  \centerline{\bf Kurzfassung}

  Es wird ein Verfahren zur Segmentierung des Dokumentbildes in Textbl"ocke ausgehend
  von Zusammenhangsobjekten beschrieben.
  Voraussetzung ist, da"s die Dokumente nur maschinengeschriebene Textgebiete
  enthalten, eine Winkelkorrektur bereits vorgenommen wurde und da"s sie aus dem romanischen
  Sprachraum entstammen. Bez"uglich der Layoutstruktur wird keine Einschr"ankung gemacht und die
  Verfahren werden automatisch mit den vorliegenden Daten parametriert.

  Textbl"ocke werden definiert als eine Menge von homogenen Zusammenhangsgebieten. Sie sind
  charakterisiert durch die mittlere Buchstabengr"o"se und --st"arke, das umschreibende Rechteck
  und der Position innerhalb der logischen Lesefolge.

  Das Verfahren basiert auf einer Kombination der Top--Down--Analyse (XY--Cut) und der
  Bottom--Up--Analyse (Subsampling und Smearing).
  Die ben"otigten Smearing--Parameter werden "uber den Zeilenabstand lokal
  (innerhalb der bei den XY--Cuts entstandenen Gebieten) aus dem Dokument berechnet.
  Die Textblockhypothesen werden in den folgenden beiden Schritten bez"uglich zweier
  Homogenit"atskriterien "uberpr"uft: Textblockhypothesen werden aufgespalten, wenn die
  mittlere Buchstabenh"ohe oder --st"arke in folgenden Zeilen stark variiert;
  Textblockhypothesen werden zusammengefa"st, wenn r"aumlich benachbarte homogene
  Textblockhypothesen in vorigen Schritten getrennt wurden.
  Im Anschlu"s werden die gefundenen Textbl"ocke unter Ber"ucksichtigung ihrer geometrischen Struktur nach ihrer logischen Lesefolge sortiert.

  Das Stichprobenmaterial umfa"st 163 Dokumente aus den Klassen der Gesch"aftsbriefe,
  wissenschaftlichen Artikel, Zeitschriften, Tageszeitungen und Steuerbescheide,
  aufgeteilt in eine Trainingsstichprobe von 47 Dokumenten,
  an derer die Verfahren erarbeitet und parametriert wurden, und in eine Teststichprobe
  von 116 Dokumenten, an derer die Leistungsf"ahigkeit und Grenzen aufgezeigt wurden.
  Die Verfahren zeigen gute bis sehr gute Ergebnisse bei den betrachteten Dokumentklassen.
  Die Erkennungsrate (Verh"altnis richtig erkannter Textbl"ocke zu der Gesamtanzahl der erkannten
  Textbl"ocke)
  liegt f"ur alle Dokumentklassen im Mittel zwischen 90 und 95\%.
\end{small}
\vfill
{\bf Erkl"arung}

Hiermit erkl"are ich, da"s ich diese Diplomarbeit selbstst"andig verfa"st und keine anderen
als die angegebenen Quellen und Hilfsmittel verwendet habe.

\vspace*{2cm}

Ulm, den 08.\ August 1995\hfill
\begin{minipage}[t]{5cm}
  \centerline{Reiner Kr"autle}
  \centerline{\small (Matr.~Nr.~223875)}
\end{minipage}
\hspace*{\fill}
\clearpage{\thispagestyle{empty}\cleardoublepage}

\pagenumbering{roman}
\parskip1.3ex plus0.5ex minus 0.5ex
\tableofcontents
\parskip1.5ex plus0.5ex minus 0.5ex
\clearpage{\thispagestyle{empty}\cleardoublepage}
\pagenumbering{arabic}

\chapter{Einleitung und Motivation}
\section{Dokumentanalyse}
\label{DokAna}

Das Papier als Informationstr"ager ist in unserer heutigen Zeit nicht
mehr wegzudenken. Die meisten B"ucher, Zeitschriften und Artikel, vor
allem "alteren Jahrgangs, sind nur in ihrer gedruckten Form
erh"altlich. Dennoch m"ochte man die Papierberge, die sich auf
Schreibtischen und in Aktenschr"anken t"urmen, mit dem
B"urocomputer elektronisch verarbeiten. 
Schon vor Jahren wurde von den Propheten der Computergesellschaft das
Zeitalter der {\em papierlosen B"uros\/} angek"undigt. 
Der Wunsch ganz auf das Papier
zu verzichten liegt deshalb nahe, er wird sich jedoch auf lange Zeit nicht verwirklichen lassen.
Um diesem Ziel n"aher zu kommen, mu"s die
auf das Papier gebrachte Information zun"achst in eine dem Computer
verst"andliche Form konvertiert und interpretiert werden. Anschlie"send m"ussen
entsprechende Handlungen ausgel"ost werden. Dies sind Aufgaben der
{\em Dokumentanalyse\/} (s.\ \BildRef{Dokumentanalyse} und
\cite{PixelToContents}):

\Bild{Dokumentanalyse}{15}{Ablauf der Dokumentanalyse}

\begin{enumerate}

\item {\em Digitalisierung:}\\
Das grundlegende Funktionsprinzip eines Scanners besteht in dem zeilenweisen 
Abtasten (engl.\ {\em to scan\/}) der
Vorlagen durch eine Leiste lichtempfindlicher Sensoren. 
Beim Digitalisieren wird die Vorlage mit Hilfe einer starken
Lichtquelle angestrahlt, wodurch das Bild auf eine Reihe von CCD--Elementen
({\sl C\/}harge {\sl C\/}oupled {\sl D\/}evices) geworfen wird. Diese
Sensoren nehmen das unterschiedlich reflektierte Licht auf und wandeln
die Helligkeitsinformation in digitale Werte. F"ur die Texterkennung
gen"ugen i.d.R.\ Ein--Bit--Scanner, so da"s nach der Digitalisierung ein
Bin"arbild  zur Verf"ugung steht, d.h.~jeder Bildpunkt wird mit einem Bit kodiert.

Zu dem wichtigsten Qualit"atsmerkmal eines Scanners geh"ort die Aufl"osungsf"ahigkeit. 
Es hat sich 300dpi ({\em d\/}ots {\em p\/}er {\em i\/}nch) 
in der Texterkennung als Standard etabliert. 
Jeweils 300 CCD--Elemente liegen pro Inch nebeneinander.
Die CCD--Zeile bewegt sich au"serdem mit 300 Schritten pro Inch unter der Vorlage hinweg. Jedes
Element beliefert bei jedem Schritt einen Lichtwert als Pixel an den Scanner. Bei einer Aufl"osung
von 300x300dpi in einem Feld von 1x1 Inch werden 90.000 Punkte erzeugt. Da die Anzahl
der Punkte in der horizontalen und vertikalen Achse identisch sind, spricht man abk"urzend
nur jeweils von der Anzahl der Punkte in einer Ebene. Bei 300dpi handelt es sich um eine
Aufl"osung, die normalerweise f"ur Texte im Zeitungs-- und Zeitschriftenbereich ausreichend ist.
Sind allerdings sehr kleine Schriften zu erkennen, mu"s die Aufl"osung erh"oht werden, um ein gutes
Ergebnis zu erzielen.

\item {\em Bildaufbereitung:}\\
Bedingt durch die Vorlagenqualit"at sowie anderer Beeintr"achtigungen,
wie z.B.\ Papierfehler, Knicke oder Staub auf der Glasfl"ache des Scanners,
kann die Erkennung nachhaltig gest"ort werden. Zur Beseitigung dieser
St"orungen ist die Anwendung verschiedenartiger Filter\-ungs\-methoden zu empfehlen. Verfahren
zur Beseitigung von {\em Salt and Pepper Noise\/} (wei"se Einschl"usse in schwarzen Gebieten
und schwarze St"orungen auf wei"sem Hintergrund) sind z.B\ in \cite{Bartneck90} beschrieben.
Zus"atzlich kann durch unsachgem"a"ses Einlegen oder falsch justierte 
automatische Papiereinz"uge die Vorlage schr"ag erfa"st werden. 
Ist dies der Fall, mu"s das schr"ag vorliegende Dokumentbild horizontal ausgerichtet (rotiert) werden.

\item {\em Segmentierung:}\\
Das Dokument wird in seine geometrische Struktur zerlegt, d.h.\
in Spalten bzw.\ Bl"ocke, Textzeilen, W"orter und letztendlich in
Schriftzeichen \cite{Bohnacker93}. Zuvor m"ussen grafische Bestandteile von
den Textbereichen getrennt werden. Die Segmentierung in Spalten
bzw.\ Bl"ocke ist Bestandteil dieser Arbeit.

\item {\em Zeichenbedeutung:}\\
Die bei der Segmentierung gefundenen geometrischen Objekte werden zu
einer oder mehreren Bedeutungshypothesen zugeordnet \cite{Schuermann77}, d.h.\ es wird jedes
einzelne Schriftzeichen erkannt (Klassifikation).

\item {\em Kontextverfahren:}\\
Konkurrierende Zeichenbedeutungen werden durch Betrachtung der Nachbarzeichen bewertet (Kontext).
%geometrischer Kontext, linguistischer Kontext, etc.

\item {\em Interpretation:}\\
Im letzten Schritt wird versucht das Dokument in verschiedene
Textkategorien einzuteilen (Rechnung, Bestellung, Gesch"aftsbrief
etc.). Zus"atzlich werden strukturierte und unstrukturierte
Informationen extrahiert, und ggf.\ ein Antwortschreiben automatisch
generiert.
\end{enumerate}

\section{Das Projekt InfoPort}
\label{InfoPort}

Die Arbeiten der Abteilung {\sl Textverstehende Systeme\/} des
Daimler--Benz Forschungszentrums in Ulm befassen sich im wesentlichen mit 
der Erkennung von Schriftzeichen --- maschinengeschriebenen, handblockschriftlichen, kursiven oder 
Kanji\footnote{Japanische Schriftzeichen} ---  und
der Interpretation von Texten. Diese Texte k"onnen als bildhafte Daten
(Papierdokument) oder als elektronischer Text repr"asentiert
sein. Ziel des Projektes {\sl InfoPort\/} \cite{InfoPortLab} ist es,
die Information aus diesen Texten im Kontext einer speziellen Anwendung
zu extrahieren, um die Weiterverarbeitung dieser Information zu unterst"utzen,
insbesondere, wenn es sich um Routinearbeit handelt. 
Zu diesen Hilfen z"ahlen unter anderem die automatische
Indexierung und Speicherung in Datenbanken, die automatische
Informationsweiterleitung an die zust"andige Person und die
Unterst"utzung bei der Bearbeitung der Dokumente.

Als beispielhafte Anwendung wurde der Text der Dokumentklasse
Gesch"aftsbriefe in einem speziellen Anwendungskontext (Anfragen
nach Zusendung eines Daimler--Benz Ge\-sch"afts\-berichts) gew"ahlt. Das
Ergebnis dieser Anwendung ist ein automatisch generiertes
Antwortschreiben mit passender Adressierung an den Absender.

Die Analyse erfolgt in vier Arbeitsschritten:
\begin{itemize}
  
  \item {\sl Dokumentbildanalyse:}\\
    Transformation des Bildes in elektronischen Text --
    optische Zeichenerkennung.\footnote{{\em O\/}ptical {\em C\/}haracter {\em R\/}ecognition (OCR)}

  \item {\sl Textkategorisierung:}\\
    Bestimmung des Textthemas.
  
  \item {\sl Informationsextraktion aus strukturierten Texten:}\\
    Interpretation der strukturierten Textteile (Absender, Empf"anger,
    etc.).

  \item {\sl Informationsextraktion aus unstrukturierten Texten:}\\
    Interpretation des Briefrumpfes

\end{itemize}

Ziel weiterer Forschungsaktivit"aten ist es, die Robustheit zu
steigern und das Wissen, das die einzelnen Verfahren zur Interpretation ben"otigen,
automatisch zu erwerben. Die Methoden werden f"ur weitere 
konkrete Beispiele umgesetzt, u.a. auf Steuerbescheide (Unterst"utzung von Steuerberatern bei
Einspr"uchen) und auf per Telefax eingehende Stra"senverkehrsmeldungen (Aktualisierung einer digitalen Stra"senkarte).

\section{Einordnung der Arbeit in den Analyseablauf}

Der bisherige Segmentierungsproze"s ging von der Annahme aus, da"s die Dokumentseite
ausschlie"slich Text einer Schriftgr"o"se beinhaltet. 
Es wurde die statistisch dominierende Zeichengr"o"se
bestimmt (s.\ \BildRef{AnalyseBisherJetzt}), die 
alle nachfolgenden Analysemethoden beeinflu"ste (Segmentierung in
Zeilen, W"orter und Zeichen) \cite{Bohnacker93}. Diese Annahme f"uhrt bei Dokumenten
mit deutlich variierenden Schriftgr"o"sen (wie in \BildRef{UlrichsProbleme}) zu Fehlern.
Die bisherigen Methoden arbeiten zudem nur mit einspaltigem Text. 
Bei mehreren Spalten mu"ste bisher jede einzelne Spalte manuell segmentiert werden, 
wobei auf die logische Lesefolge nicht geachtet wurde.

\Bild{AnalyseBisherJetzt}{8}{Vergleich der Segmentierungsabl"aufe}

Um dies zu vermeiden, mu"s das Dokument in Bereiche eingeteilt werden, welche gleichartige
(homogenen) Textobjekte enthalten. Aus diesen Bereichen k"onnen nun jeweils die f"ur die
weitere Segmentierung wichtigen Gr"o"sen abgeleitet werden: Zeichengr"o"se, Strichdicke,
Zeilenabstand, etc.

Somit stellt diese Arbeit -- die Segmentierung in homogene Textbl"ocke (Paragraphen) --
die logische und konsequente Fortsetzung und Verbesserung der bestehenden Methoden zur
Dokumentanalyse dar.

\Bild{UlrichsProbleme}{14}{Fehlerhafte Zeilensegmentierung durch global berechnete Parameter (Verbessertes Ergebnis durch vorgeschaltete Textblocksegmentierung siehe \protect\BildRef{UlrichsProblemGeloest} Seite \protect\pageref{eps:UlrichsProblemGeloest})}

\chapter{Aufgabenbeschreibung}

Eine wesentliche Teilaufgabe bei der Analyse eines Textdokuments ist die Segmentierung
des Dokumentbildes in Textbl"ocke ({\em Zoning\/}). Wenn die Textbl"ocke bestimmt sind, k"onnen
die Zeilen, W"orter und die Schriftzeichen in diesem Textblock synthetisiert und somit
auch klassifiziert werden.

Die Aufgabe der Diplomarbeit besteht darin, die in der Literatur bekannten Verfahren
zur Segmentierung in Textbl"ocke zu sichten (\KapRef{LayoutanalyseMethoden}), geeignete
Verfahren f"ur die zu bearbeiteten Dokumenttypen (Anhang \ref{DokuKlassen}) 
auszuw"ahlen, zu modifizieren und geeignet zu kombinieren.
Die Erkennungsleistung des entwickelten Segmentierungsverfahrens und deren Parameter
sollen an einer Lernstichprobe optimiert und an einer Teststichprobe deren Leistungsf"ahigkeit
und Grenzen aufgezeigt werden (\KapRef{ErgebnisZoning}).

Die Basis der zu entwickelnden Verfahren bilden Zu\-sammen\-hangs\-objekte
(\KapRef{Zusammenhangsanalyse}), die das Bin"arbild durch Konturen dieser Gebiete kodieren.
Diese Vielzahl dieser primitiven Objekte -- von 1000 bis weit "uber 10000 -- mu"s
zu Textbl"ocken mit homogenen Objekten zusammengefa"st werden. Homogenit"at bezieht
sich dabei auf die Eigenschaften, die sich aus den Zusammenhangsobjekten ableiten
lassen (\KapRef{Homogenitaet}). Insbesondere soll mehrspaltiges Layout unterst"utzt und die
nat"urliche Abfolge der Textbl"ocke ermittelt werden (\KapRef{Lesefolge}). 

Die entwickelten Algorithmen werden in die bestehende Entwicklungsumgebung
(NeXT Workstation unter einer grafischen Programmierumgebung) eingebunden 
(Anhang \ref{Experimentierumgebung}),
die die Entwicklung von Verfahren durch eine Vielzahl von bestehenden Software--Moduln unterst"utzt.

\newpage
{\large \bf Anforderungen}\label{Anforderungen}

Folgende Anforderungen sollen an die zu erarbeitenden Methoden gestellt werden:

\begin{itemize}

  \item Hauptanwendung: Gesch"aftsbriefe aus dem Umfeld {\em Investor Relations\/}
        (Anfragen bez"uglich Gesch"aftsberichte), wissenschaftliche 
	Ver"offentlichungen, Zeitschriften, Tageszeitungen und Steuerbescheide.
	Typische Beispiele dieser Dokumentklassen siehe Anhang \ref{DokuKlassen}.

  \item Mehrspaltiges Layout mit beliebigem Konturverlauf.

  \item Dokumente beinhalten ausschlie"slich Textgebiete und sind im Winkel horizontal
  ausgerichtet.
  
  \item Dokumente sind dem romanischen Sprachraum entnommen, d.h.\ ihre Schreibrichtung ist
        von links nach rechts, und von oben nach unten.
  
  \item Das zu entwickelnde Verfahren soll nur layoutunabh"angige oder selbstadaptierende
  Parameter verwenden.
	
  \item Transparenz der Parameter, d.h\ alle (soweit sinnvoll) Parameter sollen dem Benutzer
        zug"anglich und vor Beginn der Analyse ver"anderbar sein.
	
  \item Integration in die bestehende Entwicklungs-- bzw.\ Experimentierumgebung.
  
\end{itemize}

\chapter{Grundlegende Techniken}
\label{GrundlegendeTechniken}

Die in diesem Abschnitt beschriebenen Techniken dienen als Grundlage f"ur die Layoutanalysemethoden 
in Kapitel \ref{LayoutanalyseMethoden}. Kapitel \ref{Morphologie} und \ref{Zusammenhangsanalyse}
stellen Methoden der Bildverarbeitung vor:

\begin{description}
  \item[Ikonische Bildverarbeitung:] Die ikonische Bildverarbeitung besch"aftigt sich mit 
  Operationen, die ausschlie"slich mit einzelnen Bildpunkten arbeiten (\KapRef{Morphologie}).

  \item[Symbolische Bildverarbeitung:] Nach der Transformation des Bin"arbildes in den
    symbolischen Bereich stehen den symbolischen Bildverarbeitungsmethoden Objekte
    mit Informationen "uber ihre Kontur und Farbe zur Verf"ugung
    (\KapRef{Zusammenhangsanalyse} und \ref{Homogenitaet}).
    
\end{description}
Die Histogrammanalyse wird in Kapitel \ref{Histogrammanalyse} als Methode zur Bestimmung einer
Verteilungsfunktion aus einer Stichprobe vorgestellt.  
Diese grundlegenden Verfahren sind nicht Gegenstand dieser Arbeit, Implementationen hiervon existieren bereits in der verwendeten Softwareumgebung.

%%%%%%%%%%%%%%%%%%%%%%%%%%%%%%%%%%%%%%%%%%%%%%%%%%%%%%%%%%%%%%%%
\section{Morphologische Operationen}
\label{Morphologie}

In diesem Kapitel werden Methoden der ikonischen Bildverarbeitung beschrieben. Diese
Verfahren ben"otigen als Eingangsdaten ein Bin"arbild und liefern als Ergebnis
wiederum ein Bin"arbild. 

Lineare Filtermethoden (wie z.B.\ Tiefpa"sfilter) arbeiten in einem Wertebereich
au"serhalb des bei der Klassifikation von Schriftzeichen verwendeten Bereichs von
nur zwei Grauwerten. Diese Methoden liefern bei der Anwendung auf bin"are Muster
im allgemeinen keine Ergebnisse, deren Werte ebenfalls wieder bin"ar sind. Dies mu"s
in einer nachfolgenden Schwellwertoperation sichergestellt werden. Eine andere
M"oglichkeit besteht darin, f"ur bin"are Muster spezialisierte nichtlineare Operationen
zu verwenden \cite{Niemann83}. Beispielsweise haben lineare Operationen bei der Gl"attung
den Nachteil, da"s sie nicht nur hochfrequente St"orungen beseitigen, sondern auch 
Bildkonturen verschleifen. Um diesen Nachteil zu vermeiden wurden nichtlineare Operationen
entwickelt, die auf der Ordnung der Funktionswerte einer kleinen Nachbarschaft beruhen.
Dieses ist auch das Prinzip der allgemeinen Rangordnungsoperationen. 
Man betrachtet einen Funktionswert
$f_{jk} \in \re$ aus einem zweidimensionalen Raum in der Folge $[f_{jk}]$ und 
bezeichnet mit $N_M$ eine Nachbarschaft von $f_{jk}$, die $M$ Werte enth"alt,
\begin{eqnarray*}
N_M &=& \left\{ f_{j+\mu, k+\nu} \;|\; \mu = 0, \pm 1,\ldots, \pm m;\; \nu = 0, \pm 1, \ldots, \pm n \right\},\\
M   &=& (2m + 1)(2n + 1).
\end{eqnarray*}
Die Elemente von $N_M$ werden der Gr"o"se nach geordnet, wobei der kleinste Wert aus $N_M$
mit $r_1$ bezeichnet wird, das n"achstgr"o"sere mit $r_2$ und so weiter. Dieses ergibt die
Rangordnung der um $f_{jk}$ liegenden Funktionswerte
$$R_{jk} = \left\{ r_1, r_2, \ldots, r_M \;|\; r_\nu \in N_M,\;r_\nu \le r_{\nu+1},\;\nu = 1,\ldots,M \right\}.$$
Eine Rangordnungsoperation ist definiert durch 
$$h_{jk} = \varphi\:(R_{jk}).$$
Drei spezielle Operationen sind
\[ \begin{array}{llcl}
\mbox{Erosion}          & h_{jk} &=& r_1,\\
\mbox{Median}           & h_{jk} &=& r_{(M + 1) / 2},\\
\mbox{Dilatation\qquad} & h_{jk} &=& r_M.
\end{array} \]
Angewandt auf Bin"arbilder mit $f_{jk} \in \{ 0, 1 \}$ lassen sich diese Operationen wie folgt anschaulich beschreiben:
\begin{description}
  \item[Erosion:] Ein Bildpunkt wird dann gesetzt, wenn {\em alle} Bildpunkte innerhalb 
  der Maske $N_M$
  gesetzt sind. Somit schrumpfen Schwarzbereiche zusammen (\BildRef{Erosion}).
  \Bild{Erosion}{7}{Ausgangsbild links, Ergebnis der Erosion rechts}
  \item[Median:]  Ein Bildpunkt wird dann gesetzt, wenn mindestens die {\em H"alfte} der Bildpunkte 
  innerhalb der Maske gesetzt ist. Dadurch l"a"st sich Rauschen (Pepper Noise) von einer
  Fl"ache mit dem Fl"acheninhalt $g$ l"oschen, wenn $g \le \frac{M}{2}$ gilt.
  \item[Dilatation:] Ein Bildpunkt wird dann gesetzt, wenn {\em mindestens ein} Bildpunkt 
  innerhab der Maske gesetzt ist. Somit breiten sich Schwarzbereiche aus.
\end{description}

Um St"orungen, die eine gewisse Gr"o"se nicht "uberschreiten, zu beseitigen, k"onnen Dilatation und
Erosion kombiniert eingesetzt werden:

\begin{description}
  \item[Opening:] Die Aufeinanderfolge von Erosion und Dilatation wird als {\em Opening\/}--
  Operation bezeichnet. Rauschen, bedingt durch eine schlechte Vorlagenqualit"at,
  kann mit diesem Filter abgeschw"acht werden (\BildRef{Opening}).
  
  \Bild{Opening}{7}{Reduzierung von Rauschen mit dem Opening--Filter.}
  
  \item[Closing:] Die umgekehrte Abfolge, also Dilatation mit anschlie"sender Erosion ist
  als \mbox{\em Closing\/} definiert. Sie dient dazu Zwischenr"aume zu schlie"sen, wie zum Beispiel
  das Verbinden zerbrochener Buchstaben (\BildRef{Closing}).
  
  \Bild{Closing}{3.5}{Zerbrochene Buchstaben werden durch die Closing--Operation verbunden} 
\end{description}

%\clearpage
%%%%%%%%%%%%%%%%%%%%%%%%%%%%%%%%%%%%%%%%%%%%%%%%%%%%%%%%%%%%%%%%
\section{Zusammenhangsanalyse}
\label{Zusammenhangsanalyse}

Ausgangsbasis f"ur die Segmentierung ist ein Bin"arbild. Es besteht aus einer Matrix aus
kleinen quadratischen Bildelementen, denen jeweils ein bestimmter Farbwert, entweder schwarz
oder wei"s, zugeordnet ist. Jedes Bildelement, im weiteren Pixel genannt, l"a"st sich "uber
eine Koordinatenangabe in einem orthogonalen Koordinatensystem ansprechen 
(s.~\BildRef{Binaerbild}). Merkmale "uber den Inhalt des Bildes (wie z.B.\ die Buchstabenh"ohe)
lassen sich hierbei nur schwer berechnen.

\Bild{Binaerbild}{3}{Repr"asentation eines Bin"arbildes als Pixelmatrix}

Die Zusammenhangsanalyse, BCC--Analyse ({\em B\/}inary {\em C\/}onnected {\em C\/}omponent)
genannt, "uberf"uhrt ein Rasterbild nicht wiederum in ein anderes Rasterbild, sondern in
Bildobjekte. Dabei werden aus der Bildmatrix topologisch
zusammenh"angende Mengen schwarzer Pixel, die sogenannten Zusammenhangskomponenten, extrahiert.
Dieser Verfahrensschritt ist datenerhaltend und deshalb reversibel. Es wird
eine rechnerinterne Darstellung mit Hilfe einer hierarchischen Datenstruktur verwendet. In
ihr werden die R"ander (Kontur) eines Objekts sowie die {\em Enthaltensein\/}--Relation von
inneren Konturen zu ihren "au"seren Konturen vermerkt. Diese Darstellung erm"oglicht eine
effiziente Berechnung von Merkmalen (s.~\KapRef{Homogenitaet}), die f"ur
die Layoutanalyse ben"otigt werden. 

Die Kanten der Kontur bestehen
ausschlie"slich aus waagerechten und senkrechten Ge\-ra\-den\-st"u"cken, 
wobei einer senkrechten Kante
immer eine waagerechte folgt. Des weiteren kreuzen sich keine Kanten und die Kontur
ist stets geschlossen.
Ineinanderliegende Konturen werden als "ubereinanderliegende, sich "uberdeckende Zusammenhangsobjekte
alternierender Farben interpretiert. Diese hierarchische Zuordnung wird {\em Vater--Sohn} Relation
genannt; weiterhin wird zwischen Objekten, die unmittelbar in einem gemeinsamen
Zusammenhangsgebiet liegen, eine {\em Bruder} Beziehung definiert (s.~\BildRef{BccHierarchie}).

\Bild{BccHierarchie}{11.5}{Veranschaulichung der Hierarchiebeziehungen}

Neben der Kontur, der Farbe und den Hierarchiebeziehungen wird 
au"serdem das umschreibende Rechteck jedes Zusammenhangsgebietes berechnet 
({\em BCC--Bounding Box\/}).
Diese Rechtecke sind in der Layoutanalyse von besonderer Bedeutung 
(s.~\BildRef{BccBoxes}). Weitere Informationen und Implementierungsdetails sind \cite{MaOb90} 
zu entnehmen.

\Bild{BccBoxes}{13}{Bin"arbild, BCC--Konturen, umschreibende Rechtecke (v.l.n.r.)}

%%%%%%%%%%%%%%%%%%%%%%%%%%%%%%%%%%%%%%%%%%%%%%%%%%%%%%%%%%%%%%%%
\section{Histogrammanalyse}\label{Histogrammanalyse}

Es sei eine Stichprobe $(x_1,\ldots,x_n)$ mit dem Merkmal $X$ gegeben. Die Verteilung von $X$
sei nicht bekannt. Um bei einem qualitativen Merkmal eine Vorstellung von dieser Verteilung
zu erhalten, konstruiert man das sogenannte {\em Histogramm}.

Man f"uhrt eine Zerlegung der reellen Achse in endlich viele aneinander grenzende Intervalle 
$\Delta_1,\ldots,\Delta_k$, sogenannte {\em Klassen}, durch. Es wird nachgez"ahlt, wieviele
Werte $x \in X$ in $\Delta_j \; (1 \le j \le k)$ liegen. Die Anzahl $m_j$ der Objekte wird
{\em Klassenh"aufigkeit\/} genannt. 
"Uber $\Delta_j$ wird ein Rechteck der H"ohe $\frac{m_j}{n}$ gezeichnet
(relative Klassenh"aufigkeiten). Das so entstehende Stufenbild wird als Histogramm 
zum Merkmal $X$ bezeichnet.

Eine h"aufige Anwendung ist das Finden von Klassenh"aufigkeitsmaxima ({\em Peaks\/}) innerhalb der Verteilung
(z.B.\ bei der Zeilensuche \KapRef{Zeilenabstand}).
Hier empfiehlt es sich einen sogenannten {\em Gravitationsfilters\/} anzuwenden.
Bei diesem iterativen Filter wandern die
Histogrammeintr"age in Richtung der gr"o"sten `Massen', d.h.\ in Richtung der gr"o"sten Anh"aufung von
Histogrammeintr"agen. Der Einflu"s der `Schwerkraft' (Filterradius) ist variabel einstellbar.
Durch die wiederholte Anwendung werden die Maxima herausgearbeitet, und man erh"alt schlie"slich ein 
Histogramm, das idealerweise nur noch aus einzelnen Peaks besteht (s.\ \BildRef{DemoHistogramm}).

\Bild{DemoHistogramm}{7}{Histogramm ungefiltert (links) und nach der Gravitationsfilterung}


\chapter{Bestimmung von Textbl"ocken}\label{BestimmungZones}

\section{Stand der Technik in der Layoutanalyse}\label{LayoutanalyseMethoden}

Die in der Literatur beschriebenen Verfahren zur Segmentierung von Layoutbl"ocken
lassen sich grob in zwei Vorgehensweisen unterteilen (s.\ \BildRef{VergleichTopDownBottomUp}):

\Bild{VergleichTopDownBottomUp}{8}{Top--Down Verfahren (links) und Bottom--Up Verfahren.}

\subsection{Top--Down Vorgehen}\label{XYCutTheroie}

Die Analyse nach der Top--Down Strategie beginnt typischerweise mit einer sehr groben
Einteilung des Dokuments.
Diese Hypothesen werden in den nachfolgenden Schritten weiter verfeinert und die Analyse endet
bei der niedersten Stufe, den Pixeln oder Zusammenhangsgebieten.

George Nagy schlug in \cite{Nagy84} eine Methode vor, mit der ein
Dokument in eine hierarchische Datenstruktur (von den Autoren {\em XY--Tree\/} genannt)
"uberf"uhrt werden kann:
Durch alternierende Schnitte parallel zu der X-- bzw.\ Y--Achse ({\em XY--Cut\/}) wird die Seite in zwei oder mehrere Unterbl"ocke aufgeteilt (s.\ \BildRef{XYCutIdee}).
Dieses Verfahren wird so lange rekursiv angewandt, bis eine Seite nicht weiter segmentiert werden
kann. Die Position der Schnitte wird durch ein horizontales bzw.\ vertikales Projektionsprofil
der schwarzen Pixel bestimmt.

\Bild{XYCutIdee}{11}{Idee der rekursiven Schnitte (XY--Cuts)}

%\pagebreak
{\bf Vorteile:}
\begin{itemize}
  \item Anschauliche Vorgehensweise
  \item Transformation des Dokuments in eine leicht handhabbare Datenstruktur
\end{itemize}

{\bf Nachteile:}
\begin{itemize}
  \item Nur f"ur {\em Manhattan--Layouts} geeignet. Dies sind Layouts, die ausschlie"slich
        durch eine Menge von horizontalen und vertikalen Geraden, die durch wei"se Gebiete laufen,
        getrennt werden k"onnen, vergleichbar mit der Stra"senanordnung von Manhattan
        (Nicht--Manhattan--Layout s.\ \BildRef{DemoXYCutError} und
        \ref{eps:ZoningErg2} auf Seite \pageref{eps:ZoningErg2}).

  \item Erzeugung des Projektionsprofils aus dem Bin"arbild ist wegen der gro"sen Datenmengen sehr zeitintensiv.

  \item Die Bestimmung der Schnittpositionen, d.h.\ der Minima, innerhalb des Projektionsprofils
        ist, speziell bei einer sehr feinen Segmentierung, problematisch.

\end{itemize}

\Bild{DemoXYCutError}{7}{Unl"osbares Problem f"ur die XY--Cut Analyse}

\subsection{Bottom--Up Vorgehen}\label{BottomUp}

Diese Methode schreitet den umgekehrten Weg: Ausgehend von der kleinsten Detailierungsstufe
(Pixel oder Zusammenhangsgebiete) werden immer gr"o"sere Bl"ocke gebildet --
Buchstaben zu W"ortern, W"orter zu Zeilen, Zeilen zu Textbl"ocken -- solange,
bis die gesamte Seite bearbeitet wurde.

\begin{enumerate}

  \item {\bf Smearing:}\\
        Ziel des Smearings ist, kleine Schwarzgebiete (Buchstaben, W"orter und Zeilen) zu
        gr"o"seren Bl"ocken zu verschmelzen (s.\ \cite{Wahl} und \BildRef{DemoSmearing}). Wie der
        Name erkennen l"a"st, werden die Schwarzfl"achen des Bin"arbildes in horizontaler und/oder vertikaler
        Richtung {\it verschmiert}. Ein sehr wichtiger Faktor f"ur zufriedenstellende Ergebnisse
        ist hierbei die Wahl der Parameter (Smearing--Window), um nicht f"alschlicherweise mehrere
        nebeneinanderliegende Spalten zu verbinden. Da bei einer statischen Festlegung die
        unterschiedlichen Layoutstrukturen nicht ber"ucksichtigt werden k"onnen, ist eine direkte
        Parameterberechnung aus dem Dokument unbedingt notwendig.\label{Smearing}

        {\bf Vorteile:}
        \begin{itemize}
          \item einfache Implementierung mittels der morphologischen Operationen Dilatation und Erosion (Closing)
          \item gute Ergebnisse bei einer Vielzahl von Layouttypen
        \end{itemize}

        {\bf Nachteile:}
        \begin{itemize}
          \item Lokale Parametrierung notwendig.
          \item Das Smearing verbindet jede Art von Objekten, so da"s Konsistenzpr"ufungen notwendig
                werden.
          \item Da die Smearingalgorithmen auf der Pixelebene mit einem gro"sen Datenvolumen arbeiten,
                sind die Laufzeiten entsprechend hoch.
        \end{itemize}

        \Bild{DemoSmearing}{8}{Smearing an einem Beispiel.}

  \item {\bf Unterabtastung (Subsampling):}\\
        Hierbei wird das Bin"arbild in nicht"uberlappende Rechtecke
        konstanter Gr"o"se eingeteilt. Jeder dieser Bl"ocke wird nun auf einen Pixel abgebildet
        (reduziert). Beispiele siehe \BildRef{DemoSubsampling} und \BildRef{Subsampling} auf Seite
        \pageref{eps:Subsampling}.\label{Subsampling}

        Diese Operation wird Rangordnungsfilter genannt. Zun"achst wird f"ur jedes Rechteck seine
        Rangordnung bestimmt, d.h.\ die Anzahl der schwarzen Pixel gez"ahlt. Ist diese
        Anzahl gleich oder gr"o"ser einem bestimmten Schwellwert, so wird das korrespondierende Pixel
        gesetzt, ansonsten gel"oscht.

        Das Resultat der Datenreduzierung ist eine
        h"ohere Verarbeitungsgeschwindigkeit (siehe \KapRef{Speed}) nachfolgender
        Algorithmen wie z.B.\ morphologischer Operationen
        (eine mit 300dpi digitalisierte DIN--A4--Seite hat
        ein Datenvolumen von etwa 1MB -- bei einer Reduzierungsrate von 64:1, d.h.\ ein 8x8 Pixel gro"ser
        Block wird auf ein Pixel abgebildet, verbleiben noch 17KB).
        Zus"atzlich entstehen gr"o"sere
        zusammenh"angende Schwarzgebiete "ahnlich denen beim Smearing.

        \Bild{DemoSubsampling}{8}{Unterabgetastetes Bin"arbild (wieder in Originalgr"o"se)}

  \item {\bf Zusammenhangsobjekte verbinden:}\\
        Der erste Schritt bei dieser Methode ist die Transformation des Bin"arbildes in
        Zusammenhangsobjekte. Um eine Segmentierung in hierarchische,
        ineinander aufbauende Bl"ocke zu erhalten, m"u"sen die BCC--Objekte in einem zweiten Schritt
        zu gr"o"seren Gruppen verbunden werden. Die dazu ben"otigten Regeln lassen sich aus Abstandsma"sen,
        Gr"o"senverh"altnissen und deren geometrischen Strukturen herleiten (s.\ \KapRef{Zusammenfassen}).

        \pagebreak[3]
        {\bf Probleme:}
        \begin{itemize}
          \item Ausgehend von nur einzelnen BCC--Objekten ist es schwierig, verl"a"sliche Aussagen
                "uber den Grad ihrer "Ahnlichkeit zu machen, da die Statistik sehr unsicher ist, wenn
                nur wenige Elemente zur Verf"ugung stehen.
          \item Da Dokumente mehrere tausend Zusammenhangsobjekte enthalten k"onnen, ist
                der Vorgang des Verbindens zeitaufwendig.
        \end{itemize}

\end{enumerate}

\subsection{Hybride Vorgehensweisen}

Ferner existieren Methoden, die sich nicht eindeutig in obige zwei Sparten einordnen lassen.
\begin{enumerate}
  \item
        Ein Ansatz hybrider Vorgehensweisen besteht darin, nicht die schwarzen Gebiete des Vordergrundes,
        sondern den wei"sen Hintergrund zu untersuchen. In aufwendigen Layout\-strukturen, bei denen Textspalten sehr nahe beieinander stehen, dienen d"unne schwarze Linien und wei"se Fl"achen als Layoutbegrenzungen.

        \begin{itemize}
          \item Henry Baird schl"agt in \cite{Baird92} vor, die gefundenen wei"sen Rechtecke zu sortieren und
                ihrer Reihenfolge nach bis zu einer Schwelle wieder zu vereinigen (vom Autor {\em
                    `global--to--local'} Strategie genannt), so da"s schrittweise die Untergliederung verfeinert
                wird. Probleme bereitet der rechtzeitige Abbruch der
                Vereinigungsphase sowie das Design der Sortierreihenfolge
                (Regel: `gro"se' sowie `lang und schmale' Rechtecke werden zuerst verbunden).

          \item Masayuki Okamoto \cite{Okamoto93} verbindet Zusammenhangsobjekte, die zwischen Linien
                und wei"sem Zwischenraum liegen, zu Bl"ocken. Hier treten "ahnliche Probleme, wie bei den
                zuvor besprochen Bottom--Up Methoden, auf.
        \end{itemize}

  \item
        J.~Wieser und A.~Pinz schlagen in \cite{Wieser93} eine mehrmalige Abfolge von Smearing und XY--Cuts vor:
        Durch Smearing mit einem relativ gro"sen Smearing--Window sollen viele Objekte auf dem Rasterbild
        zu gro"sen Schwarzbereichen verbunden werden. Diese Schwarzbereiche werden zun"achst durch
        rekursieve XY--Cuts in kleinere St"ucke zerteilt, auf die im Originalbild wiederum Smearing mit
        anschlie"senden XY--Cuts angewandt wird. Die dabei ben"otigten Parameters"atze werden aus diesen
        kleinen Bl"ocken gewonnen.
        Eine Zusammenfassung zerbrochener Bl"ocke bildet den Schlu"s der Segmentierung.

          {\bf Vorteil:}
        \begin{itemize}
          \item Lokale Berechnung der Smearing Parameter direkt aus dem Dokument (nur beim zweiten
                Smearing). Dadurch sind die Verfahren relativ robust gegen"uber verschiedenen Schriftgr"o"sen
                innerhalb eines Dokuments.
        \end{itemize}

        \pagebreak[3]
        {\bf Nachteile:}
        \begin{itemize}
          \item Zweimaliger zeitaufwendiger Smearing-- und XY-Cut--Proze"s.
          \item Es erfolgt keine Konsistenzpr"ufung.
          \item Mittels XY--Cuts k"onnen nur Bl"ocke getrennt werden, bei denen ein horizontaler
                oder vertikaler Durchschu"s (also eine Leerzeile oder --spalte) existiert. Bei "uberlappenden
                Objekten scheitert das Verfahren.
        \end{itemize}
\end{enumerate}

\section{Konzept}

Die in Kapitel \ref{GrundlegendeTechniken} und \ref{LayoutanalyseMethoden}
vorgestellten Methoden bilden die Grundlage f"ur das hier beschriebene Vorgehen.
Die Layoutanalyse ausschlie"slich auf eines dieser Verfahren zu st"utzen, erscheint, aus den oben
genannten Gr"unden, nicht sinnvoll. Zweckm"a"siger ist es, diese Verfahren zu kombinieren, um somit von ihren jeweiligen Vorteilen profitieren zu k"onnen. Keine der in der
Literatur gefundenen Methoden "uberpr"uft die Homogenit"at innerhalb der gefundenen hypothetischen
Textbl"ocke. Es wird stets davon ausgegangen, da"s die vorangegangene Segmentierung
  {\em richtige\/} Ergebnisse liefert. Das liegt unter anderem an der Aufgabenstellung und
den fehlenden M"oglichkeiten zur Berechnung von geeigneten Homogenit"atskriterien.

Grunds"atzlich l"a"st sich das Verfahren in vier Abschnitte gliedern (\BildRef{KonzeptPhasen}):

\begin{enumerate}
  \item Das Bin"arbild und die durch die Zusammenhangsanalyse gewonnenen BCC--Objekte bilden
        die Eingangsdaten. Das Dokument sollte im Winkel korrigiert und Grafik\-objekte
        sollten bereits entfernt worden sein. Ein solches Text--Grafik--Unter\-scheid\-ungs\-modul ist
        in der jetzigen Ausbaustufe des Projekts noch nicht beinhaltet; es mu"s deshalb ein
        (einfacher) Filter implementiert werden. Dieser Filter bewirkt, da"s sehr kleine und sehr
        gro"se Objekte, bei denen davon ausgegangen werden kann, da"s es sich nicht um Text handelt,
        aus der Menge der Zusammenhangsobjekte aussortiert werden. Zus"atzlich
        sollten Linien und Textrahmen, die als Layoutbegrenzer dienen, extrahiert und f"ur die sp"atere
        Segmentierung gespeichert werden.

  \item Der zweite Schritt besch"aftigt sich mit dem Finden von Textbl"ocken und der Berechnung der
        dazu ben"otigten Parameter. Das in Kapitel \ref{BottomUp} Abschnitt \ref{Smearing} vorgestellte Smearing eignet sich
        sehr gut, angrenzende Schwarzgebiete (Buchstaben) zu gr"o"seren Bl"ocken zu verbinden.
        Das Laufzeitproblem, das dieses morphologische Verfahren mit sich bringt, l"a"st sich
        durch eine drastische Reduzierung des Datenvolumens (s.\ \KapRef{BottomUp} Abschnitt \ref{Subsampling}) beseitigen.
        Damit die Smearing Parameter m"oglichst lokal berechnet werden k"onnen, wird das Dokument zuvor
        durch XY--Cuts (s.\ \KapRef{XYCutTheroie}) in seine grobe (rechtwinklige) Struktur zerlegt.
        Innerhalb dieser Gebiete werden die Parameter berechnet und das Smearing ausgef"uhrt.
        Eine Zusammenhangsanalyse auf dem verschmierten Bin"arbild ergibt die Kontur des
        hypothetischen Textblocks. Alle BCC--Objekte innerhalb dieser Kontur
        werden zusammengefa"st und bilden den hypothetischen Textblock.

  \item Im vorangegangenen Schritt wurden die BCC--Objekte aufgrund ihrer geometrischen N"ahe
        zusammengefa"st, ohne dabei die Eigenschaften der Objekte zu betrachten.
        Da man nie auf Anhieb optimale Ergebnisse f"ur alle m"oglichen Dokumentarten erreichen kann,
        mu"s an dieser Stelle eine Konsistenzpr"ufung der hypothetischen Textbl"ocke
        folgen. Entsprechen die Bl"ocke nicht den Anforderungen werden sie
        aufgespalten:
        \begin{itemize}
          \item Hypothetischer Block ist nicht homogen bez"uglich der Buchstabengr"o"se und --st"arke
          \item Textblock wird durch ein Liniensegment durchtrennt
          \item Nebeneinanderliegende Spalten sind verbunden
        \end{itemize}

  \item Eine zweite Fehlerquelle nach dem zweiten Schritt bilden Bl"ocke, die zwar homogen und benachbart
        sind, aber dennoch nicht durch das Smearing
        verbunden wurden. Eine Zusammenfassung dieser Bl"ocke durch Relaxation unter Ber"ucksichtigung
        der Homogenit"atskriterien bildet den Schlu"s der Segmentierung.
\end{enumerate}

\Bild{KonzeptPhasen}{3.4}{Phasen der Textblocksegmentierung}
\clearpage
Anhand von Beispielen ist in Abbildung \ref{eps:AblaufAnBeispielen} dargestellt,
wie die Ergebnisse der einzelnen Schritte aussehen k"onnten.
\Bild{AblaufAnBeispielen}{10}{Schritte der Textblocksegmentierung an Beispielen}
%%%%%%%%%%%%%%%%%%%%%%%%%%%%%%%%%%%%%%%%%%%%%%%%%%%%%%%%%%%%%%%%%%%%%%%%%%%%%%%%%%%%%%%%%%
\section{Einfache Filterung} \label{Filterung}

Bevor die eigentliche Segmentierung durchgef"uhrt werden kann, sollte sichergestellt sein, da"s
sich auf dem Dokument nur noch Textobjekte befinden. Andere Elemente, wie Diagramme,
Bilder und durch schlechte Vorlagenqualit"at bedingte St"orungen sind in der sp"ateren Texterkennung
nicht sinnvoll zu verwerten und sollten deshalb so fr"uh wie m"oglich aus dem Dokument
entfernt werden. Um die BCC--Objekte zu klassifizieren, wird hier ein Gr"o"senfilter
angewandt. Dieser Filter l"a"st alle Objekte passieren, die innerhalb des Bereichs
$[w_{min}, w_{max}]$ f"ur die Breite\footnote{ist von H"ohe und Breite in Bezug auf
  Zusammenhangsobjekte die Rede, so ist damit die H"ohe und Breite des umschreibenden
  Rechtecks ({\em Bounding--Box\/}) gemeint} und $[h_{min}, h_{max}]$ f"ur die H"ohe liegen.
An dieser Stelle mu"s der Bereich sinnvoll festgelegt werden. Es sollten so wenig
Textelemente wie m"oglich entfernt werden, andererseits sollten Schmutzpartikel (Rauschen) und
gro"se Grafikteile eliminiert werden. Experimente mit verschiedenen Parametern haben gezeigt, da"s
Schriftgr"o"sen unter 6pt ({\em Punkt\/} ist die kleinste Ma"seinheit im typographischen System
1m = 2660pt, 1pt = 0.376mm) -- entspricht etwa 16 Pixel bei 300dpi -- selten vorkommen.
Um aber nicht kleine Bestandteile wie Satzzeichen, i--Punkte,
Umlautzeichen ("a, "A, "u, "U, "o, "O) und Akzente (\'{e}, \u{o}, \.o, etc.), bei Schriftgr"o"sen
zwischen 6pt -- 10pt zu l"oschen, wurde die untere Grenze f"ur die minimale H"ohe $h_{min}$ und
Breite $w_{min}$ auf 5 Pixel festgelegt.

Die Wahl der oberen Schranke $h_{max}$ und $w_{max}$ ist dagegen unproblematisch.
Da sehr gro"se "Uberschriften
selten h"oher als 120pt (etwa 350 Pixel bei 300dpi) sind -- Boulevard--Presse ausgeschlossen --
ist es sinnvoll diese Grenze zu w"ahlen.
In Zeitschriften und Tages\-zeit\-ungen werden Rahmen um die
inhaltlich zusammenh"angenden Texte gezogen. Deshalb werden alle umschreibenden
Rechtecke, die die obere Schranke "uberschreiten, als sp"atere Segmentierungshilfe gespeichert.

\Bild{Filter}{13}{Klassifikation und Filterung der BCC--Objekte}

Gleichzeitig erfolgt eine Klassifikation in Liniensegmente, die
analog den Textrahmen f"ur die weitere Bearbeitung als Hilfe dienen. Die Klassifikation erfolgt
nach dem H"ohen-- und Breitenverh"altnis des umschreibenden Rechtecks des BCC--Objekts.
Zus"atzlich wird verlangt, da"s die hypothetischen Linien
frei von wei"sen Einschl"ussen sind. Hierbei ist es nur m"oglich, horizontale und vertikale Linien
zuverl"assig zu erkennen. Zerfallene oder punktierte Linien, sowie Linien, deren Winkel nicht um
$0^\circ$ bzw.\ $90^\circ$ liegen, werden nicht erkannt und wirken beim folgenden
Smearing st"orend.

Die als zu klein, zu gro"s oder als Linien eingestuften Objekte werden aus der Menge der
BCC--Objekte und innerhalb des Bin"arbildes gel"oscht und nicht weiter betrachtet.
Zusammenfassend ist die Filterung in \BildRef{Filter} dargestellt.

Diese Klassifikation ist sehr einfach gehalten und sollte zu einem sp"ateren Zeitpunkt durch ein
Text--Grafik--Separationsmodul ersetzt werden.

  {\bf Parametrierung:}

Aufgrund der h"oheren Anschaulichkeit sind die Gr"o"senangaben bei der Parametereinstellung
nicht in Punkten,
sondern in Pixeln anzugeben (s.\ \BildRef{ParameterPanelFilter}).
Die Standardparameter bei 300dpi sind 5 Pixel
f"ur kleine und etwa 350 Pixel f"ur gro"se Objekte (jeweils f"ur H"ohe und Breite).
Bei der Berechnung dieser Werte wird von einer minimalen Buchstabenh"ohe (Fontsize) von 6pt
ausgegangen. Die Umrechnung von Punkten in Pixel erfolgt "uber die Sch"atzung der
Scannaufl"osung, wobei angenommen wird, da"s das Dokument im DIN--A4--Hochformat vorliegt
und entsprechend viele Pixelzeilen besitzt. Bei Dokumenten, die verkleinert oder
quer eingelesen wurden, mu"s die Umrechnung von Punkt in Pixel angepa"st werden.

%$$S[\mbox{dpi}] = \frac{S \mbox{Pixel}}{1 \mbox{in}} = \frac{S \mbox{Pixel}}{72.27 \mbox{pt}}$$
%$$1\mbox{pt} = \frac{S}{72.27}\mbox{pt}$$

\Bild{ParameterPanelFilter}{11}{Parameter f"ur die Filterung}

%%%%%%%%%%%%%%%%%%%%%%%%%%%%%%%%%%%%%%%%%%%%%%%%%%%%%%%%%%%%%%%%%%%%%%%%%%%%%%%%%%%%%%%%%%
\section{Textbl"ocke erzeugen}
Der zweite Analyseschritt soll erreichen, da"s m"oglichst viele Objekte mit der
Smearing--Methode zusammengefa"st werden:
\begin{itemize}
  \item Um die Smearing--Parameter lokal zu berechnen, wird das Dokument mittels XY--Cuts in
        seine grobe Struktur zerlegt. Die gefundenen rechtwinkligen Gebiete seien $T_{xy}$.

  \item Die hohe Datenmenge des Bin"arbildes wird durch Unterabtastung um den Faktor 64
        reduziert. Dadurch wird die Laufzeit nachfolgender morphologischer Operationen
        erheblich verkleinert.

  \item Innerhalb von $T_{xy}$ wird nun jeweils das Smearing--Window "uber den
        Zeilenabstand direkt aus den BCC--Objekten berechnet und das Smearing mittels
        der Closing Operation durchgef"uhrt.

  \item Eine BCC--Analyse auf dem {\em verschmierten\/} Bin"arbild ergibt die
        Konturen der Textblockhypothesen. Zusammenfassung der urspr"unglichen BCC--Objekte innerhalb der
        gefundenen Konturen zu Textbl"ocken.
\end{itemize}
\subsection{XY--Cut}\label{XYCut}

In diesem Schritt wird versucht, ausgehend von den Zusammenhangsobjekten, das Layout in
seine grobe Struktur zu zerlegen. Die Zerlegung erfolgt alternierend mit Schnitten parallel
zur X-- und Y--Achse. Um die Schnittpositionen parallel zur Y--Achse zu bestimmen, werden alle umschreibenden Rechtecke der BCC--Objekte auf die Horizontale projiziert (analog X--Achse).
Es entsteht ein sogenanntes Projektionsprofil oder Projektionshistogramm
(s.\ \BildRef{Projektionsprofil}).

\Bild{ProjektionT}{8}{Projektion einer BCC--Box}

In das Histogramm wird nicht jeder einzelne schwarze Pixel aufgenommen, sondern nur die
obere und untere Kante (bzw.\ linke und rechte) der umschreibenden Rechtecke und
da\-zwischen\-liegende Werte im Abstand von 10 Pixel.
Beispielsweise ist das BCC--Objekt in \BildRef{ProjektionT} etwa 40 Pixel hoch,
was f"unf Eintr"agen im Projektionshistogramm entspricht. W"urde man dagegen jeden
schwarzen Pixel projizieren, k"ame es zu etwa 200 bis 400 Eintr"agen.

\Bild{Projektionsprofil}{13}{Vertikales Projektionsprofil}

In der folgenden Histogrammuntersuchung werden Bereiche $[x_{min}, x_{max}]$ bzw.\
$[y_{min}, y_{max}]$ mit einer absoluten H"aufigkeit von Null bestimmt, d.h.\ innerhalb dieser
Spalten bzw.\ Zeilen befindet sich kein BCC--Objekt. Die Schwelle
$s_j$, bei der die Gebiete getrennt werden, wird mit jedem Wechsel der
Schnittrichtung (Level) erniedrigt (siehe Parametrierung).

Dadurch werden immer kleinere Bereiche
als Schnittposition ausgew"ahlt und die Segmentierung wird immer feiner. Da im vorliegenden Fall
nur eine sehr grobe Einteilung gefordert ist, reichen drei bis vier Schritte aus.

Der XY--Cut Proze"s wird horizontal begonnen. Existieren hier
keine m"oglichen Schnittpositionen, so wird die vertikale Projektion ermittelt und fortgefahren.
Die Ergebnisse werden in einer hierarchischen Baumstruktur (XY--Tree) abgelegt. Die Bl"atter
des Baumes enthalten die Gebiete (repr"asentiert als Rechtecke)
die f"ur die weiteren Schritte ben"otigt werden
(Gebiete 0.1, 0.2.2, 0.2.1.1 und 0.2.1.2 aus \BildRef{XYTree}).

\Bild{XYTree}{7}{XY--Tree vom Dokument aus \protect\BildRef{XYCutIdee}}

\pagebreak[4]
{\bf Parametrierung:}

Es gilt nun, die Schwelle $s_j$ unabh"angig vom Dokumenttyp zu parametrieren.
Dazu wird die mittlere H"ohe $\bar{h}$ der BCC-Bounding Boxes im betrachteten Block
berechnet und mit $c_{XYj}$ gewichtet (s. Parametereingabe \BildRef{ParameterPanelTopDown}).
So ergibt sich $s_j = c_{XY_j}\bar{h}$. Der Wert $\bar{h}$ wird bei jedem Wechsel der
Projektionsrichtung neu berechnet
und somit an die neuen Gr"o"senverh"altnisse innerhalb der getrennten Gebiete angepa"st.

Ein vertikaler Schnitt erfolgt, wenn gilt (horizontal analog):

$$x_{max} - x_{min} > s_j \qquad\mbox{mit}\qquad s_j = c_{XY_j}\bar{h}
  \qquad\mbox{und}\qquad j=0,1,2\ldots 8$$

Die mittlere Box--H"ohe $\bar{h}$ berechnet sich aus dem arithmetischen Mittel der H"ohen der
umschreibenden Rechtecke der BCC--Objekte. Objekte mit einer H"ohe unter 6pt werden nicht mit in die
Berechnung eingeschlossen, da es sich dabei meist um Satzzeichen o."a.\ handelt, die die
Statistik verf"alschen w"urden.

\begin{eqnarray}\label{MeanHeight}
  \bar{h} = \frac{1}{n} \sum_{j=1}^n h_j \qquad h_j > 6\mbox{pt} \;\forall\; j
\end{eqnarray}

\Bild{ParameterPanelTopDown}{14}{Parameter f"ur das XY--Cut Verfahren}

\subsection{Subsampling}
Das von Schmutzteilchen und Nicht--Textobjekten gereinigte Bin"arbild wird nach folgendem
Schema unterabgetastet (Rangordnungsfilter):
Das Bin"arbild wird in nicht"uberlappende Rechtecke von 8x8 Pixel eingeteilt.
Jeder dieser Bl"ocke wird nun auf einen Pixel abgebildet. Dieses Vorgehen entspricht einer
Reduzierung des Datenvolumens um den Faktor 64. Gez"ahlt werden alle schwarzen Pixel
innerhalb jedes einzelnen Blocks. Ist die Anzahl gleich oder gr"o"ser einem Schwellwert,
so wird der Pixel im reduzierten Bild schwarz, ansonsten wei"s.
Dieser Schwellwert sollte relativ klein gew"ahlt werden, damit alle schwarzen Pixel
erhalten bleiben. Standardeinstellung f"ur die Schwelle ist eins (siehe \BildRef{Subsampling}
und Parametrierung \BildRef{ParameterPanelBottomUp} Seite \pageref{eps:ParameterPanelBottomUp}).

\Bild{Subsampling}{6}{Reduktionsschema: Jeder 8x8 Block wird auf ein Pixel reduziert}

Bei dieser Art der Datenreduktion gehen keine f"ur die folgenden Schritte ben"otigten
Informationen verloren. Eine OCR--Analyse w"are aufgrund der fehlenden Detailinformation,
aber mit reduziertem Bin"arbild nicht mehr m"oglich (entspr"ache einer Scanneraufl"osung
von 40dpi).

\subsection{Smearing}
Den Ausgangspunkt f"ur das Smearing--Verfahren bilden die beim XY--Cut Verfahren gefundenen Gebiete
$T_{XY}$ sowie das unterabgetastete Bin"arbild.
Innerhalb der Gebiete $T_{XY}$ werden nun die Smearingparameter
berechnet. Hier mu"s ein Kompromi"s gefunden werden: Wird die Maskenbreite oder --h"ohe zu gro"s
gew"ahlt, k"onnen Fehler entstehen, die durch nachfolgende Schritte nicht, oder nur durch
aufwendige Verfahren, korrigierbar sind. Andererseits bringt das Smearing bei zu kleiner
Paramerterwahl auch keine zufriedenstellenden Ergebnisse -- hier ist ebenfalls eine aufwendige
Nachbearbeitung erforderlich. Des weiteren ist es nicht sinnvoll, die Parameter statisch,
d.h.\ nur auf eine Dokumentklasse optimiert, festzulegen.
Da das System die ben"otigten Parameter selbst"andig an das Layout anpassen mu"s,
wird zun"achst der Zeilenabstand gesch"atzt (s.\ \BildRef{SmearingOptimal}).
Dieser ist definiert als die Summe aus Schriftgr"o"se und Durchschu"sh"ohe. Die Durchschu"sh"ohe
ist dabei der Abstand der Unterkante eines kleinen Buchstabens mit
Unterl"ange (z.B.\ "`g"') bis zur Oberkante eines Gro"sbuchstabens.

\Bild{SmearingOptimal}{10}{Optimale Smearing--H"ohe}

\subsubsection{Zeilenabstandsch"atzung}\label{Zeilenabstand}

Die Sch"atzung des Zeilenabstandes erfolgt nach folgendem Prinzip: Die Mittelpunkte der
Zusammenhangsobjekte werden auf die vertikale Achse projiziert.
Dadurch erh"alt man ein Profil, bei dem die Maxima
die Textzeilen und die Minima die Zeilenzwischenr"aume repr"asentieren. Dieses Verfahren wird
daher {\em Profiling\/} genannt (siehe auch \cite{Bohnacker93}).
Aus dem Profil l"a"st sich nun der h"aufigste Abstand der Maxima
bestimmen. Dieser Wert ist eine gute Sch"atzung f"ur den Zeilenabstand.

Verfahrensschritte:
\begin{itemize}
  \item{\bf Kleine BCC--Objekte herausfiltern:}\\
  Da die Positionen von kleinen Objekten innerhalb einer Zeile stark variieren, m"ussen
  sie vor der Projektion herausgefiltert werden. Beispielweise befinden sich Satzzeichen am
  unteren Zeilenrand, dagegen Anf"uhrungs- und Umlautzeichen am oberen. Die Filterschwelle
  f"ur kleine Zeichen wird in Abh"angigkeit der mittleren Boxh"ohe eingestellt.

  \item {\bf Zeilenprofil erstellen:}\\
        Projiziert man die Mittelpunkte der gefilterten Objekte auf die vertikale Achse, entsteht
        in einem Histogramm ein Zeilenprofil (s.\ \BildRef{ZeilenProfil}).
        Um die Histogramm\-auswertung zu beschleunigen, wird die Klassenanzahl reduziert. Hierbei werden
        $\bar{h}$ (\ref{MeanHeight}) Klassen (Zeilen) zu f"unf Histogrammspalten zusammengefa"st.
        Die weitere Konzentration der Maxima wird erreicht, indem das Histogramm zus"atzlich
        mit dem Gravitationsfilter bearbeitet wird.

        \item{\bf Zeilenabstand sch"atzen:}\\
        Die Abst"ande der Maxima werden wiederum in einem Histogramm aufbereitet.
        Der Zeilenabstand $\bar{Z}$ ist nun die Klasse mit den meisten Eintragungen (\BildRef{ZeilenHistogramm}).
        Weniger wahrscheinliche Werte, die z.B.\
        kleiner als die Zeichenh"ohe sind, werden verworfen, und das Maximum mit der
        n"achst kleineren H"aufigkeit untersucht.
\end{itemize}

\Bild{ZeilenProfil}{15}{Zeilenprofil -- vertikale Projektion der Objektmittelpunkte
  vor und nach der Gravitationsfilterung}

\Bild{ZeilenHistogramm}{5}{Histogramm "uber die Abst"ande der Maxima im Zeilenprofil}

\subsubsection{Smearing--Window}

Mit Hilfe des Zeilenabstandes $\bar{Z}$ kann nun die H"ohe der Smearing Maske $S_h$
bestimmt werden (s.\ \BildRef{Zeilenabstand}), wobei Untersuchungen gezeigt haben, da"s
$S_h$ anstelle $S_h = \bar{Z} - \bar{h}$ etwas gr"o"ser gew"ahlt kann
($\bar{h}$ entspricht der mittleren Zeichenh"ohe siehe Gleichung (\ref{MeanHeight})):
$$S_h = \bar{Z} - \frac{\bar{h}}{2}$$

\Bild{Zeilenabstand}{8}{Zusammenhang zwischen Zeilenabstand und Smearing--H"ohe}

Die dazugeh"orige Breite $S_w$ sollte mit Bedacht gew"ahlt werden. Werden "uber mehrere
Spalten hinweg Objekte verbunden, m"ussen diese -- wenn im nachhinein noch m"oglich -- in nachfolgenden
Korrekturschritten wieder getrennt werden. Andererseits sollen die vorhandenen
Textbl"ocke nicht in viele Einzelteile zerfallen.
Untersuchungen haben ergeben, da"s die mittlere Objekth"ohe $\bar{h}$ multipliziert mit
dem konstanten Wert $1.4$ gute Ergebnisse liefert. Der Wert $c_w$ ist "uber die
Parametereingabe "anderbar (s.\ Parametrierung \BildRef{ParameterPanelBottomUp} Seite
\pageref{eps:ParameterPanelBottomUp}).

$$S_w = c_w  \bar{h}\qquad c_w = 1.4$$

\subsubsection{Konkrete Ausf"uhrung}

Die Closing--Operation kann sehr effizient ausgef"uhrt werden. Bei der Dilatation wird eine
Kopie des Bin"arbildes um einen Pixel versetzt erstellt und mit dem Original durch ein
logisches ODER verkn"upft.
Da nach dem ersten Schritt keine alleinstehenden Schwarzpixel mehr vorkommen, wird das
Bin"arbild beim zweiten Schritt um zwei Pixel versetzt, beim dritten um vier usw. Die
anschlie"sende Erosion erfolgt analog, mit dem Unterschied, da"s die Bin"arbilder durch ein
logisches UND zu verkn"upfen sind.

\subsection{Zuordnen der BCC--Objekte zu Textbl"ocken}
Nach dem Smearing wird auf dem daraus resultierenden Bin"arbild eine Zusammenhangsanalyse
vorgenommen. Es ergeben sich nach der R"ucktransformation von der reduzierten Aufl"osung
auf die Originalaufl"osung die "au"seren Konturen der hypothetischen Textbl"ocke
(s.\ \BildRef{SmearingKontur}).

\Bild{SmearingKontur}{8}{Eine BCC--Analyse auf dem {\em verschmierten\/} Bin"arbild ergeben die Konturen der Textbl"ocke}

Nun gilt es, die BCC--Objekte des Originalbildes, die sich innerhalb derselben Blockkontur befinden, zu einem Textblock zu vereinigen.
"Uberlappen sich die umschreibenden Rechtecke der Blockkontur nicht
(s.\ \BildRef{SmearingKonturZuordnung} TB3) w"urde es gen"ugen, die Objekte zu einem Block
zusammenzufassen, die sich innerhalb des
Rechtecks befinden. Im Normalfall mu"s man aber davon ausgehen, da"s sich die umschreibenden
Rechtecke der Blockkonturen "uberlappen (s.\ \BildRef{SmearingKonturZuordnung} TB1 und TB2).
Daher m"ussen diejenigen Objekte aussortiert werden,
die sich zwar innerhalb der Bounding--Box, aber au"serhalb der betrachteten Blockkontur (TB1) befinden.

\Bild{SmearingKonturZuordnung}{10}{"Uberlappende Boxes der Blockkonturen}

Die Blockkontur hat die Eigenschaft, da"s sie die BCC--Objekte immer vollst"andig enth"alt.
Deshalb gen"ugt es, zu untersuchen, ob sich ein beliebiger Punkt aus der BCC--Kontur des
einzugliedernden Objekts innerhalb einer Blockkontur befindet.

Hierzu bietet sich folgende L"osung an:
Man zeichne eine von dem zu untersuchenden Punkt beginnende Halbgerade
(die so lange ist, da"s ihr anderer Endpunkt garantiert au"serhalb des Polygons der Blockkontur
liegt) und z"ahle die Polygonabschnitte, die sie schneidet.
Ist die Anzahl ungerade, so liegt der Punkt innerhalb des Polygons; ist sie gerade,
liegt er au"serhalb (\BildRef{PunktInPolygon}).

\Bild{PunktInPolygon}{12}{Punkt im einem Polygon}

Da Konturpolygone aus Bin"arbildern erzeugt werden, bestehen sie nur aus
waagerechten und senkrechten Linienelementen, so da"s es sich empfiehlt, die Halbgerade ebenso
waagerecht oder senkrecht zu w"ahlen. Dadurch vereinfacht sich die Untersuchung, ob und wie
oft sich die Gerade mit dem Polygon schneidet.

  {\bf Parametrierung:}

\Bild{ParameterPanelBottomUp}{14}{Parameter f"ur die Unterabtastung und das Smearing}

%%%%%%%%%%%%%%%%%%%%%%%%%%%%%%%%%%%%%%%%%%%%%%%%%%%%%%%%%%%%%%%%%%%%%%%%%%%%%%%%%%%
\clearpage
\section{Aufspalten von Textblockhypothesen}

Prinzipiell verbindet das Smearing s"amtliche Gebiete, die in r"aumlicher N"ahe zueinander
stehen, ohne auf deren Gestalt R"ucksicht zu nehmen.
Folgende Fehler in den Textblockhypothesen m"ussen deshalb korrigiert werden:
\begin{itemize}
  \item Die BCC--Objekte innerhalb eines Blocks sind nicht homogen
        (s.\ \KapRef{Konsistenzpruefung}).
  \item Ein Textblock wird durch ein Liniensegment durchtrennt (s.\ \KapRef{LinienSchnitte}).
  \item Nebeneinanderliegende Spalten werden verbunden (s.\ \KapRef{HorKleb}).
\end{itemize}

\subsection{Konsistenzpr"ufung}\label{Konsistenzpruefung}

Da als grundlegende Forderung an die Segmentierung die Homogenit"at der
Textbl"ocke gestellt wurde, mu"s dem Smearing eine Konsistenzpr"ufung folgen. Versuche mit
verschiedenen Dokumentklassen haben gezeigt, da"s h"aufig Textzeilen verschiedener
Schriftgr"o"sen oder Schriftst"arken verbunden werden (s.\ \BildRef{SmearingFehler}). Deshalb
werden die Eigenschaften jeder Textzeile bestimmt und mit der ihr folgenden verglichen.
Liegen die Differenzen "uber einer bestimmten Schwelle, wird der Block zwischen diesen zwei
Zeilen durchtrennt. Hierbei wird angenommen, da"s die Objekte innerhalb einer Zeile
konsistent sind.

\Bild{SmearingFehler}{8}{Fehlerhafte Verschmelzungen durch das Smearing (links Schriftgr"o"se, rechts Schriftst"arke)}

Die Homogenit"at bezieht sich auf die Eigenschaften, die sich
aus den Zusammenhangselementen ableiten lassen. Es wird die mittlere H"ohe der
umschreibenden Rechtecke der BCC--Objekte (R"uckschl"usse "uber Homogenit"at der Buchstabengr"o"se),
und die Strichst"arke (R"uckschl"usse "uber Homogenit"at der Gestalt) betrachtet. Weitere Kriterien
wie Schriftauspr"agungen (serifenlos, proportional, kursiv) oder Schriftklassen
(Antiqua, Schreibschriften, Frakturen) sind vorstellbar, wurden jedoch nicht untersucht.

\subsubsection{Homogenit"atskriterien}\label{Homogenitaet}
\begin{enumerate}

  \item{\bf Mittlere Objekth"ohe:}\\
  Die mittlere Boxh"ohe $\bar{h}$ wurde bereits in den vorherigen Abschnitten eingesetzt
  um die Parameter
  unabh"angig von der Layoutstruktur einzustellen. Sie berechnet sich aus dem arithmetischen
  Mittel der H"ohen der BCC--Objekte:
  %
  $$\bar{h} = \frac{1}{n} \sum_{j=1}^n h_j \qquad h_j > 6\mbox{pt} \;\forall\; j.$$

  \item{\bf Strichdicke:}\\
  Die Strichdicke oder Strichst"arke eines Zeichens ist definiert als das Verh"altnis der
  Strichbreite $d$ zur H"ohe $h$ des Zeichens. Im allgemeinen wird die Strichdicke auf die H"ohe
  der Gro"sbuchstaben bezogen und in Prozent angegeben, so da"s der Wert unabh"angig von der
  individuellen Schrifth"ohe ist.

  Die Strichdicke eines Zeichens l"a"st sich mit folgendem Ansatz bestimmen:
  Ein Strich wird durch ein Rechteck der L"ange $l$ und Dicke $d$ modelliert,
  welches denselben Umfang $U$ und Fl"acheninhalt $F$ wie das Bin"arbild des Zeichens besitzen soll
  (s.\ \cite{Kuropka90}).
  %
  \begin{eqnarray}
    U_{Strich} &=& 2 * ( l + d)\label{UStrich}\\
    F_{Strich} &=& l * d\label{FStrich}
  \end{eqnarray}
  %
  Aus (\ref{UStrich}) und (\ref{FStrich}) ergibt sich die absolute Strichdicke $d$:
  %
  \begin{eqnarray}
    d^2 - \frac{U}{2}d + F = 0\nonumber\\
    d = \frac{U}{4} * \left( 1 - \sqrt{ 1-16*\frac{F}{U^2} } \;\right)\label{Strich}
  \end{eqnarray}
  %
  Da verlangt wird, da"s $U_{Strich} = U_{Zeichen} = U$ und $F_{Strich} = F_{Zeichen} = F$, mu"s
  die Fl"ache $F_{Zeichen}$ und der Umfang $U_{Zeichen}$ jedes Zeichens berechnet werden.
  Da etwa 90\% der Zusammenhangsobjekte einzelne Buchstaben darstellen, k"onnen
  Umfang und Fl"ache eines Buchstabens direkt aus den BCC--Objekten berechnet werden.
  Da Buchstaben nur wei"se Gebiete enthalten, jedoch keine weiteren schwarzen Gebiete,
  gen"ugt es nur die S"ohne des BCC--Objektes zu betrachten.

  Der Umfang eines Zeichens ist die Summe aus dem "au"seren schwarzen Umfang und dem Umfang
  der inneren wei"sen Einschl"usse:
  %
  $$U_{Zeichen} = U_{BCC} + \sum_{\mbox{\footnotesize S"ohne}} U_{BCC}$$
  %
  Der Umfang eines BCC--Objektes ist definiert als die L"ange der durch die Rastergrenzlinie beschriebenen Kontur und ergibt sich als Summe der L"angen der Teilkonturen
  (mit $n$ als die Anzahl der Eckpunkte und $x$/$y$ als die St"utzpunkte der BCC--Kontur):
  %
  \begin{eqnarray*}
    U_{BCC} &=& \sum^{n/2}_{i=1} \left|y_{2i} - y_{2i+1} \right| + \left|x_{2i} - x_{2i-1} \right|\\[2mm]
    y_{n+1} &=& y_1
  \end{eqnarray*}
  %
  Analog wird der Fl"acheninhalt berechnet: Fl"ache, die die "au"sere Kontur einschlie"st, abz"uglich
  der inneren wei"sen Fl"achen:
  %
  $$F_{Zeichen} = F_{BCC} - \sum_{\mbox{\footnotesize S"ohne}} F_{BCC}$$
  %
  Die Fl"ache eines BCC--Objektes, beschrieben durch die Kontur, ergibt sich nach l"angerer
  Rechnung wie folgt \cite{Bartneck87}:
  %
  $$F_{BCC} = \sum^{n/2}_{i=1} \left(x_{2i} - x_{2i-1} \right) y_{2i}$$
  %
  Die mittlere relative Strichdicke $\bar{d_r}$ ergibt sich aus dem arithmetischen Mittel aller
  Strichdicken $d$ innerhalb eines Textblocks, normiert auf die mittlere Buchstabenh"ohe~$\bar{h}$.
  Hierbei werden analog zur Berechnung von $\bar{h}$ Gleichung (\ref{MeanHeight}) nur BCC--Objekte
  betrachtet, deren H"ohen gr"o"ser als 6pt sind:
  %
  \begin{eqnarray}
    \bar{d_r} = \frac{1}{n \bar{h}} \sum_{j} d_j
  \end{eqnarray}
  %
  Abbildung \ref{eps:StrichdickeFonthoehe} zeigt die Abh"angigkeit der Strichdicke $\bar{d_r}$
  von der Buchstabengr"o"se. Da\-raus ist ersichtlich, da"s Objekte gleicher Buchstabengr"o"se, aber
  unterschiedlicher Strichdicken, sicher unterschieden werden k"onnen.
  %
  \Bild{StrichdickeFonthoehe}{8}{Abh"angigkeit der Strichdicke von der Zeichenh"ohe}
\end{enumerate}

\subsubsection{Homogenit"at zwischen Textzeilen}
Der Ablauf ist wie folgt gegliedert:
\begin{enumerate}
  \item {\bf Zeilen erzeugen:}\\
        Mit dem Profiling--Verfahren wird analog der Berechnung des Zeilenabstandes ein Zeilenprofil
        erstellt. Eine vorherige Filterung der kleinen Segmente findet hier nicht statt, da
        diese sonst nachtr"aglich den Zeilen zugeordnet werden m"u"sten.
  \item {\bf Nachfolgende Zeilen vergleichen:}\\
        Es werden die mittlere Boxh"ohe und die Strichdicke f"ur jede Zeile berechnet und mit
        aufeinanderfolgenden Zeilen verglichen
        ($\bar{h_1}$, $\bar{h_2}$ seien die mittleren H"ohen und $\bar{d_{r1}}$, $\bar{d_{r2}}$ die
        mittleren relativen Strichst"arken der zwei zu untersuchenden Zeilen). Die
        maximale relative Differenz $c_{err}$ l"a"st sich als Parameter einstellen
        (s.\ \BildRef{ParameterPanelSplit} Seite \pageref{eps:ParameterPanelSplit}).
        Eine Trennung dieser Zeilen erfolgt, wenn

        $$\frac{|\bar{h_1}-\bar{h_2|}}{\min (\bar{h_1},\bar{h_2})} > c_{err}
          \qquad\mbox{oder}\qquad
          \frac{|\bar{d_{r1}}-\bar{d_{r2}|}}{\min (\bar{d_{r1}},\bar{d_{r2}})} \;>\; c_{err}.$$

        Dabei galt es folgende Probleme zu l"osen:
        \begin{itemize}
          \item Besteht eine der zwei Zeilen aus nur wenigen Objekten
                (die Objektanzahlen in den Zeilen seien $n_1$ und $n_2$), dann ist die berechnete
                Statistik "uber $\bar{h}$ und $\bar{d_r}$ sehr unsicher. Um zu vermeiden, da"s kurze Zeilen
                f"alschlicherweise von einem Block getrennt werden, wird daher $c_{err}$ mit $g(n_1,n_2)$
                gewichtet (\BildRef{ErrWeight}).
                Diese Funktion wurde heuristisch aus Experimenten gewonnen. Befinden sich
                in einer Zeile weniger als 20 BCC--Elemente, darf sich der Fehler bis um den Faktor zwei
                erh"ohen, ohne da"s es zu einer Trennung der Zeilen kommt.

                \Bild{ErrWeight}{6.5}{Gewichtungsfunktion}

          \item Da bei der Erzeugung des Zeilenprofils die kleinen Segmente nicht entfernt werden, kann
                es vorkommen, da"s Zeilen entstehen, die ausschlie"slich Satzzeichen
                oder Umlautzeichen enthalten
                (s.\ \BildRef{ProfilingFehler}). Um eine Trennung, die zwangsl"aufig geschehen w"urde, zu
                vermeiden, m"ussen diese Zeilen zuverl"assig erkannt werden. Zuerst werden die
                umschreibenden Rechtecke der zu untersuchenden Zeilen berechnet. "Uberlappen sich diese
                Rechtecke in der Vertikalen und ist die Buchstabengr"o"se einer Zeile sehr viel kleiner
                ($c_{err} > 2.5$) als die der anderen, so handelt es sich h"ochstwahrscheinlich um einen
                Fehler im Zeilenprofil -- folglich d"urfen diese zwei Zeilen nicht getrennt werden.

                \Bild{ProfilingFehler}{8}{Fehlerhaftes Zeilenprofil}

          \item Ein typisches Gestaltungsmittel f"ur Artikel in Zeitungen und Zeitschriften
                besteht darin, da"s der erste Buchstabe in der ersten Zeile
                "uberdimensional gro"s -- zumeist so hoch wie zwei bis drei Zeilen -- gedruckt wird.
                Diese Buchstaben dienen als Blickfang, man spricht hier von einem {\em Initial\/}
                (s.\ \BildRef{Initial}).

                \Bild{Initial}{8}{Beispiele f"ur Initiale}

                Diese Initiale, die eine Nichthomogenit"at {\em innerhalb\/} einer Zeile darstellen,
                k"onnen relativ einfach erkannt werden (Widerspruch der Hypothese, da"s die Elemente in
                einer Zeile konsistent sind):
                Dazu werden die ersten vier Zeilen am linken Rand auf ein extrem gro"ses BCC--Objekt hin
                untersucht. Betr"agt die H"ohe mindestens das Zweifache der mittleren Buchstabenh"ohe $\bar{h}$,
                wird das BCC--Objekt als Initial klassifiziert, von der Zeile getrennt und in einem
                separaten Textblock angelegt (Ergebnis siehe \BildRef{HomoCheckErg} Seite
                \pageref{eps:HomoCheckErg}).
        \end{itemize}
\end{enumerate}

\subsection{Schnitt mit Liniensegmenten}\label{LinienSchnitte}

In einem weiteren Korrekturschritt werden Liniensegmente und Textrahmen
(Extraktion siehe Filterung \KapRef{Filterung}) bei der
Textblocksegmentierung einbezogen. Sie werden h"aufig im Layout von Tageszeitungen eingesetzt
und dienen dazu inhaltlich unabh"angige Textbereiche voneinander zu trennen.

\Bild{LineIntersec}{11}{Liniensegmente dienen als Trenner, wenn sie in Textbl"ocke hineinragen}

F"ur jedes Liniensegment wird berechnet um wieviel es in die umschreibenden Rechtecke der
Textblockhypothesen hineinragt (s.\ \BildRef{LineIntersec}).
Dieser Wert, $l$ genannt, wird auf die Textblockbreite $w_{TB}$
bei horizontalen Linien bzw.\ auf die Textblockh"ohe $h_{TB}$ bei vertikalen Linien normiert.
Es ergibt sich

$$l_r = \left\{
  \begin{array}{l@{\qquad}l}
    \frac{l}{w_{TB}} * 100\% & \mbox{bei horizontalen Linien} \\
    \frac{l}{h_{TB}} * 100\% & \mbox{bei vertikalen Linien}
  \end{array}\right.$$

"Ubersteigt $l_r$ eine als Parameter einstellbare Schwelle $l_{rCUT}$ (s.\ Parametrierung
\BildRef{ParameterPanelSplit}), so wird der Textblock an der Linienschnittstelle aufgespalten.

\subsection{Horizontale Verbindungen zwischen Spalten}\label{HorKleb}

Durch die statische Festlegung der Smearing--Fensterbreite k"onnen sich Spalten horizontal verbinden. Dies ist zum Beispiel der Fall, wenn Liniensegmente aufgrund zu
kleiner Scanner--Aufl"osung in viele kleine St"ucke zerfallen. Diese St"ucke sind dann nicht mehr als
Linie zu erkennen, aber oftmals wiederum zu gro"s, um bei der Filterung in Kapitel
\ref{Filterung} gel"oscht zu werden. Sie dienen beim Smearing als {\em Br"ucke\/} zwischen
Textspalten (s.\ \BildRef{HorizKleb}). Im Gegensatz zum vorigen Abschnitt \ref{LinienSchnitte}
stehen hier keine Liniensegmente als Layoutbegrenzung zur Verf"ugung.

\Bild{HorizKleb}{10}{Linienreste zwischen Textspalten f"uhren zur Verbindung von Spalten}

Ein einzelner sekrechter XY--Cut, angewandt auf jede Textblockhypothese,
versucht Verklebungen zwischen Spalten festzustellen:
\begin{enumerate}
  \item Um gr"o"sere St"orungen und zerfallene Linienst"ucke zu entfernen, wird die Menge der
        BCC--Objekte erneut gefiltert. Hierbei wird die Filterschwelle $h'_{min}$ und $w'_{min}$
        kleiner als in Kapitel \ref{Filterung} angesetzt:

        $$h'_{min} = \frac{3}{5}\bar{h} \qquad w'_{min} = \frac{1}{2}\bar{h}$$

  \item Die anschlie"senden Verfahren entsprechen einem vertikalen XY--Cut (s.\ \KapRef{XYCut}):
        Die BCC--Objekte werden auf die horizontale Archse projiziert und das sich ergebende
        Projektionshistogramm wird analog den XY--Cuts ausgewertet, d.h.\ es wird ein einzelner
        vertikaler XY--Cut durchgef"uhrt. Als Parameter l"a"st sich der XY--Cut Startlevel einstellen
        (s.\ \BildRef{ParameterPanelSplit}).

  \item Konnte das XY--Cut Verfahren ein oder mehrere vertikale Leerr"aume finden, so wird
        angenommen, da"s Spalten durch das Smearing horizontal verbunden wurden.
        Die Textblockhypothese wird an diesen Stellen aufgespalten und die
        beiseitegelegten `kleinen' BCC--Objekte in die entsprechenden neuen Bl"ocke wieder eingeordnet.
\end{enumerate}

{\bf Parametrierung}

\Bild{ParameterPanelSplit}{16}{Parameter f"ur das Aufspaltung von Textbl"ocken}

%%%%%%%%%%%%%%%%%%%%%%%%%%%%%%%%%%%%%%%%%%%%%%%%%%%%%%%%%%%%%%%%%%%%%%%%%%%%%%%%%%
\clearpage
\section{Zusammenfassen von Textblockhypothesen}\label{Zusammenfassen}

Bisherige Schritte verhindern nicht, da"s homogene Textbl"ocke dennoch getrennt werden, u.a.\
bei nichtproportionaler Schrift (i.d.R.\ mit der Schreibmaschine) und zus"atzlichem Blocksatz (s.\
\BildRef{FixedSpacedDemo}).

\Bild{FixedSpacedDemo}{10}{Nichtproportionaler Text mit zus"atzlichem Blocksatz f"uhrt zu ungewollten Trennungen}

Dieses Problem k"onnte durch eine Erh"ohung der Smearing--Window--Breite vermieden werden,
jedoch f"uhrt dies zu Problemen bei mehrspaltigem Layout.
Deshalb werden in diesem Bearbeitungsschritt aufgrund von Regeln, die die geometrische Ausrichtung und die Homogenit"at ber"ucksichtigen, Bl"ocke weiter verbunden:
\begin{itemize}
  \item Sortierung der Textbl"ocke nach ihrer r"aumlichen N"ahe durch die Aufstellung eines
        minimalen Spannbaumes.
  \item Aufstellen von Kriterien und Regeln bez"uglich der Vereinigung.
  \item Verbinden der Textbl"ocke durch Relaxation.
\end{itemize}

\subsection{Minimum Spanning Tree}

Fa"st man Textblockhypothesen zusammen, ist es nicht sinnvoll, jeden Textblock mit jedem
anderen zu vergleichen, da nur r"aumlich benachbarte Hypothesen zu einem Block vereinigt werden.
Bei $n$ Paragraphen w"aren $n(n-1)/2$ Vergleiche notwendig.
Allerdings kann der Aufwand von $O(n^2)$ deutlich
reduziert werden, wenn nur r"aumlich benachbarte Bl"ocke verglichen werden.
Um dieses Problem zu l"osen wird eine zweidimensionale Sortierung der Textbl"ocke nach r"aumlicher
N"ahe durchgef"uhrt. Diese Sortierung erfolgt durch die Berechung des
  {\em minimalen Spannbaumes\/} (Minimum Spanning Tree).

\newpage
{\bf Definition:}\\
Ein {\em Graph\/} $G=(V,E)$ ist eine Menge von {\em Knoten\/} (oder {\em Ecken\/}) $V$
und {\em Kanten\/} $E$. Knoten sind einfache Objekte; eine Kante ist eine Verbindung
zwischen zwei Knoten $(u,v)\in E$. Mit jeder Kante, die die Knoten $u$ und $v$ verbindet,
sind {\em Gewichte\/} oder {\em Kosten\/} $w(u,v)$ verkn"upft.

Der {\em minimale Spannbaum\/} eines gewichteten Graphen mit $V$ Knoten ist die
Menge $T \subseteq E$ von ($V-1$) Kanten, die alle Knoten so verbindet, da"s die Summe $w(T)$
der Kantengewichte minimal ist.

$$w(T) \stackrel{\scriptstyle\min}{=} \sum_{(u,v) \in T} w(u,v)$$

Der Berechnungsalgorithmus l"a"st sich folgenderma"sen beschreiben:
Einem Graphen werden,
beginnend mit der k"urzesten Kante, d.h.\ der Kante mit dem geringsten Gewicht,
nacheinander Kanten (die keinen Zyklus bilden) mit n"achstgr"o"seren Gewichten hinzugef"ugt.
Ein Zyklus ist ein Pfad, auf dem sich kein Knoten wiederholt.
Dieser Algorithmus ist bereits seit 1956 bekannt und wird J.\ Kruskal zugeschrieben
\cite{Algorithmen}.

Die Knoten werden hier als Textbl"ocke interpretiert. Jede der $V-1$ Kanten verbindet somit
zwei Textbl"ocke, d.h.\ diese Bl"ocke stehen in r"aumlicher N"ahe zueinander. Es gen"ugt nun
diese $V-1$ Paare f"ur die weitere Betrachtung heranzuziehen.
Um die Anzahl der zu untersuchenden Textbl"ocke noch weiter zu senken, werden
Bl"ocke, die sich zusammen in der durch XY-Cuts berechneten Gebiete befinden,
separat betrachtet. Ein Beispiel f"ur einen minimalen Spannbaum ist in
Abbildung \ref{eps:MSTDemo} zu sehen.

\Bild{MSTDemo}{6}{Minimaler Spannbaum}

Diese Kantengewichte $w(u,v)$ entsprechen den Abst"anden der Textbl"ocke zueinander, wobei
zus"atzlich der horizontale $\Delta x$ und vertikale $\Delta y$ Abstand unterschiedlich
bewertet werden kann ($w_x$ und $w_y$ sind als Parameter einstellbar -- siehe
\BildRef{ParameterPanelMerge} Seite \pageref{eps:ParameterPanelMerge}).
Ist zum Beispiel das Verh"altnis $w_x / w_y$ kleiner eins,
bedeutet dies, da"s nebeneinanderliegende Textbl"ocke bevorzugt durch den minimalen
Spannbaum verbunden werden.

$$w(u,v) = w_x \Delta x + w_y \Delta y$$

\Bild{Abstaende}{3}{Definition des Abstands zwischen zwei Textbl"ocken}

Zwei spezielle Teilprobleme bei der Berechnung des minimalen Spannbaumes sollen hier n"aher
betrachtet werden.
\begin{itemize}
  \item
        Erstens m"ussen die Kanten entsprechend der wachsenden Reihenfolge ihres Gewichts nacheinander
        betrachtet werden. Eine M"oglichkeit w"are, sie zu sortieren, doch es erweist sich als
        g"unstiger, eine Priorit"atswarteschlange (s.u.) zu benutzen, vor allem weil im Allgemeinen nicht
        alle Kanten betrachtet werden m"ussen, und daher eine vollst"andige Sortierung nicht
        notwendig ist.
  \item
        Zweitens mu"s "uberpr"uft werden, ob eine gegebene Kante einen Zyklus erzeugt, wenn sie den bisher
        verwendeten Kanten hinzugef"ugt wird. Dies kann mit der Vereinigungssuche (s.u.) erledigt werden.
        Der Proze"s wird beendet, wenn $V-1$ Kanten gefunden wurden, und somit alle $V$ Knoten mit diesen
        Kanten verbunden werden k"onnen.
\end{itemize}

\subsubsection{Priorit"atswarteschlangen}

Beim Aufbau des minimalen Spannbaumes m"ussen Datens"atze mit Schl"usseln (in diesem Fall
Gewichte bzw.\ Abst"ande) der Reihe nach verarbeitet werden; jedoch nicht in einer vollst"andig
sortierten Reihenfolge.
Eine geeignete Datenstruktur mu"s unter solchen Bedingungen die Eigenschaft haben, da"s sie
Operationen des Einf"ugens eines neuen Elements und des L"oschens das gr"o"sten Elementes
unterst"utzt. Eine derartige Datenstruktur wird {\em Priorit"atswarteschlange\/}
(Priority Queue) genannt.

Die Datenstruktur, die Operationen mit Priorit"atswarteschlangen erm"oglicht,
beinhaltet das Speichern der Datens"atze in einem Feld in der Art und Weise,
da"s jeder Schl"ussel garantiert
gr"o"ser ist als die Schl"ussel auf zwei bestimmten anderen Positionen. Jeder dieser
Schl"ussel mu"s wiederum gr"o"ser sein als zwei weitere Schl"ussel usw. Diese Ordnung
l"a"st sich sehr leicht veranschaulichen, indem das Feld in Form einer
bin"aren Baumstruktur gezeichnet wird. Von jedem Knoten f"uhren Linien
nach unten zu den beiden Knoten, von denen bekannt ist, da"s sie kleiner sind
(s.\ \BildRef{HeapBaum}).

\Bild{HeapBaum}{6}{Darstellung eines Heaps als vollst"andiger bin"arer Baum}

Diese Stuktur wird {\em vollst"andiger bin"arer Baum\/} genannt: Ausgehend von einem
Knoten ({\em Wurzel\/}) werden nach unten von links nach rechts jeweils
zwei weitere Knoten unter dem Knoten der vorangegangenen Ebene gezeichnet und mit
ihm verbunden. Die beiden Knoten unter jedem
Knoten werden dessen (direkte) {\em Nachfolger\/} (Children) genannt, der Knoten "uber
jedem Knoten hei"st dessen (direkter) {\em Vorg"anger\/} (Parent). Nun wird gefordert,
da"s die Schl"ussel im Baum der Heap--Bedingung gen"ugen:

\begin{quote}
  Der Schl"ussel in jedem Knoten mu"s gr"o"ser (oder gleich) sein als die Schl"ussel
  in seinen Nachfolgern (falls diese existieren). Der Knoten mit dem gr"o"sten Schl"ussel
  sitzt somit in der Wurzel.
\end{quote}

Die hier verwendete M"oglichkeit den Heap darzustellen, ist sequentiell innerhalb
eines Feldes $a[i]$, indem
die Wurzel auf die Position 1 gesetzt wird, ihre Nachfolger auf die Positionen
2 und 3, die Knoten der folgenden Ebene auf die Positionen 4, 5, 6 und 7 usw. Dabei
gilt
$$a[i] \; \ge \; a[2i] \qquad \mbox{und} \qquad a[i] \; \ge \; a[2i+1] \qquad 1\le i\le
  \left\lfloor n/2 \right\rfloor.$$

Alle Algorithmen, wie das Einf"ugen eines Knoten oder das Entfernen des gr"o"sten Knotens,
operieren entlang eines {\em Pfades\/} von der Wurzel zum unteren Ende des Heaps.
Dabei befinden sich in einem Heap mit $N$ Knoten auf allen Pfaden $\log_2 N$ Knoten.
Folglich k"onnen alle Operationen mit Priorit"atswarteschlangen bei Verwendung von Heaps in
logarithmischer Zeit ausgef"uhrt werden.

Um die Priorit"atswarteschlangen bei der Berechnung von Minimum Spanning Trees nutzen zu
k"onnen, ist zu beachten, da"s nicht die l"angste, sondern die {\em k"urzeste\/} Kante
(also die Kante mit dem kleinsten Gewicht) in der Wurzel sitzt.
Dies wird erreicht, wenn bei der Erzeugung des Heaps nicht das Gewicht
einer Kante (entspricht Abstand zwischen zwei Bl"ocken), sondern der davon negierte Wert
in die Knoten eingetragen wird.
%Somit ist sichergestellt, da"s die k"urzeste Kante in der Wurzel des Heaps sitzt.

\subsubsection{Vereiningungssuche}

Die Vereinigungssuche hat die Aufgabe festzustellen, ob die Knoten $u$ und $v$ miteinander
verbunden sind. Dies l"a"st sich "uber eine "`Verkettung mit dem Vorg"anger"' erreichen:

Das Feld $dad[u]$ mit $u=1\ldots V$ enth"alt f"ur jeden Knoten $u$ den Index seines Vorg"angers
mit einem Eintrag Null f"ur den Knoten, der sich an der Wurzel eines Baumes befindet. Um
den Vorg"anger $pred$ eines Knoten $u$ zu finden, setzt man $pred := dad[u]$, $u := pred$
und wiederholt diese Operation solange, bis die Wurzel des Baumes, die
zu dem der Knoten $u$ geh"ort, gefunden wurde (d.h.\ $u = 0$).
Dasselbe wird f"ur den Knoten $v$ durchgef"uhrt. Sind
die beiden gefundenen Wurzeln identisch so existiert bereits eine Verbindung
zwischen diesen Knoten, und das Einf"ugen der Kante $(u,v)$ erzeugt einen Zyklus.
Sind die Wurzeln verschieden, so wird der Baum mit der Wurzel $u$ mit dem Baum der Wurzel $v$
verkn"upft: $dad[v] := u$.

Der Aufwand des Minimum Spanning Tree--Algorithmus ist abh"angig von der Sortiermethode und
entspricht bei den hier implementierten bin"aren Heaps $O(E \log_2 E)$.

\subsection{Kriterien f"ur das Zusammenfassen}

Im folgenden werden nur noch Textblockhypothesen betrachtet, die im minimalen Spannbaum
mit einer Kante verbunden sind. Es sollen zun"achst folgende Eigenschaften zwischen diesen Blockpaaren untersucht werden:
\begin{description}
  \item[E1:]Bl"ocke sind (wie in Kapitel \ref{Konsistenzpruefung} definiert) zueinander homogen.
  \item[E2:]Die Summe des horizontalen und vertikalen Abstands (s.\ \BildRef{Abstaende})
  der Bl"ocke ist kleiner als $2\bar{h}$, d.h. $\Delta x + \Delta y < 2 \bar{h}$.
  \item[E3:]Bl"ocke liegen untereinander (s.\ \BildRef{UnterNeben}), wenn gilt
  ($X_{min}$, $Y_{min}$ entspricht der oberen linke Ecke der Bounding--Box, $X_{max}$, $Y_{max}$
  der unteren rechten Ecke und $\bar{h} = \max(\bar{h}_1,\bar{h}_2)$
  dem Maxima der mittleren Zeichenh"ohen der Bl"ocke)
  $$X_{min1} - \bar{h} < X_{min2}\quad\wedge\quad X_{max1} + \bar{h} > X_{max2}$$
  oder
  $$X_{min2} - \bar{h} < X_{min1}\quad\wedge\quad X_{max2} + \bar{h} > X_{max1}$$

  \item[E4:]Bl"ocke liegen nebeneinander (s.\ \BildRef{UnterNeben}), wenn gilt
  $$Y_{min1} - \bar{h} < Y_{min2}\quad\wedge\quad Y_{max1} + \bar{h} > Y_{max2}$$
  oder
  $$Y_{min2} - \frac{\bar{h}}{2} < Y_{min1}\quad\wedge\quad Y_{max2} + \frac{\bar{h}}{2} > Y_{max1}$$

  \item[E5:]Beide Bl"ocke enthalten nur nichtproportionale Schrift. Die Klassifikation proportional
  bzw.\ nichtproportional erfolgt mit einem bestehenden Modul. Sie war nicht Gegenstand dieser Arbeit.
  \item[E6:]Ein Block enth"alt nur eine Textzeile, d.h.
  $$\frac{Y_{max1} - Y_{min1}}{\bar{h}_1} \; < \;3 \qquad\mbox{oder}\qquad
    \frac{Y_{max2} - Y_{min2}}{\bar{h}_2} \; < \;3$$
\end{description}

\Bild{UnterNeben}{10}{Definition f"ur das Untereinander-- und Nebeneinanderliegen von Bl"ocken}

Die zwei Bl"ocke werden verbunden, wenn gilt:
$$E1 \; \wedge \; E2 \; \wedge \; ( \; E3 \; \vee \; E4 \; \wedge \; ( \; E5 \; \vee \; E6 \; )\;)$$

Grunds"atzliche Bedingung ist die Homogenit"at zwischen den Textbl"ocken (E1) und ein nicht zu gro"ser
Abstand (E2). Wenn die Bl"ocke au"serdem untereinander liegen (E3), werden sie verbunden.
Bei der Verbindung von nebeneinanderliegenden Bl"ocken mu"s darauf geachtet werden, da"s nicht "uber Spalten hinweg verbunden wird.
Hierbei wurde folgende Beobachtung gemacht: Innerhalb der Trainingsdokumente gab es keinen
Fall, bei dem neben nichtproportionaler Schrift (h"aufig in Gesch"aftsbriefen) mehrere
Spalten vorkamen.  Daher ist es nicht kritisch, Bl"ocke, die eine nichtproportionale Schrift
enthalten (E5) und nebeneinander liegen (E4), zu verbinden.
Enth"alt eine Blockhypothese nur eine Textzeile (E6), so wird es sich wahrscheinlich nicht
um eine angrenzende Spalte, sondern um eine getrennte Zeile handeln. In diesem Fall werden
trotz Proportionalschrift (d.h.\ E5 gilt nicht) nebeneinanderliegende Bl"ocke (E4) verbunden.

\subsection{Relaxationsmethode}

Nachdem der minimale Spannbaum aufgebaut wurde, erh"alt man $n-1$
in r"aumlicher N"ahe stehende Blockpaare. Diese Bl"ocke werden auf die im vorherigen Abschnitt
beschriebenen Kriterien hin untersucht und gegebenenfalls miteinander verbunden.

Um sicherzustellen, da"s auch zusammengefa"ste Textbl"ocke wiederum verbunden werden k"onnen,
wird das Vorgehen solange iterativ angewandt, bis ein station"arer Zustand vorherrscht
(s.\ \BildRef{Relaxation}). Dieses Vorgehen entspricht einer Relaxation:

\begin{enumerate}
  \item Aufbau des minimalen Spannbaumes.
  \item Auswertung der Kanten des Graphen. Es werden $n-1$ Vergleiche durchgef"uhrt und ggf.\
        Textblockpaare verbunden.
  \item Wurde mindestens ein Blockpaar zusammengefa"st, so wird zum ersten Schritt mit dem neuen
        Satz von Textbl"ocken zur"uckgesprungen.
\end{enumerate}

\Bild{Relaxation}{5}{Iteratives Vorgehen beim Zusammenfassen von Textbl"ocken}

{\bf Parametrierung}

\Bild{ParameterPanelMerge}{15}{Parameter f"ur das Zusammenfassen von Textbl"ocken}

\clearpage
\section{Segmentierungsergebnisse}\label{ErgebnisZoning}

\subsection{Lern-- und Teststichprobe}
Um die Leistungsf"ahigkeit, d.h.\ sowohl St"arken als auch Schw"achen,
aufzuzeigen, sind die Verfahren an 163 unterschiedlichen Dokumenten und Datens"atzen getestet worden.
Die Stichproben umfassen f"unf verschiedene Dokumentklassen, mit jeweiligen Besonderheiten
bez"uglich ihres Layouts (Beispiele siehe Anhang \ref{DokuKlassen}):

\begin{description}
  \item[Gesch"aftsbriefe:]Einspaltig, variables Layout, gemischtes Vorkommen vom
  Pro\-por\-tio\-nal-- und Nichtproportionalschrift,
  sehr variable Schriftgr"o"sen, Nicht--Textelemente wie Unterschrift, handschriftliche Anmerkungen,
  Firmenlogos.
  \item[Wissenschaftliche Artikel:]Ein-- bis dreispaltig, klar strukturiertes Manhattan--Layout,
  Proportionalschrift, gleichm"a"sige Schriftgr"o"se (au"ser "Uberschriften),
  Nicht--Text\-ele\-mente wie Diagramme, Tabellen, Formeln.
  \item[Zeitschriften:]Ein-- bis vierspaltig, variables Nicht--Manhattan--Layout,
  Proportionalschrift, gleichm"a"sige Schriftgr"o"se (au"ser "Uberschriften), trennende Liniensegmente,
  Nicht--Textelemente wie Diagramme, Tabellen, Photos.
  \item[Tageszeitungen:]Bis zu sechsspaltig, klar strukturiertes Layout, Proportionalschrift,
  gleichm"a"sige Schriftgr"o"se (au"ser "Uberschriften), viele trennende Liniensegmente, nahe 
  aneinandergrenzende Textspalten,
  Nicht--Textelemente wie Diagramme, Tabellen, Formeln, Halbtonbilder.
  \item[Steuerbescheide:]Ein-- bis zweispaltig, Text in Tabellenform, Nichtproportionalschrift,
  gleichm"a"sige Schriftgr"o"se, Nicht--Textelemente wie Tabellen, Stempel, handschriftliche
  Anmerkungen.
\end{description}

Anhand einer Lernstichprobe von 47 zuf"allig ausgew"ahlten Dokumenten aus den oben 
genannten Klassen wurden die Verfahren getestet und
die Parameter optimiert. Die St"arken und Schw"achen wurden an einer Teststichprobe von einem Satz
von 116 unbekannten Dokumenten aufgezeigt.

Um die berechneten Ergebnisse automatisch bewerten zu k"onnen, mu"sten Referenzdaten erzeugt werden.
Dazu wurden alle Dokumente beider Stichproben manuell in rechteckige Gebiete eingeteilt und
entweder mit dem Namen "`Text"' oder "`Nicht--Text"' beschriftet (engl.\ {\em to label\/}).
Jedes dieser Gebiete ({\em Labels\/}) ist somit durch seinen Namen
und sein umschreibendes Rechteck charakterisiert.

Bei der manuellen Einteilung der Dokumente stellte sich die Frage nach der Definition
eines Textblocks. Der Leser kann als Beispiel diese und die n"achste Dokumentseite heranziehen. 
Bei der nummerierten Liste k"onnte eine Textblocksegmentierung
auf mehrere Arten erfolgen: Entweder {\em logisch\/}, d.h.\ jeder
einzelne Aufz"ahlungspunkt wird in einem Textblock untergebracht, oder {\em geometrisch\/}, d.h.\
die gesamte Liste entspricht, da sie homogenen Text enth"alt, einem einzigen Block. 
Kombinationen von logischer und geometrischer Segmentierung sind ebenfalls
vorstellbar. Je komplexer das Dokumentlayout, desto uneindeutiger ist die Aufgabe. Das Spektrum
reicht von relativ einfach zu segmentierenden wissenschaftlichen Artikeln bis zu dem
Extremfall der Steuerbescheide, die u.a.\ viele Tabellen enthalten. 
Wie sollen diese Tabellen segmentiert werden?
Eine Vielzahl vom M"oglichkeiten bietet sich an: ein Block
f"ur die gesamte Tabelle, ein Block f"ur jede Zeile bzw.\ Spalte oder ein Block f"ur jeden
einzelnen Tabelleneintrag.

Weil die entwickelten Verfahren nur die geometrischen Verh"altnisse ber"ucksichtigen
k"onnen, wurde folgende Pr"amisse eingegangen:
\begin{quote}
Ein Textblock ist definiert als die maximale Menge von homogenen Zusammenhangsobjekten,
homogen nach den beiden definierten Kriterien Buchstabenh"ohe und Strichdicke. 
Ein Textblock ist
charakterisiert durch die darin enthaltenen BCC--Objekte, das umschreibende Rechteck, die
mittlere Objekth"ohe, die mittlere relative Strichst"arke, die Teilung (proportional oder
nichtproportional) und durch die Position innerhalb der Lesefolge (siehe Kapitel
\ref{Lesefolge})
\end{quote}

Zus"atzlich zur Lern-- und Teststichprobe stand eine Datenbank von der 
Universit"at Washington mit fertig `gelabelten' Dokumenten zur Verf"ugung ({\em UWash\/}--Daten).
Diese Dokumente hatten die Besonderheit, da"s die Labels logische Textbl"ocke enthalten
Zudem wurden nur Bereiche mit
Diagrammen und Tabellen als Nicht--Text beschriftet, jedoch nicht St"orungen durch Rauschen,
Papierknicke etc. Die sich daraus ergebenden Probleme sind im n"achsten Abschnitt beschrieben.
Um aussagekr"aftige Ergebnisse zu erhalten, ist es zu empfehlen, da"s die Person,
die die Labels erzeugt, mit dem zugrundeliegenden Segmentierungsverfahren oder zumindest mit der
Textblockdefinition vertraut ist.

\subsection{M"oglichkeiten zur Bestimmung der Erkennungsleistung}

Bei der Auswertung wird anhand von Referenztextbl"ocken und Textblockhypothesen
eine Erkennungsrate berechnet, die als Ma"s f"ur die Leistungsf"ahigkeit der Verfahren dienen soll.
In den vorangegangenen Erkennungsphasen wurden $n$ Textblockhypothesen erzeugt, die mit $k$
vorliegenden Referenzen in "Ubereinstimmung gebracht werden m"ussen, wobei in der Regel
$k \ne n$ ist. Dabei ist es aufwendig $k$ aus $n$ Textbl"ocken miteinander zu vergleichen.
Desweiteren ist nicht eindeutig festgelegt, unter welchen Bedingungen zwei Textbl"ocke
"ubereinstimmen (bei Identit"at oder wenn Textblockhypothesen innerhalb der Referenzbl"ocke liegen).
Im Vergleich zu der Auswertung von Zeichenklassifikationsergebnissen handelt es sich hier
folglich um ein viel schwierigeres Problem: Im letzteren Fall gilt es n"amlich nur zu
"uberpr"ufen, ob die Zeichenhypothese mit der Referenz "ubereinstimmt.

Erkennungsraten lassen sich "uber zwei Ans"atze definieren -- {\em Recall\/} und {\em Precision\/}:
\begin{description}
\item[Recall:]Die Erkennungsrate $ER_{R}$ ergibt sich aus der Anzahl der korrekt erkannten
Textblockhypothesen $H_K$ bezogen auf die Gesamtanzahl der Hypothesen $H$:

$$ER_{R} = \frac{H_K}{H}$$

Bei dieser Definition ergibt sich folgendes Problem: Es k"onnen durchaus alle tat\-s"ach\-lich
existierenden Textbl"ocke korrekt erkannt worden sein. Die durch die Vielzahl kleiner St"orungen
zus"atzlich entstandenen vermeintlichen Textblockhypothesen bewirken aber, da"s die
Erkennungsrate $ER_{R}$ sinkt. 
Dies soll an einem Zahlenbeispiel verdeutlicht werden: Ein Dokument
enthalte 10 Textbl"ocke. Die Segmentierung erkennt zus"atzlich zu den 10 richtigen Bl"ocken 
($H_K=10$) noch 5 weitere, die nur St"orungen beinhalten ($H=15$). 
Die Erkennungsrate $ER_{R}$ l"age in diesem Fall bei 67\%, obwohl alle Textbl"ocke erkannt wurden
und das Gesamterkennungsergebnis auf Zeichenebene gut sein kann.

\item[Precision:]Um diesen Nachteil zu vermeiden wird eine zweite Erkennungsrate $ER_{P}$ definiert. Hierbei
erfolgt die Normierung der korrekt erkannten Textblockhypothesen $H_K$ auf die Anzahl der
Referenzbl"ocke $R$, d.h.\ es flie"sen nur die richtigen Textblockhypothesen in die Erkennungsrate
$ER_{P}$ ein:

$$ER_{P} = \frac{H_K}{R}$$

Angewandt auf obiges Beispiel ergibt sich eine Erkennungsrate $ER_{P} = 100$\%. 
Neben diesem Wert sollte in diesem Fall zudem die Anzahl der Textblockhypothesen 
angegeben werden, die zus"atzlich zu den korrekten erzeugt wurden:
$$H_{Rest} = H - H_K$$
\end{description}

Um die Erkennungsergebnisse auch im Bezug auf nachfolgende Segmentierungsschritte und OCR--Analysen
vergleichen zu k"onnen, werden neben Recall und Precision weitere Erkennungsraten definiert.
Die Auswertung erfolgt nach folgenden Regeln:
\begin{enumerate}
  \item Bl"ocke, die sich innerhalb von gelabelten Nicht--Textgebieten befinden,
  werden aus der Menge der hypothetischen Textbl"ocke $H$ entfernt. Daraus ergibt sich $H_T$.
  
  \item Die Erkennungsraten $ER_{R}$ und $ER_{P}$ ergeben sich, wie zuvor geschildert:
  $ER_{R}$ aus der Anzahl der korrekt erkannten Textbl"ocke $H_K$, bezogen auf die 
  Gesamtanzahl der Textblockhypothesen $H_T$ und $ER_{P}$ aus $H_K$ bezogen auf die Anzahl
  der Referenztextbl"ocke $R_T$. Die zus"atzlich zu den korrekten Textblockhypothesen
  erzeugten Bl"ocke lassen sich "uber $ H_{Rest} = H_T - H_K$ berechnen. Es handelt sich hierbei
  um diejenigen Bl"ocke, die f"alschlicherweise zusammengefa"st, 
  getrennt oder durch St"orungen erzeugt wurden.
  
  $$ER_{R} = \frac{H_K}{H_T}$$
  $$ER_{P} = \frac{H_K}{R_T}\qquad\qquad H_{Rest} = H_T - H_K$$
  
  \item Da eine Trennung von homogenen Textbl"ocken in mehrere Teilbl"ocke die folgenden
  Analyseschritte nicht negativ beeinflu"st, wird eine weitere Erkennungsrate $ER_{R}^{g}$
  berechnet. In ihr werden zus"atzlich zu den korrekt erkannten Bl"ocken $H_K$ noch diejenigen
  Textblockhypothesen ber"ucksichtigt, die sich innerhalb der Referenztextbl"ocke befinden 
  ($H_{g}$). Hierbei wird nicht auf die Anzahl der Referenzbl"ocke $R_T$ normiert,
  sondern auf die Anzahl der erkannten Textblockhypothesen $H_T$, da sich ansonsten 
  Erkennungsraten von "uber 100\% ergeben w"urden.
  
  $$ER_{R}^{g} = \frac{H_K + H_{g}}{H_T}$$
    
  \item Um festzustellen wie stark Nicht--Textobjekte das Segmentierungsergebnis beeinflussen wird
  eine dritte Rate $ER_{R}^{NTv}$ definiert: In ihr werden neben $H_K$ und 
  $H_{g}$ noch Textblockhypothesen einbezogen, die sich mit Nicht--Textgebieten "uberlappen 
  ($H_{NTv}$), d.h.\ bei denen Text-- und Nicht--Textobjekte verbunden wurden:
  $$ER_{R}^{NTv} = \frac{H_K + H_{g} + H_{NTv}}{H_T}$$
\end{enumerate}

Zwei Arten von Segmentierungsfehlern sind in Anbetracht der weiteren 
Segmentierungsschritte und der OCR--Analyse zu unterscheiden:
\begin{enumerate}
  \item {\bf Homogene Bl"ocke sind getrennt worden}; dies ist der Fall, wenn die Konsistenzpr"ufung
  einen Block aufgrund zu stark einschr"ankender Homogenit"atsbedingungen getrennt hat, oder wenn
  die Regeln f"ur das Zusammenfassen von Textbl"ocken keine Verbindung vorsahen. 
  Es entstehen bei der OCR--Analyse keine Folgefehler aufgrund dieser fehlerhaften 
  Segmentierung. Der einzige Nachteil dieses Fehlers besteht darin, da"s folgende Analyseschritte,
  aufgrund kleinerer Textbl"ocke mit einer geringeren Anzahl von Elementen,
  mit kleineren Statistiken auskommen m"ussen.
  
  \item {\bf Bl"ocke sind f"alschlicherweise zusammengefa"st worden}, so da"s
  der resultierende Block
    \begin{itemize}
      \item nichthomogen bez"uglich Buchstabenh"ohe oder --st"arke ist,
      \item Text-- und Nicht--Textgebiete beinhaltet oder
      \item zwei nebeneinanderliegende Spalten zusammenfa"st.
    \end{itemize}
  Segmentierungsfehler durch Nichthomogenit"aten oder Nicht--Textobjekte innerhalb von
  Textbl"ocken f"uhren meist zu einzelnen Fehlern bei der anschlie"senden Zeilen--, W"orter-- und
  Zeichensegmentierung und der OCR--Analyse. Der Extremfall w"are die Verbindung von einzelnen
  Textspalten, die das Gesamtergebnis allerdings nahezu unbrauchbar machen.
\end{enumerate}

Aus dieser Unterscheidung folgt, da"s Fehler aus dem ersten Fall toleriert werden k"onnen,
aber Fehler aus dem zweiten Fall unbedingt vermieden werden m"ussen.

Folglich ist nach dieser Definition ein {\em echter\/} Fehler nur dann aufgetreten, wenn eine
Hypothese mehrere Referenzbl"ocke enth"alt. In diesem Fall kann der Block inhomogen sein
oder mehrere Textspalten einschlie"sen. 
An dieser Stelle wird das Problem bei der Auswertung des UWash Datensatzes deutlich:
Diese Dokumente wurden in {\em logische\/} Bl"ocke eingeteilt, die Segmentierungsverfahren 
fassen aber die gr"o"stm"ogliche Menge von homogenen Objekten zu einem Block zusammen. Liegen
mehrere Referenzbl"ocke innerhalb einer Textblockhypothese, werden die Hypothesen
als {\em falsch\/} bewertet (Die Auswertung schlie"st auf eine Nichthomogenit"at). 
Grunds"atzlich sollten aus diesem Grund die Referenzbl"ocke nicht zu fein gelabelt werden.

\subsection{Erkennunsergebnisse}
Mit diesen Definitionen sind folgende Ergebnisse erziehlt worden:

\subsubsection{Lernstichprobe}
Die Lernstichprobe umfa"ste 47 Dokumente aus den oben genannten Klassen. Sie hatte das Ziel, 
die Methoden zu testen und ggf.\ Fehler zu beseitigen bzw.\ auf Sonderf"alle hinzuweisen.

\begin{center}\small
\begin{tabular}{|l|r|r|r|r|r|r|r|}
\hline
Dokumentklasse   &   Anzahl & Mittlere& $ER_{P}$& $H_{Rest}$ & $ER_{R}$ & $ER_{R}^{g}$ & $ER_{R}^{NTv}$\\
                 &          &Blockanzahl& &  & & & \\
\hline
Gesch"aftsbriefe           & 14 & 14 & 97.1 & 1.1 & 91.4 & 95.7 & 96.8\\
Wissensch.\ Artikel       & 11 & 10 & 87.0 & 1.4 & 88.1 & 95.9 & 95.9\\
Zeitschriften             & 13 & 16 & 96.3 & 1.5 & 94.0 & 96.6 & 97.5\\
Tageszeitungen            & 11 & 25 & 92.5 & 3.8 & 84.2 & 95.6 & 96.6\\
Steuerbescheide           &  8 & 30 & 85.4 & 5.6 & 85.5 & 92.8 & 96.2\\
\hline
Gesamt                    & 47 & 19 & 91.7 & 2.7 & 88.6 & 95.3 & 96.6\\
\hline
\end{tabular}\\[3mm]
Ergebnisse der Lernstichprobe
\end{center}

\subsubsection{Teststichprobe}
Die Teststichprobe umfa"ste 116 Dokumente mit den identischen Typenklassen der Lernstichprobe,
wobei 50 Dokumente aus der UWash--Datenbank entnommen wurden (wissenschaftliche Artikel mit
mehrern Spalten und variablem Layout).

\begin{center}\small
\begin{tabular}{|l|r|r|r|r|r|r|r|}
\hline
Dokumentklasse   &   Anzahl & Mittlere& $ER_{P}$& $H_{Rest}$ & $ER_{R}$ & $ER_{R}^{g}$ & $ER_{R}^{NTv}$\\
                 &          &Blockanzahl& &  & & & \\
\hline
Gesch"aftsbriefe           & 20 & 11 & 93.8 & 1.5 & 88.4 & 96.3 & 99.2\\
Wissensch.\ Artikel       & 20 & 15 & 91.5 & 2.8 & 83.5 & 95.3 & 97.3\\
Zeitschriften             & 10 & 19 & 85.6 & 5.6 & 71.9 & 89.9 & 90.9\\
Tageszeitungen            &  6 & 20 & 90.3 & 2.2 & 90.2 & 96.2 & 97.5\\
Steuerbescheide           & 10 & 21 & 92.2 & 4.2 & 90.7 & 94.5 & 95.2\\
UWash Dokumente           & 50 & 19 & 26.9 & 15.5& 21.5 & 45.9 & 48.3\\
\hline
Gesamt                    &116 & 18 & 80.1& 5.3& 72.7 & 86.3 & 88.1\\
\hline
Gesamt ohne UWash         & 66 & 17 & 90.7& 3.3& 82.9 & 94.4 & 96.0\\
\hline
\end{tabular}\\[3mm]
Ergebnisse der Teststichprobe
\end{center}

\subsubsection{Laufzeitmessung} \label{Speed}
Die Algorithmen wurden an vielen Stellen auf Geschwindigkeit optimiert (z.B.\ Zusammenfassen
von Histogrammklassen, Unterabtastung, Minimum Spanning Tree, etc.). Die Versuche wurden auf zwei
verschiedenen Rechnern durchgef"uhrt: Auf einer  NeXT--Workstation mit Motorola 68040 25MHz
Prozessor und auf einer leistungsf"ahigeren Hewlett Packard 712/80 Maschine mit einem 
HP-PA RISC 80MHz Prozessor.

\begin{center}\small
\begin{tabular}{|p{3cm}|c|c|c|c|c|c|c|c|}
\hline
Sekunden &\multicolumn{2}{p{25mm}|}{Gesch"aftsbrief}&\multicolumn{2}{p{25mm}|}{Wissenschaft\-licher Artikel}&
    \multicolumn{2}{p{25mm}|}{Zeitschrift}&\multicolumn{2}{p{25mm}|}{Tageszeitung}\\
    & 68k & HP-PA & 68k & HP-PA & 68k & HP-PA & 68k & HP-PA\\
\hline
BCC--Anzahl &\multicolumn{2}{c|}{928}&\multicolumn{2}{c|}{2515}&
\multicolumn{2}{c|}{7057}&\multicolumn{2}{c|}{9924}\\
\hline
Filterung & 0.09 & 0.03 & 0.6 & 0.05 & 6.0 & 1.5 & 5.3 & 1.4\\
\hline
XY--Cut   & 0.9 & 0.3 & 1.7 & 0.5 & 1.6 & 0.5 & 1.8 & 1.1\\
\hline
Smearing und& 3.2 & 0.7 & 5.5 & 1.2 & 9.0 & 2.4 & 10.8 & 5.3\\
Blockzuordnung& (11.3)& (2.4) & (15.3) & (3.5) & (25.5) & (6.0) & (30.0) & (8.2) \\
\hline
Nachbearbeitung & 4.5 & 0.7 & 5.4 & 1.2 & 12.0 & 3.2 & 12.2 & 4.5\\
\hline
Sortierung & 0.03 & $<$ 0.01&0.02 & $<$ 0.01& 0.03 & $<$ 0.01& 0.03 & $<$ 0.01\\
\hline\hline
Summe & 8.7 &1.7& 13.2 & 3.0& 28.6 & 7.6& 30.1 & 12.3\\
 & (16.8) & (3.4) & (23.0) & (5.3) & (45.1) & (11.2) & (49.3) & (15.2)\\
\hline
\end{tabular}\\[3mm]
Beispielhafte Ausf"uhrungszeiten in Sekunden\\(Werte in Klammern gelten ohne Unterabtastung)
\end{center}

%%%%%%%%%%%%%%%%%%%%%%%%%%%%%%%%%%%%%%%%%%%%%%%%%%%%%%%%%%%%%%%%%%%%%%%%%%%%%%%%%%%%%%%
\subsection{Diskussion der Ergebnisse}\label{ErgDiskuss}

Die in dieser Arbeit vorgestellten Methoden zur Segmentierung eines Dokumentbildes in
homogene Textbl"ocke wurden an 163 Dokumenten mit verschiedenartigem Layout
untersucht. Die Verfahren zeigen gute bis sehr gute Ergebnisse bei den betrachteten
Dokumentklassen. Es wurde eine Erkennungsrate von bis zu 96\% ($ER_{R}^{g}$) erreicht. 
Zwischen der bekannten Lernstichprobe und der unbekannten Teststichprobe traten keine
nennenswerten Verschlechterungen in den Erkennungsraten auf. Dies l"a"st auf eine
gute Adaption der Parameter schlie"sen.

Die Ausrei"ser bei den Dokumenten aus der UWash Datenbank ($ER_{R}^{g}$ = 45.9\%) 
sind durch zwei Effekte entstanden:
Bei den UWash Dokumenten wurde jedem logischen Absatz ein eigener Textblock zugeordnet. Dies
widerspricht der oben genannten Definition von der maximalen Gr"o"se eines Blocks.
Kamen mehrere Referenzbl"ocke innerhalb einer Textblockhypothese zum liegen, wurden die Hypothesen
als {\em falsch\/} bewertet (Die Auswertung schlie"st auf eine Nichthomogenit"at). Daraus
ergibt sich eine $ER_{P}$ von 26.9\% und eine sehr hohe Anzahl von
nicht korrekten Textblockhypothesen (im Durchschnitt $H_{Rest} = 15.5$). Diese geringe
Erkennungsrate ist zudem durch die fehlende Labelung von St"orungen (schlechte Vorlagenqualit"at)
als Nicht--Textgebiete zu begr"unden.

Vergleicht man die Werte von $ER_{R}$ und $ER_{R}^{g}$  ist daraus ersichtlich, da"s
die Verfahren die Textbl"ocke immer feiner segmentieren als es die Referenztextbl"ocke
vorschreiben. Meist trennen die Konsistenzpr"ufungen st"arker (objektiver) oder es wird weniger
zusammengefa"st als es der Bearbeiter bei der Erstellung der Referenzdaten
vermutet.

\Bild{TrenntZuStark}{14.5}{Konsistenzpr"ufung trennt zu stark}
\newpage
Dies ist der Fall, wenn (s.\ \BildRef{TrenntZuStark})
\begin{itemize}
  \item nur ein Teil der Textzeile eine h"ohere Strichdicke aufweist wie der Rest (1), 
    
  \item die Nichthomogenit"aten f"ur den (menschlichen) Betrachter kaum sichtbar sind (2), 
  
  \item zwischen zwei Textzeilen die mittlere Zeichenh"ohe variiert,
  z.B.\ wenn innerhalb einer Zeile viele Gro"sbuchstaben, Zahlen oder Klammern auftauchen (3),
  
  \item der Abstand der Textbl"ocke (knapp) "uber der Schwelle liegt (4)

  \item Unterscheidet sich die Blockkontur zu sehr von dem umschreibenden Rechteck, so kann
   beim Zusammenfassen von homogenen Bl"ocken folgender Fehler auftreten 
   (s.\ \BildRef{ZusammenfassenProblem}): Man betrachte die Bl"ocke TB1, TB3 und TB4. 
   Der minimale Spannbaum verbindet nur TB3 mit TB1 und TB4 mit TB1 aufgrund der
   Abstandsverh"altnisse der umschreibenden Blockrechtecke mit einer Kante (Abstand 
   zwischen TB1--TB3 und TB1--TB4 ist jeweils null, der Abstand TB3--TB4 dagegen gr"o"ser null).
   Obwohl TB3 und TB4 zueinander homogen sind und sich in
   ausreichender N"ahe zueinander befinden, werden sie nicht zusammengefa"st.
   Diese Problem tritt speziell bei komplexen Zeitschriftenlayouts auf 
   (s.\ \BildRef{ZusammenfassenProblemDemo}). Aus diesem Grund l"a"st sich
   auch die um etwa 20\% niederere Erkennungsrate $ER_{R}$ der Zeitschriften im Vergleich zu den
   weniger komplexen Layouts der sonstigen Klassen erkl"aren.
   
   \Bild{ZusammenfassenProblem}{8}{Das Zusammenfassen bereitet Probleme, wenn sich die
   umschreibenen Rechtecke der Blockkonturen "uberlappen}
   \Bild{ZusammenfassenProblemDemo}{8}{Aufgrund "uberlappender Bounding--Boxes verbindet keine Kante
   des minimalen Spannbaumes die homogenen Zeilen in der Dokumentmitte}
\end{itemize}

Eine Trennung von homogenen Textbl"ocken beeinflu"st die nachfolgenden Dokumentanalyseschritte,
mit Ausname der aufgrund weniger Objekte bedingten Statistikeffekte, nicht. 
Aus diesem Grund gibt der Wert $ER_{R}^{g}$ mehr Aufschlu"s "uber die
Leistungsf"ahigkeit der Segmentierungsverfahren innerhalb des Gesamtsystems.

Im Gegensatz dazu f"uhrt eine Zusammenfassung von nichthomogenen Bl"ocken in der Regel
zu Fehlern in den nachfolgenden Erkennungsschritten.
Das Spektrum reicht hierbei von einzelnen Zeichenerkennungsfehlern bis zu
unbrauchbaren Ergebnissen bei verbundenen Textspalten.

  In vier F"allen wurden bei der Segmentierung von Textbl"ocken in Tageszeitungen
  innerhalb der Teststichprobe 
  angrenzende Spalten verbunden. Bei genauer Analyse der Ursachen, stellte sich heraus, da"s diese
  Dokumente jeweils quer digitalisiert wurden. Wie schon in Kapitel \ref{Filterung}
  (Filter-Parametrierung) erw"ahnt, mu"s bei Dokumenten die nicht im DIN--A4--Hochformat 
  vorliegen die Umrechnung von Punkt in Pixel angepa"st werden. Nach der
  Adaption der Parameter trat eine Spaltenverbindung nur noch in einem
  Fall auf (s.\ \BildRef{SpaltenVerbunden}): Eine Grafik zwischen den zwei Spalten wirkte beim
  Smearing als Br"ucke. 

  \Bild{SpaltenVerbunden}{6.5}{Einziger Fall von verbundenen Spalten}
\newpage
An dieser Stelle soll, in Bezug auf die bestehenden Probleme, diskutiert werden, inwieweit die
gestellten Anforderungen an die Verfahren aus Kapitel \ref{Anforderungen} Seite
\pageref{Anforderungen} erf"ullt wurden:

\begin{itemize}
  \item Die Verfahren zeigen konstant gute Ergebnisse auf den verschiedenen Dokumentklassen, wobei
  die Klasse mit sehr komplexem Layout und vielen Nicht--Textelementen (Zeitschriften)
  erwartungsgem"a"s 5\% unter der durchschnittlichen Erkennungsrate liegt.
  
  \item Eine Vielzahl von Parametern sind dem Benutzer zug"anglich und vor dem Beginn der Analyse
  ver"anderbar (n"aheres siehe in den Parametrierungsabschnitten von Kapitel \ref{BestimmungZones}
  und zusammenfassend im Anhang \ref{ParameterPanel}).

  \item Alle Verfahren werden grunds"atzlich automatisch parametriert. Die Ausnahme bilden
  Dokumente, die nicht im DIN--A4--Hochformat digitalisiert wurden. In diesem Fall m"ussen die
  entsprechenden Parameter angepa"st werden.
  
  \item Dokumente mit mehrspaltigem Layout werden ebenso korrekt segmentiert wie Layouts mit 
  nicht rechteckigem Konturenverlauf. Hierbei bereiten "uberlappende, umschreibende Rechtecke der
  Blockkonturen beim Zusammenfassen teilweise Probleme (s.u.).
  
  \item S"amtliche Segmentierungsverfahren wurden implementiert und in die bestehende
  Experimentierumgebung integriert.

\end{itemize}
\newpage
Im Folgenden werden weitere aufgetretene Probleme diskutiert:
\begin{itemize}
  \item  Das gr"o"ste Problem stellen die vorhandenen Nicht--Textobjekte dar.
  Dies sind Unterschriften, Grafiken und sonstige handschriftliche
  Eintragungen in Gesch"aftsbriefen, sowie Tabellen und Halbtonbilder in Zeitschriften oder
  Tageszeitungen. Hinzu kommen die fast unvermeidbaren St"orungen durch schlechte, oft kopierte
  Vorlagen. Die Anwendung der Homogenit"atskriterien bewirkt auf diesen Gebieten keine
  sinnvollen Ergebnisse. Da diese Nicht--Textobjekte selten als solche erkannt und gel"oscht 
  werden k"onnen, es sei denn, sie "uber-- oder unterschreiten die Filterschwelle, werden sie von den
  Verfahren wie Textobjekte behandelt. Dies bedeutet, da"s sie sich zusammen mit
  Textobjekten oder separat innerhalb von Textbl"ocken befinden k"onnen 
  (s.\ \BildRef{SmearingGrafik}). Ein Vergleich zwischen $ER_{R}^{g}$ und $ER_{R}^{NTv}$ zeigt, 
  da"s bei einer vorgschaltetem Text--Grafik--Unterscheidung die Erkennungsraten im Einzelfall
  (Steuerbescheide) um bis zu 5\% erh"oht werden k"onnten. 
   
    \Bild{SmearingGrafik}{11}{Text-- und Nicht--(Maschinengeschriebene)--Textobjekte innerhalb eines Blocks}
  \item Schwierigkeiten bereitete die Aufstellung der Gewichtungsfunktion f"ur den maximalen 
  relativen Fehler (s.\ \BildRef{ErrWeight} auf Seite \pageref{eps:ErrWeight}). Ziel war es,
  die Homogenit"atskriterien bei einer kleinen Statistik, d.h.\ bei wenig Objekten in einer Zeile,
  weniger streng anzuwenden. Dennoch m"ussen auch kurze nichthomogene Zeilen von einem
  Block abgetrennt werden k"onnen. Diese widerspr"uchlichen Anforderungen an die
  Gewichtungsfunktion k"onnen im Einzelfall zu Fehlern f"uhren (s.\ \BildRef{ProblemGewichtFkt}).
  
  \Bild{ProblemGewichtFkt}{9}{Probleme bei wenigen Objekten: kurze Zeile abgetrennt (oben), nichthomogene kurze "Uberschrift zusammengefa"st (unten)}

  \item Bei der Untersuchung einer Textzeile nach einem beinhalteten Initial
  wurde von der Annahme ausgegangen, da"s sich ein Initial stets am
  Anfang, also in der oberen linken Ecke, eines Blocks befindet. Aus diesem Grund werden
  Initialien, welche sich innerhalb eines hypothetischen Blocks befinden 
  (s.\ \BildRef{InitialImBlock}) nicht als solche klassifiziert und deshalb auch nicht extrahiert.
  
  \Bild{InitialImBlock}{5}{Nicht erkanntes Initial in einem Block}
  
   \item Ein m"oglicher Fehler, der aber in den Trainings-- und Testdatens"atzen nicht auftrat, 
   w"are eine nicht zu verhindernde falsche Zusammenfassung von Bl"ocken durch die in Kapitel
   \ref{Zusammenfassen} aufgestellten Regeln. Sie verhindern nicht, da"s "uberlappende Textbl"ocke,
   bei denen es sich um verschiedene Spalten handeln k"onnte, zusammengefa"st werden.
   Um sich dies zu verdeutlichen betrachte man die Textbl"ocke TB1 und TB2 aus 
   Abbildung \ref{eps:ZusammenfassenProblem}.
     
  \item Wird bei Tabellen, die in der Dokumenklasse der Steuerbescheide geh"auft 
  auftreten, f"ur jede Tabellenspalte oder Tabellenzeile ein eigener Textblock gew"unscht, so
  mu"s die interne Grafikstruktur eines BCC--Objekts bei der Segmentierung ber"ucksichtigt werden.
  Die derzeitige Linienklassifikation "uber das H"ohen--Breiten--Verh"altnis der umschreibenden
  Rechtecke ber"ucksichtigt diese nicht, deshalb ist es m"oglich, da"s benachbarte
  Tabellenzeilen in einem Textblock zusammengefa"st werden (s.\ \BildRef{SmearingTabelle}). 
  Ein m"oglicher Ansatz ist im Ausblick \KapRef{ZoningAusblick} beschrieben.
  
  \Bild{SmearingTabelle}{6}{Benachbarte Tabellenzeilen wurden zu einem Textblock zusammengefa"st}

  \item Linien, die bei der Digitalisierung zerfallen, werden in der Reglel nur teilweise 
  als Linie erkannt und aus dem Bin"arbild gel"oscht. 
  Bleiben Liniensegmente oder sonstige St"orungen zwischen Textspalten zur"uck, 
  k"onnen sich durch das Smearing Spalten Verbinden (s.\ auch \BildRef{HorizKleb} Seite 
  \pageref{eps:HorizKleb}).
  Ein vertikaler XY--Cut mit vorausgegangenem Filter (wie in \KapRef{HorKleb} beschrieben) 
  wird diese fehlerhaft segmentierten Bl"ocke meist wieder aufspalten. Problematisch ist
  hierbei die Wahl der Filterschwelle. Eine optimale L"osung w"are die Verallgemeinerung der
  Linienklassifikation auf zerfallene und punktierte Linien.

\end{itemize}

%%%%%%%%%%%%%%%%%%%%%%%%%%%%%%%%%%%%%%%%%%%%%%%%%%%%%%%%%%%%%%%%%%%%%%%%%%%%%%%%%%%%%%%%%%%%%%%%%%
\subsection{Verbesserungsm"oglichkeiten -- Ausblick}\label{ZoningAusblick}
Die in dieser Arbeit beschriebenen Methoden liefern Hypothesen f"ur homogene Textbl"ocke. Sie
bilden die Basis f"ur eine zuverl"assige Segmentierung in Zeilen, W"orter und Buchstaben. 

Aufgrund der geschilderten Probleme wurden folgende Verbesserungsm"oglichkeiten identifiziert:
\begin{itemize}
\item
Ein wichtiger Schritt um die Erkennungsleistung und die Robustheit sp"urbar zu erh"ohen, w"are
die Einf"ugung eines {\em Text--Grafik--Unterscheidungsmoduls\/}. Die Anforderung an ein solches
Modul w"are eine Filterung von Nicht--Textobjekten (St"orungen, Diagrammen und Photos).

\item Die Erkennungrate lie"se sich -- speziell bei komplexem Layout von Zeitschriften -- durch die
Ber"ucksichtigung der Blockkontur -- anstelle der umschreibeneden Rechtecke -- beim Zusammenfassen
steigern. Dazu w"are jedoch ein erheblicher Mehraufwand notwendig (z.B.\ f"ur die Berechnung
des Abstandes zweier Konturen).

\item
Um die Erkennung von {\em Linienelementen\/} zu verbessern, m"u"ste die Linienklassifikation, 
anstelle auf den umschreibenden Rechtecken, direkt auf der BCC--Kontur aufsetzen.
Es w"are folgender Ablauf denkbar:
Ausgehend von den Zusammenhangsgebieten wird ein Skelettbild erzeugt, indem von
au"sen und innen die schwarzen Zusammenhangsgebiete -- bis auf eine Pixelreihe --
schichtweise abgetragen werden. Diese einzelnen Punkte
bilden die St"utzstellen f"ur Polygone, wobei die gro"se Anzahl von St"utzstellen
durch einen geeigneten Approximationsalgorithmus (z.B.\ \cite{Wall83})
reduziert werden sollte. Eine geeignete Definition f"ur ein Ma"s, das die Kr"ummung
eines Polygons wiedergibt, w"are z.B.\ die Summe der Winkel angrenzender Polygone, normiert
auf die L"ange des Polygonzuges. Dadurch k"onnten Liniensegmente beliebigen Winkels, sowie
Linien innerhalb von Tabellen extrahiert werden.

Zus"atzlich k"onnte das Modul um eine Erkennung von zerfallenen ober punktierten Linien 
erweitert werden. Hierbei sei auf die Hough--Transformation verwiesen \cite{Illingworth88}.

\item
Die Konsistenzpr"ufung k"onnte zus"atzlich innerhalb jeder einzelnen Textzeile angewandt werden.
Dadurch w"are es m"oglich nichthomogenit"aten, beispielsweise von W"ortern, auch innerhalb einer
Textzeile, zu erkennen.

\Bild{NichtHomo}{10}{Nichthomogenit"at innerhalb einer Textzeile}

\item
Grunds"atzlich k"onnte die nachfolgende Zeilen--, Wort-- und Buchstabensegmentierung
ihre Vorteile aus den Textblock--Ergebnissen ziehen: z.B.\ m"u"sten f"ur diese Segmentierungsschritte
Werte wie Zeilenabstand, Zeichengr"o"se und Teilung nicht erneut berechnen, sondern k"onnten
auf die schon berechneten Informationen zur"uckgreifen.
Auch eine Erweiterung der OCR--Analyse ist denkbar:
Zur Zeit wird bei der Zeichenklassifikation ein universeller Klassifikator eingesetzt. Mit ihm
k"onnen proportionale und nichtproportionale Schriftarten wie Times und Courier erkannt werden.
Dieser allgemeine Ansatz ist aber innerhalb von Textbl"ocken nicht notwendig, da diese im
Idealfall nur einen Schrifttyp enthalten. Dadurch k"onnte nach einer Bestimmung des Schrifttyps
auf einen spezialisierten Klassifikator zur"uckgegriffen werden. Das Ergebnis w"are eine erh"ohte
OCR--Ge\-schwindig\-keit bei geringeren Fehlerraten.

%\item
%Zun"achst erscheint eine Optimierung der Ergebnisse durch Iteration sinnvoll:
%Die Dokumentsegmentierung und OCR--Analyse wird nach einer Variation der Parameter solange 
%durchlaufen, bis die Fehlerrate ein Minimum erreicht hat. 
%Dieses Verfahren wird an zwei Punkten scheitern: 
%Erstens k"onnen ohne Referenzdaten, die im realen Fall nicht vorliegen, die %?? eines 
%Ergebnisse (z.B.\ bzgl.\ ihrer Fehlerrate) nicht bewertet werden. 
%Zweites w"are ein intelligentes Verfahren zur Parametervariation nicht trivial,
%ganz abgesehen von der erforderlichen Rechenleistung.
\end{itemize}

Auch bei einer Realisierung aller dieser Verbesserungsm"oglichkeiten wird jedoch die 
Erkennungrate auf der Teststichprobe immer kleiner sein als auf der Lehrstichprobe. 
Die Ursache liegt darin, da"s w"ahrend des Trainings nie alle Probleme auftreten. 
Dies f"uhrt dazu, da"s sich die Verfahren 
innerhalb der Teststichprobe an unbekannten Sonderf"allen fehlerhalft verhalten.
Es gilt bei jeder Mustererkennungsaufgabe zu bedenken, da"s die Ergebnisse Hypothesen 
mit einer gewissen Unsicherheit darstellen. 
Erweiterungen und Verbesserungen k"onnen dies prinzipiell nicht entscheidend "andern. 

Der Aufwand bei der Erstellung und Optimierung der Segmentierungsverfahren mu"s dabei immer in
Relation zu dem Ziel der Dokumentanalyse gestellt werden. Wird nur eine 
Klassifikation des Textes zu einem Thema gefordert, kann es gen"ugen, einzelne W"orter 
mit der OCR--Analyse zu erkennen. 
Fehler durch st"orende Grafikeinfl"usse sind bei dieser Anwendung unproblematisch.
Soll dagegen das Dokument in eine Bibliotheksdatenbank aufgenommen werden,
mu"s das Ergebnis absolut fehlerfrei sein. Nur dann ist zum Beispiel eine Volltextsuche
realisierbar.
\newpage
\subsection*{Beispiele}

\Bild{HomoCheckErg}{10}{Ergebnisse der Konsistenzpr"ufung: Trennung bei unterschiedlicher Schriftgr"o"se (oben), bei einem Initial (Mitte), bei unterschiedlicher
Schriftst"arke (links) und auch bei "uberlappenden Textbl"ocken (rechts)}

\Bild{ZoningErg0}{16}{Ergebnis der Textblocksegmentierung bei einer Tageszeitung}

\Bild{ZoningErg1}{14}{Ergebnisse der Textblocksegmentierung: Gesch"aftsbrief (oben),
wissenschaftlicher Artikel (unten)}

\Bild{ZoningErg2}{10}{Segmentierungsergebnisse: Dokument mit Nicht-Manhattan Layout (oben), Zeitschrift mit Bl"ocken von beliebigem Konturverlauf (unten)}

\Bild{UlrichsProblemGeloest}{13}{Verbesserte Zeilensegmentierung durch vorgeschaltete
Textblocksegmentierung (zum Vergleich siehe \protect\BildRef{UlrichsProbleme} Seite \protect\pageref{eps:UlrichsProbleme})}



\chapter{Bestimmung der logischen Lesefolge}\label{Lesefolge}

Die Verfahren aus Kapitel \ref{BestimmungZones} liefern Hypothesen f"ur Textbl"ocke.
Damit der Text aus dem Dokument originalgetreu rekonstruiert werden kann, mu"s die der Segmentierung
folgende OCR--Analyse die Textbl"ocke in der richtigen Reihenfolge auswerten.

Diese Abfolge der Textbl"ocke mu"s somit der {\em logischen Lesefolge\/} innerhalb des
Dokuments entsprechen.
Somit besitzt jeder hypothetisierte Textblock seine -- wiederum hypothetische~-- Position
innerhalb der Menge der Textbl"ocke.

Die Sortierung erfolgt nach folgendem Schema \cite{Saitoh93}:
\begin{enumerate}
  \item Geometrische Beziehungen zwischen Textbl"ocken werden als hierarchische Beziehungen
  interpretiert und in eine Baumstruktur "ubertragen.
  \item Die Bl"ocke k"onnen nun einfach innerhalb des Baumes sortiert werden.
  \item Die Pre--Order Traversion\footnote{Zuerst wird die Wurzel des Baumes besucht, 
  anschlie"send die Teilb"aume} des Baumes ergibt die hypothetische Lesefolge.
\end{enumerate}

\section{Transformation der geometrischen Struktur}

Untersuchungen haben ergeben, da"s eine Vielzahl von Dokumenten hierarchisch aufgebaut ist:
Ein inhaltlich zusammenh"angender Artikel besteht meist aus einem \mbox{("Uberschrift--)} Textblock, der
darunterliegende, weniger breite Textbl"ocke "uberspannt, 
die wiederum "uber anderen stehen k"onnen usw.

Bei der Sortierung ist zu beachten, da"s sich die normale Lesefolge (von links nach rechts und
von oben nach unten) innerhalb hierarchisch angeordneter Textbl"ocke "andert 
(s.\ \BildRef{Einfluss}). Jeder Textblock "ubt somit einen {\em Einflu"s\/} auf die darunterliegenden
Textbl"ocke aus. Das Gebiet, in dem ein Textblock andere beeinflu"st, wird im folgenden
{\em Einflu"sbereich\/} genannt.

\Bild{Einfluss}{10}{Ber"ucksichtigung von hierarchischen Stukturen bei der Lesefolge}

Unter Beachtung der geometrischen Anordnung werden die Textbl"ocke in eine hierarchische
Baumstruktur "ubertragen. Ausgehend von einem alles "uberschreibenden (beeinflussenden)
virtuellen Textblock, werden Vater--Sohn-- und Bruder--Beziehungen definiert:

\begin{itemize}
  \item Befindet sich ein Textblock unter einem anderen, d.h. innerhalb dessen Einflu"sbereich,
  wird zwischen ihnen eine Vater--Sohn--Beziehung festgelegt. Der Einflu"sbereich wird von dem
  Vater--Textblock zu dem Sohn--Textblock vererbt (s.\ TB1 und TB2 in \BildRef{Einflussbereich}).
  
  Liegen direkt unter einem Textblock mehrere andere, so wird der Einflu"sbereich unter ihnen
  aufgeteilt (s.\ TB3 und TB4 in \BildRef{Einflussbereich}).
 
  \Bild{Einflussbereich}{4.5}{Vererbung und Aufteilung der Einflu"sbereiche}

  \item Liegen Textbl"ocke nebeneinander, d.h. es gilt
  $$Y_{1max} > Y_{2min} \quad \wedge \quad Y_{1min} < Y_{2max}
  \quad\wedge\quad 
  \left( X_{1max} < X_{2min} \quad \vee \quad X_{2max} < X_{1min} \right),$$
  
  wobei $X_{min}$ und $X_{max}$ der linken
  bzw.\ rechten, $Y_{min}$ und $Y_{max}$ der oberen bzw.\ unteren Kante eines Textblocks
  entspricht, definiert man in diesem Fall eine Bruder--Beziehung.
\end{itemize}

Diese Beziehungen lassen sich in eine Baumstruktur "ubertragen, wobei jedem Knoten ein Textblock
zugeordnet ist (s.\ \BildRef{BeziehungenInBaum}). 
Damit sich bei einer Pre--Order Traversion des Baumes die Lesefolge ergibt,
m"ussen die Textbl"ocke, zwischen denen eine Bruder--Beziehung besteht, 
d.h.\ die denselben Vaters besitzen, im Baum von links nach rechts, entsprechend ihrer
geometrischen Position, angeordnet werden.

\Bild{BeziehungenInBaum}{8}{"Ubertragen der geometrischen Beziehungen (links) in eine hierarchische Baumstruktur (rechts)}

Die bisherigen Definitionen und Regeln reichen bei komplexeren Dokumenten nicht aus und
m"ussen deshalb um folgenden Sonderfall erweitert werden:
Textbl"ocke, die sich nicht oder nur teilweise
in einem Einflu"sgebiet eines anderen befinden (TB 7 aus \BildRef{LesefolgeStrukturiert}), 
wurden bisher noch nicht ber"ucksichtigt.
Sie stellen eigenst"andige (unabh"angige) Bl"ocke dar und werden deshalb auf dieselbe
hierarchische Ebene wie die des virtuellen Textblocks gestellt, d.h.\ zwischen ihnen und dem
virtuellen Textblock wird eine Bruder Beziehung definiert. Da diese Textbl"ocke nicht
gezwungenerma"sen wie sonstige Br"uder--Textbl"ocke, nebeneinander liegen, m"ussen die Knoten nicht 
von rechts nach links sortiert werden. F"ur diese S"ohne der Wurzel m"ussen
weitere Sortierregeln aufgestellt werden.

\Bild{LesefolgeStrukturiert}{10}{Spezialfall: Textbl"ocke die nicht oder nur teilweise in einem Einflu"sgebiet liegen (TB 7) stellen unabh"angige Bl"ocke dar}

Die S"ohne der Wurzel werden nach folgendem
Schema sortiert. Zwei Knoten werden getauscht, wenn mindestens eine der Bedingungen
erf"ullt ist:
\begin{enumerate}
  \item Textblock des linken Knotens liegt im Einflu"sbereich des rechten Knotens (s.\ \BildRef{ErweiterteSorierung} links)
  \item $Y_{links} \; > \; Y_{rechts}$ und $X_{links} \; > \; X_{rechts}$ (s.\ \BildRef{ErweiterteSorierung} Mitte)
  \item Textblock des rechten Knotens liegt nicht unter dem Einflu"sbereich des linken Knotens
  und $X_{links} \; > \; X_{rechts}$ (s.\ \BildRef{ErweiterteSorierung} rechts),
\end{enumerate}
wobei der Punkt $(X,Y)$ die linke obere Ecke des zum Knoten geh"orenden Textblocks darstellt.

\Bild{ErweiterteSorierung}{13}{Erweiterte Sortierregeln f"ur die S"ohne der Wurzel}

\subsection*{Layoutbegrenzer}

Liniensegmente (s.\ \KapRef{Filterung}) und wei"ser horizontaler Zwischenraum 
(bestimmt durch einen horizontalen XY--Cut) dienen in einem Dokument als
Layoutbegrenzer und werden deshalb bei der Sortierung ber"ucksichtigt
(\BildRef{LayoutBegrenzer}). Liegt zwischen zwei Textbl"ocken ein solcher
Layoutbegrenzer, wird zwischen dem Textblockpaar keine Beziehung definiert.

\Bild{LayoutBegrenzer}{10}{Die urspr"ungliche Sortierreihenfolge (links) wird aufgebrochen, wenn Layoutbegrenzer zwischen Textbl"ocken liegen (rechts)}

\subsection*{Textrahmen}

Ein oft benutztes Gestaltungsmittel, um inhaltlich zusammenh"angende Textbereiche
zu kennzeichnen, ist die Verwendung von Textrahmen. Da Textbl"ocke innerhalb
dieser Rahmen dieselbe Komplexit"at aufweisen k"onnen, wie diejenigen innerhalb 
des normalen Dokuments, werden sie separat, d.h.\ wie ein eigenst"andiges Dokument, betrachtet.
Hierzu ist die
Baumstruktur geeignet zu erweitern: Jeder Textrahmen bekommt seine eigene Wurzel
zugewiesen, so da"s zuerst alle Textbl"ocke innerhalb der Textrahmen betrachtet werden,
und anschlie"send die verbleibenden.

\section{Bestimmung der Lesefolge aus den hierarchischen Beziehungen}

Die logische Lesefolge l"a"st sich durch eine Pre--Order Traversion der Baumes herauslesen.
Die Wurzeln und die virtuellen Textbl"ocke werden dabei nicht weiter beachtet
(s.\ \BildRef{Lesefolge}).

\Bild{Lesefolge}{12}{Ergebnis der logischen Textblocksortierung}

\section{Ergebnisse der Sortierung -- Ausblick}
Das dargestellte Verfahren beschreibt, wie aus der geometrischen Anordung von Textbl"ocken
deren hypothetische Lesefolge gewonnen werden kann.
Es zeigt sehr gute Ergebnisse bei klar stukturierten Dokumenten
(s.\ \BildRef{LesefolgeBeispiele} und \ref{eps:LesefolgeBeispiele2}
Seite \pageref{eps:LesefolgeBeispiele2} und \pageref{eps:LesefolgeBeispiele}).
Da als Eingangsdaten jedoch ausschlie"slich die umschreibenden Rechtecke der Textbl"ocke benutzt werden, sind Fehler durch Mehrdeutigkeiten unvermeidbar.
Die Verfahren zur Sortierung stellten nicht den Schwerpunkt dieser Arbeit dar. Aus diesem Grund,
wurde nur die grunds"atzliche Eignung der Verfahren "uberpr"uft,
jedoch die Erkennungsrate nicht qualitativ gemessen. 
Dazu w"are analog zu der Berechnung der Zoning--Erkennungsraten 
(s.\ \KapRef{ErgebnisZoning}) wiederum eine aufwendige Erstellung von Referenzdaten samt
Auswertung n"otig gewesen, die den zeilichen Rahmen dieser Arbeit gesprengt h"atte.

Nur aufgrund der geometrischen Verh"altnisse auf die Lesefolge zu schlie"sen, stellt sich,
selbst f"ur einen menschlichen Betrachter, als schwieriges Problem dar. Der Leser kann sich
dies verdeutlichen, wenn der die linke Bildh"alfte von Abbildung \ref{eps:LesefolgeProblem} 
betrachtet und dabei die rechte verdeckt. 
Der Mensch kann aufgrund des inhaltlichen Zusammenhangs auf den
logisch n"achsten Block schlie"sen -- er "uberfliegt die ersten Zeilen der in Frage kommenden
und w"ahlt intuitiv den richtigen aus. Je ausgefallener und komplexer das Layout, desto
schwieriger die Aufgabe. 
So wird deutlich, da"s die Dokumentmenge auf klar strukturierte Dokumente
(Gesch"aftsbriefe, wissenschaftlichen Artikel, Zeitschriften) beschr"ankt werden mu"ste.
Bei Steuerbescheiden tritt die Frage auf, welche Abfolge innerhalb einer Tabelle "uberhaupt 
die richtige ist.

\Bild{LesefolgeProblem}{12}{Die umschreibenden Blockrechtecke als Ausgangsdaten (links) f"ur das Sortierverfahren -- selbst f"ur den menschlichen Betrachter ist die Bestimmung der Lesefolge ohne Zuhilfenahme des Originals nicht einfach}

Folgende Probleme sind aufgetreten:
\begin{itemize}
  \item Aufgrund der umschreibenden Rechtecke kann nicht immer eindeutig auf eine Lesefolge
  geschlossen werden (s.\ \BildRef{LSZweideutigkeiten}). Die Sortierung erfolgt nach der
  ersten M"oglichkeit, obwohl bei inhaltlicher Betrachtung die zweite
  vielleicht die richtige w"are (\BildRef{LSFehler} links).
  
  \Bild{LSZweideutigkeiten}{8}{Lesefolge l"a"st sich aufgrund der Eingangsdaten (links) nicht eindeutig bestimmen}
  
  Um diese Fehler zu vermeiden, m"u"sten Regeln eingef"uhrt werden, die zus"atzlich
  zu der geometrischen Anordnung der Textbl"ocke, deren Inhalt ber"ucksichtigen 
  (z.B.\ Buchstabengr"o"se). Eine fehlerfreies Ergebnis wird man nur erhalten, wenn der Text
  gelesen und verstanden wurde.

  \item Textbl"ocke unter oder "uber Tabellen und Grafiken 
  werden direkt in den Textflu"s einbezogen (\BildRef{LSFehler} rechts). Um dies zu verhindern,
  m"u"sten Textbl"ocke unter bzw.\ "uber Nicht--Textgebieten als Bildunter-- 
  bzw.\ Bild"uberschriften
  klassifiziert und bei der Sortierung separat betrachtet werden. Hierzu m"u"sten zuvor
  Text-- und Nicht--Textgebiete erkannt und getrennt werden.
\end{itemize}

\Bild{LSFehler}{9.5}{Fehler bei der Bestimmung der Lesefolge: Mehrdeutigkeiten (links) und Einbeziehung von Bildunterschriften in den Textflu"s (rechts)}

Eine entscheidende Voraussetzung f"ur die Richtigkeit der Sortierung ist ein gutes Ergebnis aus der
Textblocksegmentierung. Traten hier schon Fehler auf, werden sich diese
bei der Sortierung weiter fortsetzen. Somit gelten die bei der Segentierung genannten
Verbesserungsans"atze auch bei der Sortierung.

Grunds"atzlich w"are jede neue auf Sonderf"alle angepa"ste, Sortierungsregel nur eine unter
vielen anderen denkbaren. 

\Bild{LesefolgeBeispiele2}{11}{Beispiele f"ur die Sortierung (1)}
\Bild{LesefolgeBeispiele}{13}{Beispiele f"ur die Sortierung (2)}


\chapter{Zusammenfassung}

In dieser Diplomarbeit wurde ein Verfahren vorgestellt, welches ein Dokumentenbild
in Textbl"ocke segmentiert. Die Textblockdefinition basiert
auf den geometrischen Eigenschaften der in r"aumlicher N"ahe stehenden Zusammenhangsobjekte.
Durch diese Textblocksegmentierung wurde erreicht, da"s nachfolgende
Segmentierungsschritte (Zeilen, W"orter und Buchstaben) in ihrer Erkennungsleistung
verbessert wurden, da ihnen nun als Eingangsdaten Textbl"ocke homogenen Inhalts zur Verf"ugnug
stehen.

Die verwendeten Methoden machen keine Einschr"ankungen bez"uglich der Layoutstruktur
und der verwendeten
Dokumentklasse. Die jeweiligen Parameter sind layoutunabh"angig bzw.\ passen sich 
automatisch dem Dokument an. Mehrspaltige Layouts mit beliebiger Kontur der Textbl"ocke
werden ber"ucksichtigt. Es wird vorausgesetzt, da"s die Dokumente nur 
Textgebiete enthalten, eine Winkelkorrektur bereits vorgenommen wurde und da"s sie aus dem
romanischen Sprachraum entstammen, d.h.\ die Schreibrichtung verl"auft von links nach rechts und von
oben nach unten.

Das Verfahren kn"upft an zwei bekannte Vorgehensweisen an: 
Der Top--Down Ansatz (XY--Cut) versucht das Dokument rekursiv, entlang wei"sem 
Zwischenraum zu trennen, und der Top--Down Ansatz (Smearing) 
verbindet kleine Objekte zu gr"o"seren Bereichen. Um die Smearing Parameter unabh"angig von dem
verwendeten Layout zu halten, werden sie, f"ur jedes beim XY--Cut entstandenen Gebiet separat,
"uber den Zeilenabstand direkt aus dem Dokument berechnet. Das Smearing wird mit der
morphologischen Operation Dilatation mit anschlie"sender Erosion (Closing) durchgef"uhrt. Um
die Datenmenge bei diesem nichtlinearen Filter zu reduzieren, wird zuvor das Bin"arbild in
seiner Aufl"osung um den Faktor acht in der Vertikalen und Horizontalen reduziert.

Im Gegensatz zu bisherigen Verfahren wird in diesem Ansatz die Konsistenz der Textbl"ocke "uberpr"uft: 
Textbl"ocke sind definiert als eine Menge von homogenen Zusammenhangsgebieten. Da die
morphologischen Operationen alle Objekte verbinden, die in r"aumlicher N"ahe stehen, mu"s die
Homogenit"at innerhalb der hypothetischen Textbl"ocke "uberpr"uft werden. Hierbei wird
die mittlere Buchstabengr"o"se und --st"arke jeder Textzeile bestimmt und mit der ihr folgenden
verglichen und ggf.\ der Textblock zwischen den zwei Zeilen getrennt.

In einem letzten Segmentierungsschritt wird versucht Textbl"ocke zu verbinden, um eine m"oglichst
kleine Menge von Textbl"ocken zu erhalten. Dabei wird die Graphenstruktur des Minimum Spanning
Tree nach der Methode von Kruskal aufgebaut und anschlie"send Kante f"ur Kante abgearbeitet.
Dieser Schritt wird solange wiederholt, bis ein station"arer Zustand erreicht ist
(Relaxationsmethode).

Im Anschlu"s werden die gefundenen Textbl"ocke nach ihrer logischen Lesefolge sortiert. Dabei
wird eine Baumstruktur aufgebaut, die die hierarchische Struktur der Textbl"ocke
beinhaltet. Bei der Sortierung werden Liniensegmente, wei"se horizontale Zwischenr"aume und Textrahmen ber"ucksichtigt.
Die Lesefolge ergibt sich aus der Pre--Order Traversion der Baumstruktur.

Das Verfahren wurde anhand einer Lernstichprobe von 47 Dokumenten sowie an einer
Teststichprobe von 116 Dokumenten, bei der die Leistungsf"ahigkeit und Grenzen aufgezeigt wurden, 
erarbeitet.
Die implementierten Methoden zeigen konstant gute Ergebnisse (Erkennungsrate 96\%) 
bei den verschiedensten Layouts (Gesch"aftsbriefe, wissenschaftliche Artikel, Zeitschriften,
Tageszeitungen und Steuerbescheide). 
Probleme bereiteten St"orungen durch Nicht--Textelemente. 
Die L"osung dieses Problems w"are ein vorgeschaltetes 
Text--Grafik--Unterscheidungs Modul, 
wobei bis zu 5\% h"ohere Erkennungsraten erreicht werden k"onnten.
Die Laufzeit der kompletten Textblocksegmentierung (XY--Cut, Smearing, Konsistenzpr"ufung und
Sortierung) ist anh"angig von der Anzahl der Zusammenhangsgebiete und 
betr"agt zwischen ca.\ 2 Sekunden f"ur einen Gesch"aftsbrief und ca.\ 11 Sekunden 
f"ur DIN--A4--gro"se Ausschnitte aus Tageszeitungen (gemessen auf einer HP--PA Workstation).

Die Algorithmen wurden in ANSI--C implementiert und in eine bestehende Softwareumgebung
integriert. 



%
\begin{appendix}
  \chapter{Experimentierumgebung}\label{Experimentierumgebung}

\section{Testumgebung}
\subsection*{Hardware und Betriebssystem}
Die Experimentierumgebung wurde auf einer NeXT--Workstation unter dem Betriebssystem
NEXTSTEP entwickelt und implementiert. Die NeXT--Workstations basieren auf
einem Motorola MC68040 Prozessor, der entweder mit 25MHz oder in der Turbo-Version mit
33MHz getaktet wird. Die Spitzenleistung nach dem LINPACK--Test betr"agt bei der 25MHz
Version 15 MIPS und 2 MFLOPS. Da sich NeXT seit 1992 aus dem Hardwaregesch"aft zur"uckgezogen
hat werden diese Maschinen zunehmend durch leistungsf"ahigere Workstations 
anderer Hersteller ersetzt. Die Firma NeXT widmet sich nun ausschlie"slich der Weiterentwicklung
ihres Betriebssystems NEXTSTEP samt Entwicklungsumgebung und Portierungen auf 
andere Prozessoren (Intel 80x86, SPARC, HP-PA RISC).

Das besondere Merkmal von NEXTSTEP ist die grafische Benutzeroberfl"ache. Sie macht alle
wichtigen Dienste und Applikationen durch Anw"ahlen von Symbolen und Men"us zug"anglich. Ein
weiterer gro"ser Vorteil liegt in der schnellen Grafikausgabe durch Display--PostScript, die
dieses System zum idealen Werkzeug f"ur Bild\-ver\-arbeit\-ungs\-auf\-gaben macht.

\subsection*{NeXT--Entwicklungsumgebung}
Kernpunkte der NeXT--Entwicklungsumgebung sind zum einen der Compiler f"ur die Programmiersprache
{\em Objective--C\/} und zum anderen die beiden Werkzeuge {\em Project--Builder\/} und 
{\em Interface--Builder\/}.

Objective--C ist eine von Brad J.\ Cox vorgeschlagende Erweiterung der Programmiersprache ANSI--C,
die diese zu einer vollst"andigen objektorientierten Sprache macht. Cox stellte sich die Frage was
an einer `herk"ommlichen' Programmiersprache erweitert oder ver"andert werden mu"s, um daraus eine 
objektorientierte zu machen. Er kam zu folgendem Ergebnis: Ben"otigt wird
\begin{itemize}
  \item ein Datentyp f"ur Objekte,
  \item eine Syntax zur Beschreibung von Klassen,
  \item eine syntaktische Struktur zum Versenden von Nachrichten,
  \item eine entsprechende Laufzeitunterst"utzung.
\end{itemize}
Die Idee des objektorientierten Programmierens besteht darin, ein Programm aus einer Menge von 
{\em Objekten\/} aufzubauen. F"ur die unterschiedlichsten Zwecke gibt es bestimmte Typen 
von Objekten, auch {\em Klassen\/} genannt. 
Die Klassendefinition bestimmt die Eigenschaften aller zugeh"origen Objekte.
Ein Objekt besteht aus Attributen und Operationen. Erstere sind die Daten, auch 
{\em Instanzvariablen\/} genannt. Der Zugriff auf die Instanzvariablen eines Objekts erfolgt 
ausschlie"slich "uber den Aufruf einer seiner Operationen ({\em Methoden\/}).
Ein solcher Aufruf erfolgt durch das Versenden einer {\em Nachricht\/} an das Objekt, mit der
Aufforderung, die Methode gleichen Namens auszuf"uhren. Objekte kommunizieren nur
durch Nachrichtenaustausch miteinander, um gemeinsam die Problemstellung des Programms zu l"osen.
F"ur eine ausf"uhrliche Beschreibung sei auf \cite{ct94} verwiesen. 

Das Herzst"uck der Softwareentwicklung unter NEXTSTEP ist der Project--Builder. Er ist f"ur die
Verwaltung alle Komponenten einer Applikation (haupts"achlich Quelltexte) in einem Projekt zust"andig
und erlaubt den schnellen Zugriff auf weitere Entwicklungswerkzeuge, wie etwa zur Fehlersuche 
(Debugger) und zur Quelltext"ubersetzung (Compiler).

Die NeXT--Rechner typische Benutzeroberfl"ache l"a"st sich komfortabel mit dem Interface--Builder erstellen.

\subsection*{APRES--Softwareumgebung}

APRES steht f"ur {\sl A\/}dvanced {\sl P\/}attern {\sl R\/}ecognition {\sl E\/}ngineering
{\sl S\/}oftware und hat zum Ziel, eine ad"aquate Entwicklungs-- und Experimentierumgebung
im Bereich Mustererkennung zu bieten.

F"ur die allermeisten Programmsysteme gilt, da"s die Implementierung der eigentlichen Algorithmen
nur einen kleinen Bruchteil das Gesamtsystems darstellt. Der Rest ist f"ur Datenhaltungsaspekte,
Interaktionsaufgaben und f"ur die Koordinierung des Ablaufs zu\-st"an\-dig.
Ein wesentlicher Punkt einer Entwicklungsumgebung ist die Datenhaltung. Die
Trennung zwischen Datenhaltung und Algorithmenimplementierung ist zu bevorzugen, da sich dann
die Wartbarkeit der Programme erh"oht und sich die Algorithmen leichter nachvollziehen lassen.

An die Datenhaltung werden folgende Anforderungen gestellt:
\begin{itemize}
  \item Ablegen der Daten nach Beendigung des Programmes, bzw.\ Bereitstellung der Daten f"ur
  andere Programme
  \item Modellierung von semantischen Zusammenh"angen (Beziehungen) zwischen Datenstrukturen 
  (Einf"uhrung einer Strukturierung)
  \item M"oglichkeit zur "Uberpr"ufung der Richtigkeit der Daten
  \item Versionierung der Datens"atze
\end{itemize}

Um diese Forderungen zu erf"ullen, bietet APRES eine objektorientierte Datenverwaltung. Jedem Objekt
ist zun"achst ein Identifikator (ID--Nummer) zugeordnet. Eine Benennung kann wahlweise 
auch durch einen aussagekr"aftigen Aliasnamen (entspricht einer symbolischen Zeichenkette) erfolgen.
Der Identifikator eines Objektes ist f"ur die
gesamte Datenhaltung eindeutig. Nat"urlich lassen sich jedem Objekt passende
Nutzdaten zuordnen. Diese Nutzdaten beschreiben in Form verschiedener Datenstrukturen bestimmte
Aspekte oder Attribute von Objekten. Beispiele dieser Datenstrukturen (Tuple genannt) sind
"`BOX"' zum Ablegen eines Rechtecks, "`INTEGER"' f"ur einen Interger--Wert
und "`BIN\_IMAGE\_TUPLE"' f"ur die Aufnahme eines
Bin"arbildes. Zu jeder Datenstruktur gibt es einen Statz von standardisierten Zugriffsfunktionen,
welche eine einfache und schnelle Handhabung der Nutzdaten zulassen.

Beziehungen zwischen zwei Objekten lassen sich "uber eine Relation ({\em Link\/} genannt)
mit einem bestimmten {\em Linktype\/} aufbauen. Diese Typen sind "uber ihre Aliasnamen definiert.
Beispiele hierf"ur sind "`is\_a"' und "`is\_part"'. Mit dem "`is\_a"'--Type wird einem Objekt eine
Bedeutung zugeordnet.
So hat z.B.\ ein Objekt, das eine Textseite repr"asentiert, einen "`is\_a"'--Link zu einem
Klassenobjekt mit dem Aliasnamen "`PAGE"'. Die "`is\_a"'--Beziehung kann mehrfach vergeben, 
und somit k"onnen einem Objekt mehrer Bedeutungen zugeordnet werden. Der "`is\_part"'--Type kennzeichnet 
eine hierarchische Struktur, wie z.B.\ eine Teilebeziehung zwischen Seite und Gesamtdokument 
(s.\ \BildRef{APRESLinks}).

\Bild{APRESLinks}{11}{Beispiel einer Datenmodellierung in APRES}

Abbildung \ref{eps:APRESvorZoning} zeigt den Datenbankzustand nach der BCC--Analyse. Der
Zustand der Datenbank nach der Bestimmung der Textbl"ocke und der Lesefolge ist in 
Abbildung \ref{eps:APRESnachZoning} ersichtlich.

Hinter dem APRES--Kontextkonzept verbirgt sich ein Schichtenmodell der Datenbank, welches es
erm"oglicht bestimmte Zust"ande (Versionen oder Instanzen) der Datenbank zu 
konservieren und sp"ater wieder herzustellen. 
Jeder gesicherte Datenbankzustand entspricht einem Kontext. 
Zu Beginn eines Experiments befindet man sich in dem Wurzel--Kontext. 
Anschlie"send besorgt man sich Einsicht 
zu den notwendigen Informationen, indem man eine Verbindung zu denjenigen Kontexten schafft, die diese 
Informationen enthalten. Man hat nur Zugriff zu den Daten des Kontextes, zu dem man eine
Verbindung geschaffen hat. Dazu geh"oren die Daten, die in diesem Kontext erzeugt wurden, aber
auch alle Daten der Vaterkontexte (Vererbungsprinzip).

In der Dokumentanalyse korrespondiert ein Kontext meist mit dem Abschlu"s eines gr"o"seren 
Verarbeitungsschrittes (BCC--Analyse, Zoning, Klassifikation, etc.).

\Bild{APRESvorZoning}{11}{Datenbankzustand nach der BCC--Analyse}

\Bild{APRESnachZoning}{15}{Datenbankzustand nach der Bestimmung der Textbl"ocke und der Lesefolge}

\clearpage
\section{Generisches CALtool}
Um die Ergebnisse aus dem Verfahren zur Bestimmung von Textbl"ocken beurteilen zu k"onnen, 
m"ussen die berechneten -- hypothetischen -- Paragraphen mit einer Referenz verglichen werden.
Das bedeutet, da"s das Datenmaterial von menschlicher Seite gesichtet und
vorklassifiziert ({\em gelabelt\/}) wurde. In unserem Fall mu"s der Bearbeiter die Dokumente
in (rechteckige) Bereiche einteilen. Hierbei kann er zwischen zwei Namen ({\em Labels\/})
ausw"ahlen: Textbereiche und Nicht--Textbereiche. Um diesen Aufwand so gering wie m"oglich zu
halten, empfiehlt es sich, den Proze"s computerunterst"utzt ablaufen zu lassen. Ein
Programm zur Unterst"utzung beim Labeln, das CALtool ({\em C\/}omputer 
{\em A\/}ided {\em L\/}abeling), bietet sich an.
Dieses CALtool bietet generische Rahmenmodule, die f"ur eine Vielzahl von
Anwendungsbereiche gleicherma"sen einsetzbar sind. Dadurch m"ussen immer wiederkehrende
Aufgaben  nicht jedesmal neu implementiert werden. F"ur eine ausf"uhrliche
Spezifikation sei auf \cite{CALtool} verwiesen.

Mit Hilfe diese Rahmens war es in kurzer Zeit m"oglich, eine eigene
auf die Klassifikation von Text-- und
Nicht--Textgebieten spezialisierte CALtool--Anwendung zu erzeugen. Die Anforderungen an das
Zoning--CALtool waren folgende:
\begin{itemize}
  \item Laden von einem oder mehreren Dokumenten aus der Datenbank oder aus entsprechenden 
  Datenfiles.
  \item Grafisch unterst"utzte Markierung von Bereichen innerhalb eines Dokuments.
  \item Benennung der Bereiche als `Text' oder `Nicht--Text'.
  \item Laden der hypothetischen Textbl"ocke als Labelhilfe.
  \item Ablegen der gelabelten Bereiche in Textdateien zur weiteren Auswertung.
\end{itemize}

%Abbildung \ref{eps:CALtool} zeigt die Oberfl"ache der CAltool Anwendung. 

\Bild{CALtool}{16}{Die grafische Oberfl"ache der CALtool Anwendung}
\clearpage
\section{Die Applikation ZoningLab}
Die Applikation {\em ZoningLab\/} dient zum Testen der Methoden zur Layoutanalyse. Sie stellt
eine komfortable Benutzeroberfl"ache zur Verf"ugung, inklusive APRES--Datenbankzugriff, 
Parametrierung und Starten der Verfahren sowie Visualisierung der Ergebnisse. S"amtliche in
Kapitel \ref{BestimmungZones} und \ref{Lesefolge} beschriebenen Verfahren wurden in das
{\em ZoningLab\/} integriert.

\Bild{ZoningLab}{16}{Die grafische Oberfl"ache des ZoningLabs}
\clearpage
\section{Parameterpanel}\label{ParameterPanel}

\BildHier{ParameterPanel}{9}{Parametereingabe f"ur das Zoning}

  \chapter{Dokumentklassen}\label{DokuKlassen}
\Bild{BspGeschWiss}{10}{Gesch"aftsbrief (oben) und wissenschaftlicher Artikel (unten)}
\Bild{BspZeitschZeitung}{13}{Zeitschrift (oben) und Tageszeitung (unten)}
\Bild{BspSteuerBILD}{13}{Steuerbescheid}

  \addcontentsline{toc}{chapter}{Literaturverzeichnis}
\begin{thebibliography}{9999}

\bibitem[Bartneck90]{Bartneck90} N.~Bartneck, "`Methods for photo noise extraction
  in postal applications,"' in {\em Proc.~Advanced Technology
  Conference}, USPS, 1990, pp.~297--310.

\bibitem[Sch"urmann92]{PixelToContents} J.~Sch"urmann, N.~Bartneck, T.~Bayer,
  J.\ Franke, E.\ Mandler, M.\ Oberl"ander, "`Document Analysis--From
  Pixel to Contents,"' in {\em Proc.~of the IEEE}, Vol.~80, No.~7,
  July 1992, pp.~1101--1119.
  
\bibitem[Bayer93]{InfoPortLab} T.~Bayer, U.~Bohnacker, H.~Mogg--Schneider,
  "`InfoPortLab --- An Experimental Document Understanding System,"' in
  {\em Proc. of the 2nd Int.~Conf.~on Document Analysis and
    Recognition (Japan)}, 1993, pp.~297--312.

\bibitem[MandlerOberl"ander90]{MaOb90} E.~Mandler, M.~F.~Oberl"ander, "`Ein single--pass
  Algorithmus f"ur die schnelle Konturkodierung von Bin"arbildern,"' in
  {\em Proc.~12th DAGM -- Symposium Mustererkennung}, Aalen, 1990, pp.~248--255.
  
\bibitem[Bartneck87]{Bartneck87} N.~Bartneck, "`Ein Verfahren zur Umwandlung der
  ikonischen Bildinformation digitalisierter Bilder in Datenstrukturen
  zur Bildauswertung,"' Dissertation, TU Braunschweig, 1987.
  
\bibitem[Bohnacker93]{Bohnacker93} U.~Bohnacker, "`Aufbau einer Methodenbank zur
  Segmentierung von Textbereichen auf der Basis von
  Zusammenhangsgebieten,"' Diplomarbeit, FH Ulm, 1993.
  
\bibitem[Sch"urmann77]{Schuermann77} J.~Sch"urmann, "`Polynomklassifikation f"ur die
  Zeichenerkennung,"' ISBN 3--486--21371--7, Oldenburg Verlag
  M"unchen, 1977.
  
\bibitem[Niemann83]{Niemann83} H.~Niemann, "`Klassifikation von Mustern,"'
  ISBN 3--540--12642--2, Springer Verlag Berlin, 1983, pp.~54--59.

\bibitem[Wahl82]{Wahl} F.~M.~Wahl, K.~Y.~Wong, R.~G.~Casey, "`Block
  Segmentation and Text Extraction in Mixed Text/Image Documents,"' in
  {\em Computational Graphics and Image Processing}, Vol.~20, 1982,
  pp.~375--390.
  
\bibitem[Wieser93]{Wieser93} J.~Wieser, A.~Pinz, "`Layout and Analysis: Finding
  Text, Titles and Photos in Digital Images of Newspaper Pages,"' in
  {\em Proc. of the 2nd Int.~Conf.~on Document Analysis and
    Recognition (Japan)}, 1993, pp.~774--777.
  
\bibitem[Nagy84]{Nagy84} G.~Nagy, S.~Seth, "`Hierarchical Representation of
  Optically Scanned Documents,"' in {\em Proc.~of the IEEE 7th
    Int.~Conf.~Pattern Recognition (Montreal, Canada)}, 1984,
  pp.~347--349.
  
%\bibitem[Cullen93]{Cullen93} J.~F.~Cullen, K.~Ejiri, "`Weak model--dependent page segmentation
%  and skew correction for processing document images,"' in {\em Proc. of the 2nd Int.~Conf.\ on
%  Document Analysis and Recognition (Japan)}, 1993, pp.~757--???.

\bibitem[Baird92]{Baird92} H.~S.~Baird, "`Background Structure in Document Images,"' in
  {\em Proc.\ IAPR Workshop on Structural and Syntactic Pattern Recognition (SSPR'92)}, Bern,
  Schweiz, 1992.

\bibitem[Okamoto93]{Okamoto93} M.~Okamoto, M.~Takahashi, "`A Hybrid Page Segmentation Method,"'
  in {\em Proc. of the 2nd Int.~Conf.~on Document Analysis and
  Recognition (Japan)}, 1993, pp.~743--748.

\bibitem[Kuropka90]{Kuropka90} L.~Kuropka, T.~Bayer, "`Untersuchungen zur Strichdickennormierung
  als Mittel zur Verbesserung der Erkennungsleistung von Schrifterkennungssystemen,"' in {\em
  Technischer Bericht}, Daimler Benz, Nr.\ FAU 007/89, 1990, pp.\ 9--14.

\bibitem[Saitoh93]{Saitoh93} T.~Saitoh, M.~Tachikawa, T.~Yamaai, "`Document Image 
  Segmentation and Text Area Ordering,"' in
  {\em Proc. of the 2nd Int.~Conf.~on Document Analysis and
    Recognition (Japan)}, 1993, pp.~323--329.
    
%\bibitem[Tsujimoto92]{Tsujimoto} S.~Tsujimoto, H.~Asada, "`Major Components of a Complete Text
%  Reading System,"' in {\em Proc.~of the IEEE}, Vol.~80, No.~7, July 1992, 
%  pp.~1133--1149.

\bibitem[Sedgewick92]{Algorithmen} R.~Sedgewick, "`Algorithmen,"' ISBN 3--89319--301-4,
  Addison--Wesley Verlag Bonn, 1992.

\bibitem[Droege94]{ct94} D.~Droege, "`Objektiv betrachtet. Objective--C --- das andere
  objektorientierte C,"' in {\em c't -- Magazin f"ur Computer Technik\/}, Nov.\ 1994, pp.~278--286.

\bibitem[Wall83]{Wall83} K.~Wall, P.--E.~Danielsson, "`A New Method for Polygonal
  Approximation of Digitized Curves,"' in {\em Proc.\ of the 3rd Scandinavian Conf.\ on
  Image Analysis\/}, 1983.

\bibitem[Illingworth88]{Illingworth88} J.~Illingworth, J.~Kittler, "`A Survey of the Hough
  Transform,"' in {\em Computer Vision, Graphics and Image Processing\/}, No.\ 44, 1988,
  pp.~87--116.

\bibitem[Gloger94]{CALtool} J.~Gloger, "`Spezifikation, Generisches CALtool, Computer Aided
  Labeling,"' Daimler--Benz, Textverstehende Systeme (F3T), 1994.

\end{thebibliography}
\clearpage
\rhead[]{\small \em Danksagung}
\lhead[\small \em Danksagung]{}
{\em Danksagung:}

Mein Dank gilt Herrn Prof.\ Dr.--Ing.\ A.\ Rothermel f"ur seine Bereitschaft, als Pr"ufer f"ur eine
externe Diplomarbeit zur Verf"ugung zu stehen. Besonderen Dank an meinen Betreuer Herrn 
Dr.\ Thomas Bayer, der mir mit seinem umfangreichen Fachwissen und Erfahrungsschatz jederzeit
zur Seite stand. Herrn Ulrich Bohnacker danke ich f"ur die vielen Detailerkl"arungen und
die immer konstruktive Kritik. Desweiteren gilt mein Dank Herrn Joachim Gloger und Herrn 
Henrik Baur, die mir ihr CALtool, nach einer gr"undlichen Einweisung, zur Verf"ugung stellten. 
Last but not least Herrn Armin Jahn f"ur seine `kurzen' Perl-Scrip-Hacks und Herrn Axel Braun 
f"ur die Unterst"utzung bei der Visualisierung meiner Ergebnisse.

\end{appendix}
%
\end{document}